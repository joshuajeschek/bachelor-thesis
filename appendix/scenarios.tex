\chapter{Bildschirmfotos Szenarios}
\label{app:scenarios}

Im Folgenden werden für jedes Szenario drei Bildschirmfotos aufgelistet: Der Ausgangszustand des Editors, ein korrekt ausgefüllter Endzustand und ein Ausschnitt der daraus dazugehörigen Ergebnistabelle.

% screenshot graphic
\newcommand{\scgraphic}[3][]{%
  \includegraphics[width=\textwidth]{assets/sc-#2.png}
  #1
  \caption{#3}}

% compact screenshot graphic
\newcommand{\cscgraphic}[2]{%
  \scgraphic[\vspace{-10mm}]{#1}{#2}}

% screenshot figure
\newcommand{\scfigure}[2]{%
  \begin{figure}[hbtp]
    \centering
    \scgraphic{#1}{#2}
  \end{figure}}

\scfigure{1-start}{Ausgangspunkt von Szenario 1.}
\scfigure{1-end}{Mögliche Lösung für Szenario 1.}
\scfigure{2-start}{Ausgangspunkt von Szenario 2.}
\scfigure{2-end}{Mögliche Lösung für Szenario 2.}
\scfigure{3-start}{Ausgangspunkt von Szenario 3.}
\scfigure{3-end}{Mögliche Lösung für Szenario 3.}
\begin{figure}[hbtp]
  \centering
  \cscgraphic{1-result}{Ergebnistabelle von Szenario 1.}
  \vspace{4mm}
  \cscgraphic{2-result}{Ergebnistabelle von Szenario 2.}
  \vspace{4mm}
  \cscgraphic{3-result}{Ergebnistabelle von Szenario 3.}
\end{figure}
