\documentclass[a4paper, 12pt, oneside, BCOR=1cm,toc=chapterentrywithdots]{scrbook}

\usepackage{graphicx}                   % use for pdfLatex
\usepackage{makeidx}                    % \printindex
\usepackage[colorlinks=false]{hyperref}
% \usepackage{tocbibind}
\usepackage{blindtext}
\usepackage{subfigure}
\usepackage{acronym}
\usepackage[shorthands=off,ngerman]{babel}
\usepackage[utf8]{inputenc}
\usepackage[T1]{fontenc}
\usepackage{csquotes}
\usepackage[style=apa,backend=biber]{biblatex}
\usepackage{xcolor}
\addbibresource{bibliography.bib}

\defbibfilter{articles}{
  type=article or
  type=inproceedings
}

% \DefineBibliographyStrings{ngerman}{
%   retrieved = {abgerufen am},
%   from      = {von},
% }

\hypersetup{bookmarksnumbered=true,bookmarksopen=true,bookmarksopenlevel=1,pdfborder=0 0 0}

% remove this later
\setcounter{tocdepth}{4}


% könnte zu Problemen führen
\MakeOuterQuote{"}

\newcounter{todocounter}
\setcounter{todocounter}{100}
\newcommand{\todo}[1]{\refstepcounter{todocounter}\textcolor{red}{\footnote[\thetodocounter]{\textcolor{gray}{TODO: #1}}}}

\begin{document}
\begin{titlepage}

  \begin{center}
    \raisebox{-1ex}{\includegraphics[scale=1.5]{assets/tuc.pdf}}\\
  \end{center}
  \vspace{0.5cm}

  \begin{center}

    \LARGE{\textbf{Title of the bachelor thesis}}\\
    \vspace{1cm}


    \Large{\textbf{Bachelor Thesis}}\\
    \vspace{1cm}
    Submitted in Fulfilment of the\\
    Requirements for the Academic Degree\\
    B.Sc.\\
    \vspace{0.5cm}
    Dept. of Computer Science\\
    Chair of Computer Engineering
  \end{center}
  \vspace{3cm}
  Eingereicht von: Joshua Jeschek\\
  Matrikelnummer: 652272\\
  Datum: 24.12.2023\\
  \vspace{0.3cm}\\
  Dr. Thomas Wilhelm-Stein \\
  Dr. Heino Rudolf

\end{titlepage}

\addchap*{Abstract}
\vspace*{0pt plus 1fill}
\begin{tucsimplesection}{Abstract}
  \todo[inline]{\blindtext}
\end{tucsimplesection}
\vspace*{0pt plus 2.5fill}


\tableofcontents

\begingroup
\cleardoublepage
\addcontentsline{toc}{chapter}{\listfigurename}
\listoffigures

\let\clearpage\relax

\cleardoublepage
\addcontentsline{toc}{chapter}{\listtablename}
\listoftables
\endgroup

\twocolumn
\addchap{Abkürzungsverzeichnis}
\begin{acronym}
  \acro{TA}{Think-Aloud-Protokoll}
  \acro{CTA}{Concurrent Think-Aloud-Protokoll}
  \acro{RTA}{Retrospective Think-Aloud-Protokoll}
\end{acronym}

\onecolumn

\chapter{Einleitung}

\section{Hintergrund und Motivation}
\section{Ziele und Umfang}
\section{Aufbau der Arbeit}

\chapter{Theoretische Grundlagen}

\section{envVisio}
\subsection{Das Realitätsmodell}
\subsubsection{Allgemeine Erklärung von envVisio}
\subsubsection{Vielleicht Aufteilung in mehrere Abschnitte?}
\subsubsection{was ist alles wichtig?}
\subsection{Laden und Konvertieren}
\subsubsection{auf Loader und insbesondere Converter eingehen}
\subsubsection{da in Converter Formular eingesetzt werden soll}
\subsubsection{wichtig, Prozesse zu verstehen}

\section{Human-Centered Design}
\subsubsection{Begriffe und Konzepte aus Design of Everyday Things}
\subsubsection{Abschnittstitel nicht final, vielleicht is HCD auch ein Unterpunkt auf gleicher Höhe wie Aff., Sign. und Map.}
\subsection{Affordances, Signifiers und Mappings}
\subsubsection{Kann gut bei 2.3.2 angewendet werden}
\subsection{Activity-Centered Design}

\section{Formulare}
\cite{wroblewskiWebForm2008}
\subsection{Funktionsweise}
\subsubsection{Technische Fkt.-weise von F.}
\subsection{Designanfordungen}
\cite{normanDesignEveryday2013} (auf Affordances, Signifiers und Mappings eingehen)
\subsubsection{Rückbezug auf 2.2}
\subsubsection{Anordnung}
\subsubsection{Kontrollelemente}
\subsubsection{Bezeichnungen und Platzhalter}
\subsubsection{Validierung}

\clearpage
\section{Usability Testing}

Um auf \textit{Usability Testing} einzugehen, lohnt es sich, einen genaueren Blick auf
\textit{Usability} zu werfen. \citefield{ISO924111}{shorttitle} definiert \textit{Usability} wie
folgt:
\begin{quote}
  extent to which a system, product or service can be used by specified users to achieve specified
  goals with effectiveness, efficiency and satisfaction in a specified context of use
  \hspace*{\fill{}}\shorthandtextcite*{ISO924111}
\end{quote}
Diese Definition ist kurz, umfasst jedoch trotzdem drei Hauptelemente, die \textit{Usability}
ausmachen. Es sollte möglich sein, Aufgaben mit ausreichender Geschwindigkeit auszuführen
(\textit{efficiency}) und das erwünschte Ziel zu erreichen (\textit{effectiveness}). Der dritte
wichtige Faktor ist die Anwenderzufriedenheit (\textit{satisfaction}), welche mehr auf die
subjektive Erfahrung von Nutzer:innen eingeht. So kann eine Aufgabe schnell und korrekt ausgeführt
werden, aber dennoch nicht zu Zufriedenheit führen, da beispielsweise wichtige Informationen nicht
angezeigt wurden, die über den Erfolg informieren. Der gleiche Effekt kann auch andersherum
stattfinden - sollte die persönliche Einschätzung von Zufriedenheit sehr gut sein, können womögliche
Probleme der Effizienz und Effektivität vernachlässigt werden, sodass das Produkt trotzdem als gut
eingeschätzt wird. \parencite{barnumUsabilityTesting2021}

Des Weiteren betont \citeauthor{barnumUsabilityTesting2021} \cite{barnumUsabilityTesting2021}, dass
es sich um spezifische Nutzer:innen, Ziele und Nutzungskontexte handelt. Das bedeutet, dass
\textit{Usability} nur für eine spezifische Gruppe an Menschen betrachtet werden kann, welche das
Produkt benutzen werden und deren Ziele denen des Produkts entsprechen müssen. Außerdem kann die
\textit{Usability} auch nur für den angedachten Nutzungskontext - die Umgebung in der ein Produkt
genutzt wird - betrachtet werden. \parencite{barnumUsabilityTesting2021}

\textcite{quesenberyDimensionsUsability2003} und \textcite{nielsenUsabilityEngineering1994}
spezifizieren folgende 5 Attribute der \textit{Usability}:
\begin{itemize}
  \item \textbf{Effektivität}: wie umfassend und akkurat Aufgaben abgeschlossen werden,
  \item \textbf{Effizienz}: wie schnell dies erfolgt,
  \item \textbf{Ansprechende Gestaltung}: wie gut Interaktionen gesteuert werden und für
    Zufriedenheit gesorgt wird (von \textcite{nielsenUsabilityEngineering1994} vernachlässigt)
  \item \textbf{Fehlertoleranz}: wie das System Fehler verhindert und sich von Fehler erholt
  \item \textbf{Erlernbarkeit}: wie intuitiv das System ist, und wie Lernen im Laufe der Nutzung
    ermöglicht wird (von \textcite{nielsenUsabilityEngineering1994} in zwei Attribute geteilt)
\end{itemize}
Laut \textcite{barnumUsabilityTesting2021} hat der Faktor der Zufriedenheit in den letzten Jahren an
Bedeutung zugenommen. Das wird auch davon unterstützt, dass
\textcite{nielsenUsabilityEngineering1994} weniger Wert auf eine ansprechende Gestaltung legte als
\textcite{quesenberyDimensionsUsability2003}.

\newpage

\textit{Usability Testing} ist ein Sammelbegriff für Methoden bei denen Benutzer:innen mit einem
System interagieren, um ein Ziel zu erreichen oder Szenario umzusetzen. Dabei sind die Umstände der
Interaktionen kontrolliert und Verhaltensdaten werden gesammelt
\parencite{wichanskyUsabilityTesting2000}. Ziel ist es, die \textit{Usability} des Systems zu
überprüfen und eventuelle Probleme zu identifizieren.

Dabei ist es laut \textcite{barnumUsabilityTesting2021} wichtig, dass die Nutzer:innen Aufgaben
erfüllen, die echt sind und eine Bedeutung für sie beinhalten. So werden Probleme, die bei der
Nutzung in der realen Welt auftreten effizienter aufgedeckt.

Wichtige Aspekte des \textit{Usability Testing} werden im weiteren Verlaufe des dieses Kapitels
beleuchtet.

\subsection{Testmethoden}
\subsubsection{Formative und Summative Studien}
\label{sec:formative-summative}
Usability Tests können in zwei Typen unterteilt werden: \textbf{Formative} und \textbf{Summative
  Studien}.

\vspace{\baselineskip}

\textbf{Formative Studien} werden während der Entwicklung durchgeführt, um Probleme zu diagnostizieren und zu lösen. Der Umfang der einzelnen Studien ist relativ klein, im Entwicklungszyklus können sie jedoch meist wiederholt durchgeführt werden \parencite{barnumUsabilityTesting2021}. Die Ergebnisse von formativen Studien können direkt in die Entwicklung einfließen, wobei erneute Studien überprüfen können, ob die Probleme aus vergangen Durchläufen behoben wurden. Des Weiteren können formative Studien aufzeigen, was Nutzer:innen wichtig ist, und somit die Entwicklung in eine Richtung lenken, die die Usability des Endprodukts drastisch verbessert \parencite{barnumUsabilityTesting2021}. Erkenntnisse sind in zeitigen Stadien der Produktentwicklung besonders wichtig, da es im Verlauf des Entwicklungszyklus immer schwieriger wird, grundlegend falsche Entscheidungen zu berichtigen \parencite{barnumUsabilityTesting2021}.

Für die Zwecke dieser Arbeit ist eine formative Studie angemessen (Vgl. \ref{sec:study-methods}). Daher werden Schlüsselelemente von formativen Studien, wie zum Beispiel Benutzerprofile, Szenarios und das \acl{TA}, im weiteren Verlauf detailliert vorgestellt.

\vspace{\baselineskip}

\textbf{Summative Studien} werden durchgeführt wenn ein Produkt fertiggestellt ist oder kurz vor der Fertigstellung steht \parencite{barnumUsabilityTesting2021}. Dabei soll unter Usability und Funktionalität auch die Zufriedenheit der Nutzer:innen bewertet werden. Diese Art von Studie findet in einem größeren Rahmen statt und erfordert in der Regel viele Teilnehmer, um statistische Relevanz zu erreichen \parencite{barnumUsabilityTesting2021}. Ähnlich zu formativen Studien werden Teilnehmer:innen Szenarios und Aufgaben bereitgestellt, darüber hinaus wird jedoch nicht weiter mit ihnen interagiert. Somit können Metriken wie die Zeit für die Bearbeitung von Aufgaben und Abschlussquoten erhoben werden.

Die Ergebnisse von summativen Studien sind meistens numerisch, während Erkenntnisse von formativen Studien konkrete Problembeschreibungen beinhalten. Es ist also einfacher, summativen Studien miteinander zu vergleichen, während formative Studien konkretere Probleme auflisten und frühzeitig und schnell zu Verbesserungen führen können \parencite{barnumUsabilityTesting2021}.

\subsubsection{\acl{TA}}
\label{sec:think-aloud}

Das \acf{TA} ist eine weit verbreite Methode, um die Usability von Applikationen einzuschätzen. Dabei sollen Nutzer:innen ihre Erfahrungen, Gedanken, Handlungen und Gefühle beim Interagieren mit der Anwendung ausdrücken \parencite{barnumUsabilityTesting2021}. Erkenntnisse, die auf diese Art gesammelt wurden, bieten Einblick in Denkprozesse von Nutzer:innen und können somit Entwicklungsentscheidungen lenken \parencite{alhadretiRethinkingThinking2018}.

Im Folgenden werden anhand von drei Studien verschiedene Funktionsweisen von \ac{TA} vorgestellt, und Auswirkungen auf den Usability Testing Prozess beleuchtet.

Typischerweise werden zwei Typen von \ac{TA} unterschieden: \ac{CTA} und \ac{RTA}. Bei \ac{CTA} teilen die Teilnehmer:innen ihre Gedanken sofort beim Ausführen der Aufgaben, während dies bei \ac{RTA} erst im Anschluss daran passiert \parencite{alhadretiRethinkingThinking2018}. Einerseits kann \ac{CTA} es einfacher machen Problembereiche zu identifizieren, da die zusätzlichen Informationen in Echtzeit bereitgestellt werden, andererseits existieren folgende Bedenken: Das zum Handeln parallel Sprechen könnte ungewohnt und irritierend für Teilnehmer:innen sein und dadurch zu einer unangenehmen Situation führen \parencite{alhadretiRethinkingThinking2018}. Des Weiteren könnte \ac{CTA} Denkprozesse aktiv beeinflussen und somit die Art und Weise in der Aufgaben abgearbeitet werden verändern \parencite{alhadretiRethinkingThinking2018}. Laut \citeauthor{alhadretiRethinkingThinking2018} sollte bei der Betrachtung der Aussagekraft der erhobenen Daten auch beachtet werden, dass durch die Verwendung von \ac{CTA} oft zusätzlich Zeit in Anspruch genommen wird. Im Gegensatz dazu beeinflusst \ac{RTA} die Denkprozesse von Teilnehmer:innen nicht, und leidet somit nicht unter den möglichen Problemen des \ac{CTA}. Problematisch an dieser Methode ist jedoch, dass sich Teilnehmer:innen korrekt an die Ausführung der Aufgabe erinnern müssen \parencite{alhadretiRethinkingThinking2018}. Außerdem besteht die Gefahr, dass spätere Handlungen und Teilschritte vorherige Gedanken beeinflussen oder relativieren \parencite{alhadretiRethinkingThinking2018}. So könnten wichtige Erkenntnise verloren gehen. Vereinzelt wird auch auf einen hybriden Ansatz gesetzt, bei dem die Erkenntnise von \ac{CTA} durch eine Nachbesprechung
angereichert werden \parencite{alhadretiRethinkingThinking2018}.

\textcite{alhadretiRethinkingThinking2018} verglichen in einer Studie \ac{CTA}, \ac{RTA} und einen hybriden Ansatz miteinander. Dabei wurden folgende Metriken betrachtet: Allgemeine Leistung beim Ausführen von Aufgaben, Erfahrungen der Teilnehmer:innen, Menge und Qualität der gefundenen Usability-Probleme, sowie die Kosten zur Nutzung der Methoden. Am besten schnitt das \acl{CTA} ab, da es am meisten Probleme fand und das beste Feedback von Teilnehmer:innen hervorrief \parencite{alhadretiRethinkingThinking2018}. Es konnte auch kein Unterschied in der Leistung der Teilnehmer:innen im Vergleich zu Durchläufen ohne \ac{TA} festgestellt werden. Außerdem ist es vorteilhaft, \ac{CTA} gegenüber \ac{RTA} und einem hybriden Ansatz zu wählen, da diese doppelt so viel Test- und Analysierungszeit benötigten \parencite{alhadretiRethinkingThinking2018}.

Abgesehen von den soeben betrachteten Typen des \acl{TA} existieren auch Unterschiede im Umfang der Interaktion zwischen Testadministrator:innen und Teilnehmer:innen. \textcite{olmsted-hawalaThinkaloudProtocols2010} testeten folgende auftretenden Kategorien, räumten jedoch ein dass die Realität oft nuancierter ist:
\begin{itemize}
  \item Traditionell: Außer "Bitte reden Sie weiter." werden keine weiteren Nachfragen angestellt.
  \item Sprachkommunikation: In Form von "um-hum" oder "un-hum" werden Teilnehmer:innen zum Weiterreden angeregt. Nach ausreichend langen Stillephasen wird mit dem zuletzt geäußerten Wort das \acl{TA} weiter angeregt. \footnote{Eine Interaktion könnte beispielsweise so aussehen:\\ Teilnehmer:in: "Das war komisch..." \\ -- 15 Sekunden Stille -- \\ Administrator:in: "Komisch?"}
  \item Coaching: Es wird aktiv in eingegriffen, indem direkte Fragen gestellt werden und assistiert wird, falls Teilnehmer:innen zu große Schwierigkeiten haben.
\end{itemize}
Die Ergebnisse der \citeyear{olmsted-hawalaThinkaloudProtocols2010} durchgeführten Studie zeigen, dass Teilnehmer:innen, bei denen Coaching angewendet wurde, erfolgreicher als andere waren und zufriedener mit der getesteten Website waren. Die Zeit zur Beendigung der Aufgaben unterschied sich nicht zwischen den verschiedenen Herangehensweisen und, genauso wie bei \citeauthor{alhadretiRethinkingThinking2018}, konnte kein Unterschied zur Kontrollgruppe (ohne \ac{TA}) festgestellt werden. Eine Beeinflussung der Ergebnisse scheint also nur beim Coaching aufzutreten - während die restlichen Variationen des \ac{TA} weder die Genauigkeit noch die Geschwindigkeit abändern. Es wird betont, dass eine genaue Beschreibung des verwendeten \acl{TA} wichtig ist, um einen hohen Grad an Reproduzier- und Vergleichbarkeit zu erreichen \parencite{alhadretiRethinkingThinking2018}.

Eine Studie von \textcite{rheniusEvaluationConcurrent1990} kam zu leicht anderen Ergebnissen. Es wurde festgestellt, dass die Zeit zum Lösen von Aufgaben bei der Verwendung von \ac{TA} steigt, die Genauigkeit jedoch nicht davon beeinflusst wurde. \citeauthor{rheniusEvaluationConcurrent1990} kamen auch zu dem Schluss, dass der Einfluss von \ac{TA} nur in den frühen Phasen der Gewöhnung an die Aufgaben auftritt \parencite{rheniusEvaluationConcurrent1990}.

Somit kann also festgehalten werden, dass die Verwendung von \ac{TA} wenn überhaupt nur eine minimale Auswirkung auf die Zeit zur Fertigstellung von Aufgaben hat. Eine Beeinflussung auf die Testergebnisse konnte in keiner der Studien festgestellt werden, außer bei der Verwendung von Coaching. Das \acl{TA} stellt sich somit als gutes Werkzeug zur Durchführung von Usability Studien heraus. Insbesondere eine Verwendung von \acl{CTA} und eine Beschränkung auf minimale Interaktionen mit den Teilnehmer:innen scheint vorteilhaft.

\subsection{Studiengröße}
\label{section:study-size}

Wie in \ref{section:formative-summative} erwähnt, existieren Usability Tests in unterschiedlichen Umfängen. Formative Studien sind klein und werden meist während der Entwicklung durchgeführt, während summative Studien mehr Teilnehmer*innen umfassen und gegen Ende der Entwicklung stattfinden.

Für diese Arbeit werden formative Studien von Relevanz sein. Dabei sollte die Anzahl der Teilnehmer*innen so klein wie möglich zu halten, während so viele Usability Probleme wie möglich entdeckt werden, denn eine höhere Anzahl an Testdurchläufen steigert die Kosten und den Zeitaufwand.
\parencites{faulknerFiveuserAssumption2003, nielsenWhyYou2000}

\textcite{nielsenMathematicalModel1993} beschreiben den Anteil an gefundenen Usability Problemen bei $n$ Teilnehmer*innen als wie folgt:
\begin{equation}
  \label{equation:finding-usability-problems}
  N(1-(1-\lambda{})^n),
\end{equation}
wobei $N$ die Gesamtanzahl der Probleme und $\lambda{}$ der Anteil der Probleme die bei einem einzigen Testdurchlauf gefunden werden. Mit dem von ihm über mehrere Projekte beobachtete Wert von $\lambda{}=31\%$ kommt \textcite{nielsenWhyYou2000} zum Resultat, dass mit fünf Teilnehmer*innen 85\% aller Probleme festgestellt werden können. \cite{nielsenWhyYou2000} Weitere Testdurchläufe sind weniger lohnenswert, da nicht genug neue Probleme aufgedeckt werden. Um alle Probleme mit einer Studie zu finden, wären 15 Teilnehmer*innen nötig, die gegen Ende jedoch nur noch vereinzelt neue Erkenntnisse liefern. Da Teilnehmer*innen oft die gleichen, oberflächlichen Fehler finden und dadurch abgelenkt werden, ist es von Vorteil nach den ersten fünf Durchläufen die gefundenen 85\% zu beheben, und dann eine weitere Studie durchzuführen. Eine dreifache Wiederholung der Tests mit fünf Durchläufen und einem verbesserten Produkt führt zu umfangreicheren Resultat als ein einziger Test mit 15 Durchläufen.
\parencite{nielsenWhyYou2000}
\todo{vielleicht update von 2012 mit einbeziehen? \cite{nielsenHowMany2012}}

Die als Faustregel adaptierte Zahl Fünf wurde in mehreren Publikationen untersucht. Während fünf Teilnehmer*innen durchschnittlich 85\% der Probleme (einer Studie) entdecken können, kann dieser Prozentsatz laut \textcite{faulknerFiveuserAssumption2003} jedoch bis auf 55\% sinken. Während die durchschnittliche Rate langsamer steigt, verringert sich die Varianz mit mehr Testdurchläufen. Außerdem ist anzumerken, dass die Frage, wie viele Teilnehmer*innen benötigt werden schwer zu  beantworten ist, und von vielen Faktoren abhängt. \footnote{genannt werden: Art und Erfahrung der Teilnehmer*innen, Wichtigkeit des Systems, mögliche Folgen von Usability Problemen}
\parencite{faulknerFiveuserAssumption2003}

\textcite{spoolTestingWeb2001} fanden einen deutlich geringeren Wert von $\lambda{}=10\%$ bei unstrukturiertem Testen von E-Commerce-Webseiten. Somit würden mit 5 Testdurchläufen nur 35\% der Probleme gefunden werden. Als Gründe dafür kann die Komplexität der Webseite und die persönlich getroffenen Entscheidungen der Teilnehmer*innen genannt werden.
\parencite{spoolTestingWeb2001}

Laut \textcite{woolrychWhyWhen2001} ist die Wahrscheinlichkeit, dass ein Problem entdeckt wird, nicht konstant sondern hängt von Schweregrad, Benutzermerkmalen, Produkttyp und Strukturierungsgrad des Testes ab. Sie erweitern deshalb Gleichung \ref{equation:finding-usability-problems} um $\lambda{}_j$ für jedes Usability Problem, welches die Wahrscheinlichkeit darstellt, dass Problem von einer zufällig ausgewählten Teilnehmer*in gefunden wird. Die erwartete Nummer an gefundenen Problemen beträgt dann laut \textcite{woolrychWhyWhen2001}:
\begin{equation}
  \sum_{j=1}^n 1-(1-\lambda{}_j)^n
\end{equation}
Um Gleichung \ref{equation:finding-usability-problems} zu reproduzieren, müsste $\lambda{}_j = \lambda{}$ für alle $j$ gelten. Das bedeutet, dass die Wahrscheinlichkeiten, einzelner Probleme gefunden zu werden, eine geringe Variabilität besitzen. Die von \textcite{nielsenMathematicalModel1993} eingeführte Gleichung \ref{equation:finding-usability-problems} wird \textcite{woolrychWhyWhen2001} zufolge unzuverlässiger, sobald eine größere Varianz im System und bei den Teilnehmer*innen existiert. Sich unterscheidende Interaktionen mit dem System und die Auswahl der Aufgaben beinflussen die Zuverlässigkeit eines einzigen Wertes $\lambda{}$.
\parencite{woolrychWhyWhen2001}

\todo{Fazit für den Abschnitt? \textcite{barnumUsabilityTesting2021} stimmt eher \textcite{nielsenMathematicalModel1993} zu, aber mit Abhandlung von vorher passt das nicht mehr zusammen}

\subsection{Personas}
\subsection{Aufgaben und Scenarios}
\subsection{Evaluation}
\subsubsection{Wie können Fragebögen zur Evaluierung erstellt werden?}
\subsubsection{Verwendung von Standard-Fragebogen}

\section{Analyse und Vergleich von existierenden Projekten}
\subsection{Home Assistant}
\subsection{DBSnap}
\url{https://www.semanticscholar.org/paper/DBSnap\%3A-Learning-Database-Queries-by-Snapping-Silva-Chon/aabf7791575f8cc62432ca5e6289862eec0f6845}
\url{https://www.semanticscholar.org/paper/DBSnap-2\%3A-New-Features-to-Construct-Database-by-Silva-Loza/61c95757d7d4151f51ea848623c6839816756385}
\subsection{SQheLper}
\url{https://www.semanticscholar.org/paper/SQheLper\%3A-A-block-based-syntax-support-for-SQL-Jacobs-Jaschke/b4f6023a4ee72d0103ee521c4f9b35768d38810b}
\subsection{Scratch}


\chapter{Implementierung}
\section{Zielformulierung}
\subsubsection{Genaue Formulierung der Problemstellung, was ist das Ziel?}
\section{Lösungsvorstellung}
\subsubsection{Erklärung des Resultats}

\chapter{Studie}
\section{Studienaufbau}
\subsection{Problemformulierung und Ziele}
\subsection{Auswahl von Teilnehmer:innen}
\subsection{Aufgaben}
\subsection{Scenarios}
\subsection{Evaluation}
\subsection{Studienablauf}
\section{Ergebnisse und Analyse}
\section{Schlussfolgerungen und Zukunftsaussicht}

\chapter{Fazit}


\nocite{*}
\chapter*{Quellen}
\begingroup
\let\clearpage\relax
\printbibliography[filter=articles, title=Artikel, heading=subbibliography]
\printbibliography[type=book, title=Bücher, heading=subbibliography]
\printbibliography[type=online, title=sonstige, heading=subbibliography]
\endgroup

\printindex

\end{document}
