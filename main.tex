\documentclass[oneside,a4paper,11pt,openright]{report}


\usepackage[utf8]{inputenc} % Einstellung der Eingabekodierung
\usepackage[ngerman]{babel} % Deutsche Trennmuster
\usepackage{newpxtext} % Schriftart Palatino
\usepackage{newpxmath} % passende Mathe-Schrift
\usepackage[T1]{fontenc} % Einstellung Schriftkodierung
\usepackage{amsmath} % Mathematik-Erweiterungen
\usepackage{microtype} % Verbesserte Mikrotypographie
\usepackage{csquotes} % Kontextabhängige Anführungszeichen
\usepackage{calc} % Berechnung von TeX-Maßen
\usepackage[%
  inner=25mm,%
  outer=35mm,%
  top=25mm,%
  bottom=30mm,%
  includeheadfoot,%
  headheight=14pt,%
  marginparwidth=25mm%
]{geometry} % Anpassung Satzspiegel
% \usepackage{parskip} % Kein Absatzeinzug
\usepackage{emptypage} % Keine Seitenzahlen auf Leerzeiten
\usepackage{array} % Tabellen-Erweiterung
\usepackage{graphicx} % Einbindung von Grafiken
\usepackage{float} % Anpassung von Gleitumgebungen
\usepackage{fancyhdr} % Anpassung von Kopf- und Fußzeilen
\usepackage{acro} % Abkürzungen
\usepackage{listings} % Codeblöcke
\usepackage{pgfplots}
\usepackage{pgfplotstable}


\usepackage[style=apa,backend=biber,backref,natbib]{biblatex}

% auf Einbindungsreihenfolge achten
\usepackage{hyperref}


% können nach Fertigstellung gelöscht werden
\usepackage{todonotes} % TODO-Anmerkungen
\usepackage{lipsum} % Blindtexte
\usepackage{blindtext} % Blindtexte
\usepackage{layout} % Darstellung des Satzspiegels

\floatplacement{figure}{htb}
\floatplacement{table}{htb}
   % Anpassungen für Gleitumgebungen
% Umgebung für das Deckblatt
\newenvironment{tuctitlepage}
{
  \cleardoublepage
  \titlepage
  \center
}
{
  \endcenter
  \endtitlepage
}


% Professur, Institut, ...
\newenvironment{tuctitleorgunit}{\center}{\endcenter}


% Logo
\newcommand{\tuctitlelogoname}{tuc/tuc_green_margin}

\newcommand{\tuctitlelogo}[1][0.6\linewidth]
{%
  \begin{center}%
    \includegraphics[width=#1]{\tuctitlelogoname}%
  \end{center}%
}


% Block mit kleiner Überschrift.
\newenvironment{tuctitleblock}[2][\medskipamount]
{\center\vspace*{#1}{#2\par}\vspace*{#1}}
{\par\endcenter}


% Art der Arbeit
\newcommand{\tuctitlethesistype}[1]
{\begin{center}{\huge\bfseries#1}\par\end{center}}


% Angestrebter Abschluss
\newcommand{\tuctitledegreephrase}{zur Erlangung des akademischen Grades}
\newcommand{\tuctitledegreeline}[1]
{{\LARGE #1}\par}
\newcommand{\tuctitledegree}[2][\medskipamount]
{\begin{tuctitleblock}[#1]{\tuctitledegreephrase}%
    \tuctitledegreeline{#2}%
  \end{tuctitleblock}}


% Autor
\newcommand{\tuctitleauthorphrase}{vorgelegt von}
\newcommand{\tuctitleauthorline}[1]
{{\Large #1}\par}
\newcommand{\tuctitleauthor}[2][\medskipamount]
{\begin{tuctitleblock}[#1]{\tuctitleauthorphrase}%
    \tuctitleauthorline{#2}%
  \end{tuctitleblock}}


% Lehrveranstaltung
\newcommand{\tuctitlecoursephrase}{zur Lehrveranstaltung}
\newcommand{\tuctitlecourseline}[1]
{{\LARGE #1}\par}
\newcommand{\tuctitlecourse}[2][\medskipamount]
{\begin{tuctitleblock}[#1]{\tuctitlecoursephrase}%
    \tuctitlecourseline{#2}%
  \end{tuctitleblock}}


% Semester
\newcommand{\tuctitletermphrase}{im}
\newcommand{\tuctitletermline}[1]
{{\Large #1}\par}
\newcommand{\tuctitleterm}[2][\medskipamount]
{\begin{tuctitleblock}[#1]{\tuctitletermphrase}%
    \tuctitletermline{#2}%
  \end{tuctitleblock}}


% Thema der Arbeit
\newenvironment{tuctitletopic}
{\center\LARGE\medskip\begingroup}
{\par\endgroup\medskip\endcenter}


%
% Deckblatttabellen ohne Einzug
%
% #1 -> Formatierung der linken Spalte
% #2 -> Anzahl rechter Spalten
%
\newenvironment{tuctitletable}[2][]
{\center\tabular{>{#1}r*{#2}{l}}}
{\endtabular\endcenter}


%
% Deckblatttabellen mit Einzug
%
% #1 -> Formatierung der linken Spalte
% #2 -> Anzahl rechter Spalten
% #3 -> Breite der linken Spalte
%
\newenvironment{tuctitletableindent}[3][]
{\flushleft\tabular{@{}>{\raggedleft#1}p{#3-\tabcolsep}*{#2}{l}}}%
{\endtabular\endflushleft}


% Datum, Ort
\newcommand{\tuctitleplacedate}[1]
{\center\medskip#1\par\endcenter}
 % Formatvorlage für das Deckblatt
\newenvironment{tucsimplesection}[2][\bigskipamount]
{
  \par\centerline{\textbf{\Large #2}}\par
  \vspace*{#1}
}
{}
 % Einfache Überschriften für Aufgabenstellung usw.
\newenvironment{tucerklaerung}[1][Selbstständigkeitserklärung]
{
  \pagestyle{empty}
  \thispagestyle{empty}
  \newcommand{\tucsignature}[2][5cm]%
  {
    \par
    \begin{flushright}
      \begin{minipage}{##1}
        \footnotesize
        \vbox{\hbox to \hsize{\rule{0pt}{7ex}\dotfill}\hbox{\hspace{1em}##2}}
      \end{minipage}
    \end{flushright}
  }
  \tucsimplesection{#1}
}
{
  \endtucsimplesection
}
 % Formatvorlage für die Selbstständigkeitserklärung

% Bei Bedarf aktivieren
%%
% Lebender Kolumnentitel
%
\fancyhf{}
\fancyhead[RE]{\slshape\leftmark}
\fancyhead[LO]{\slshape\rightmark}
\fancyhead[LE,RO]{\thepage}
\renewcommand{\headrulewidth}{0pt}
\renewcommand{\footrulewidth}{0pt}
\pagestyle{fancy}
 % Einstellungen zu Kopf- und Fußzeilen
%%
% Eine Änderung des Zeilenabstands wirkt sich nur auf Mengentext aus. In
% Fußnoten und Gleitumgebungen kommt der einfache Zeilenabstand zur Anwendung.
% Der folgende TeX-Code aktiviert den erweiterten Zeilenabstand für diese
% Fälle.
%


%
% Erweiterter Zeilenabstand in Gleitumgebungen
%
\makeatletter
\let\@xfloat=\latex@xfloat
\makeatother


%
% Erweiterter Zeilenabstand in Fußnoten
%
\makeatletter
\long\def\@footnotetext#1{%
  \insert\footins{%
    \reset@font\footnotesize
    \interlinepenalty\interfootnotelinepenalty
    \splittopskip\footnotesep
    \splitmaxdepth \dp\strutbox \floatingpenalty \@MM
    \hsize\columnwidth
    \@parboxrestore
    \protected@edef\@currentlabel{%
      \csname p@footnote\endcsname\@thefnmark
    }%
    \color@begingroup
    \@makefntext{%
      \rule\z@\footnotesep\ignorespaces#1\@finalstrut\strutbox}%
    \color@endgroup}}
\makeatother


%
% Erweiterter Zeilenabstand in Fußnote von minipages
%
\makeatletter
\long\def\@mpfootnotetext#1{%
  \global\setbox\@mpfootins\vbox{%
    \unvbox \@mpfootins
    \reset@font\footnotesize
    \hsize\columnwidth
    \@parboxrestore
    \protected@edef\@currentlabel{%
      \csname p@mpfootnote\endcsname\@thefnmark}%
    \color@begingroup
    \@makefntext{%
      \rule\z@\footnotesep\ignorespaces#1\@finalstrut\strutbox}%
    \color@endgroup}}
\makeatother
  % Erweiterter Zeilenabstand in Gleitumgebungen, Fußnoten und Minpages

\addbibresource{bibliography.bib}

% https://tex.stackexchange.com/a/302313
\makeatletter
\newcommand{\newstarcommand}[1]{%
  \DeclareRobustCommand#1{%
    \@ifstar{\csname s\string#1\endcsname}{\csname n\string#1\endcsname}%
  }%
  \edef\meta@def@name{\string#1}%
  \meta@def
}
\newcommand\meta@def[3][0]{%
  \expandafter\newcommand\csname n\meta@def@name\endcsname[#1]{#2}%
  \expandafter\newcommand\csname s\meta@def@name\endcsname[#1]{#3}%
}
\makeatother

% todo footnotes
% \newcounter{todocounter}
% \setcounter{todocounter}{100}
% \newcommand{\todo}[1]{\refstepcounter{todocounter}\textcolor{red}{\footnote[\thetodocounter]{\textcolor{gray}{TODO: #1}}}}

% inline box notes
\newcommand{\boxnote}[1]{%
  \bigbreak
  \begin{center}
    \noindent\fbox{\parbox{0.85\textwidth}{#1}}
  \end{center}
  \bigbreak
}

\defbibfilter{articles}{
  type=article or
  type=inproceedings or
  type=report
}

\defbibfilter{books}{
  type=book or
  type=inbook or
  type=incollection
}

\defbibfilter{software}{
  type=software
}

\defbibfilter{other}{
  type=misc or
  type=thesis or
  type=online or
  type=legislation
}

\DefineBibliographyStrings{ngerman}{
  nodate = {{}o\adddot \space D\adddot},
  mathesis = {Masterarbeit}
}

\makeatletter
% cite the alias like this: OGC (2016) or Open Geospatial Consortium (OGC) (2016)
\DeclareCiteCommand{\textcitealias}
{\usebibmacro{prenote}}
{\usebibmacro{citeindex}%
  \printtext[bibhyperref]{\@citealias{\thefield{entrykey}} (\citeyear{\thefield{entrykey}})}}
{\multicitedelim}
{\usebibmacro{postnote}}

% cite the alias like this: (OGC, 2016) or (Open Geospatial Consortium (OGC), (2016))
\DeclareCiteCommand{\parencitealias}[\mkbibparens]
{\usebibmacro{prenote}}
{\usebibmacro{citeindex}%
  \printtext[bibhyperref]{\@citealias{\thefield{entrykey}}, \citeyear{\thefield{entrykey}}}}
{\multicitedelim}
{\usebibmacro{postnote}}

% cite the alias with a forced short form like this: OGC (2016)
\DeclareCiteCommand{\textcitealiass}
{\usebibmacro{prenote}}
{\usebibmacro{citeindex}%
  \printtext[bibhyperref]{\@citealias{\thefield{entrykey}S} (\citeyear{\thefield{entrykey}})}}
{\multicitedelim}
{\usebibmacro{postnote}}

% cite the alias with a forced short form like this: (OGC, 2016)
\DeclareCiteCommand{\parencitealiass}[\mkbibparens]
{\usebibmacro{prenote}}
{\usebibmacro{citeindex}%
  \printtext[bibhyperref]{\@citealias{\thefield{entrykey}S}, \citeyear{\thefield{entrykey}}}}
{\multicitedelim}
{\usebibmacro{postnote}}
\makeatother

\defcitealias{ogcOGC}{\ac{OGC}}
\defcitealias{ogcCatalogue2016}{\ac{OGC}}
\defcitealias{ogcFiltering}{\ac{OGC}}
\defcitealias{ogcFilteringS}{\acs*{OGC}}
\defcitealias{ogcOGCAPI12022}{\ac{OGC}}
\defcitealias{ogcOGCAPI22022}{\ac{OGC}}
\defcitealias{ogcOGCAPI4}{\ac{OGC}}
\defcitealias{ISO924111}{ISO/TC 159 Ergonomics SC 4}
\defcitealias{unisd2021}{UNIS-D}
\defcitealias{bootstrapIcons}{Bootstrap Icons}
\defcitealias{scratchfoundationScratch}{Scratch}
\defcitealias{monigSnap}{\textit{Snap!}}

\renewcommand{\lstlistingname}{Quelltext}
\renewcommand{\lstlistlistingname}{Quelltextverzeichnis\todo{Heißt das so?}}

\makeatletter
\let\orig@lstnumber=\thelstnumber

\newcommand\lstsetnumber[1]{\gdef\thelstnumber{#1}}
\newcommand\lstresetnumber{\global\let\thelstnumber=\orig@lstnumber}
\makeatother

\lstset{numbers=left}

\colorlet{punct}{red!60!black}
\definecolor{background}{HTML}{EEEEEE}
\definecolor{delim}{RGB}{20,105,176}
\colorlet{numb}{magenta!60!black}

\lstdefinelanguage{json}{
  basicstyle=\normalfont\ttfamily,
  numbers=left,
  numberstyle=\scriptsize,
  stepnumber=1,
  numbersep=8pt,
  showstringspaces=false,
  breaklines=true,
  frame=lines,
  backgroundcolor=\color{background},
  literate=
   *{:}{{{\color{punct}{:}}}}{1}
    {,}{{{\color{punct}{,}}}}{1}
    {\{}{{{\color{delim}{\{}}}}{1}
    {\}}{{{\color{delim}{\}}}}}{1}
    {[}{{{\color{delim}{[}}}}{1}
    {]}{{{\color{delim}{]}}}}{1},
}

\lstdefinelanguage{bnf}{
  basicstyle=\normalfont\ttfamily,
  numbers=left,
  numberstyle=\scriptsize,
  stepnumber=1,
  numbersep=8pt,
  showstringspaces=false,
  breaklines=true,
  frame=lines,
  backgroundcolor=\color{background},
  literate=
   *{=}{{{\color{punct}{=}}}}{1}
    {|}{{{\color{punct}{|}}}}{1}
    {;}{{{\color{punct}{;}}}}{1}
    {\{}{{{\color{delim}{\{}}}}{1}
    {\}}{{{\color{delim}{\}}}}}{1}
    {[}{{{\color{delim}{[}}}}{1}
    {]}{{{\color{delim}{]}}}}{1},
}

\definecolor{plot1}{HTML}{4F81BD}
\definecolor{plot2}{HTML}{C0504D}
\definecolor{plot3}{HTML}{9BBB59}
\definecolor{plot4}{HTML}{9B0B59}


%\makeglossaries
%\newacronym{asq}{ASQ}{After Scenario Questionnaire}
%\newacronym{csuq}{CSUQ}{Computer System Usability Questionnaire}
%\newacronym{cql}{CQL}{Common Query Language}
%\newacronym{pssuq}{PSSUQ}{Post-Study System Usability Questionnaire}
%\newacronym{cta}{CTA}{Concurrent Think-Aloud-Protokoll}
%\newacronym{nasa-tlx}{NASA-TLX}{NASA Task Load Index}
%\newacronym{nps}{NPS}{Net Promoter Score}
%\newacronym{rta}{RTA}{Retrospective Think-Aloud-Protokoll}
%\newacronym{seq}{SEQ}{Single Ease Question}
%\newacronym{smeq}{SMEQ}{Subjective Mental Effort Questionnaire}
%\newacronym{sus}{SUS}{System Usability Scale}
%\newacronym{ta}{TA}{Think-Aloud-Protokoll}
%\newacronym{ume}{UME}{Usability Magnitude Estimation}
\DeclareAcronym{asq}{short=ASQ,long=After Scenario Questionnaire}
\DeclareAcronym{csuq}{short=CSUQ,long=Computer System Usability Questionnaire}
\DeclareAcronym{cql}{short=CQL,long=Common Query Language}
\DeclareAcronym{pssuq}{short=PSSUQ,long=Post-Study System Usability Questionnaire}
\DeclareAcronym{cta}{short=CTA,long=Concurrent Think-Aloud-Protokoll}
\DeclareAcronym{nasa-tlx}{short=NASA-TLX,long=NASA Task Load Index}
\DeclareAcronym{nps}{short=NPS,long=Net Promoter Score}
\DeclareAcronym{rta}{short=RTA,long=Retrospective Think-Aloud-Protokoll}
\DeclareAcronym{seq}{short=SEQ,long=Single Ease Question}
\DeclareAcronym{smeq}{short=SMEQ,long=Subjective Mental Effort Questionnaire}
\DeclareAcronym{sus}{short=SUS,long=System Usability Scale}
\DeclareAcronym{ta}{short=TA,long=Think-Aloud-Protokoll}
\DeclareAcronym{ume}{short=UME,long=Usability Magnitude Estimation}


% könnte zu Problemen führen
\MakeOuterQuote{"}

\begin{document}
\pagenumbering{roman}
\makeatletter
\begin{tuctitlepage}
  \begin{tuctitleorgunit}
    \organization
  \end{tuctitleorgunit}

  \tuctitlelogo

  \tuctitlethesistype{\thesistype}
  \bigskip
  \tuctitledegree{\degree}
  \bigskip
  \tuctitleauthor{\@author}

  \vspace*{0pt plus 2fill}
  \begin{tuctitletopic}
    \@title \\
    \large{\@subtitle}
  \end{tuctitletopic}
  \vspace*{0pt plus 2fill}

  \begin{tuctitletable}[\bfseries]{1}
    Erstprüfer:  & \erstpruefer \\
    Zweitprüfer: & \zweitpruefer \\
  \end{tuctitletable}

  \vspace*{0pt plus 1fill}

  \tuctitleplacedate{\place, \@date}
\end{tuctitlepage}
\makeatother

\cleardoublepage
\vspace*{0pt plus 1fill}
\begin{tucsimplesection}{Abstract}
  \blindtext\todo{anpassen}
\end{tucsimplesection}
\vspace*{0pt plus 2.5fill}

\tableofcontents

\addcontentsline{toc}{chapter}{\listfigurename}
\listoffigures

\addcontentsline{toc}{chapter}{\listtablename}
\listoftables

\addcontentsline{toc}{chapter}{\lstlistlistingname}
\lstlistoflistings

\addcontentsline{toc}{chapter}{Abkürzungen}
\printacronyms


\pagenumbering{arabic}

\chapter{Einleitung}

\section{Hintergrund und Motivation}
\section{Ziele und Umfang}
\section{Aufbau der Arbeit}

\chapter{Theoretische Grundlagen}

\section{Simplex4Data}
\parencite{rudolfUmweltdatenmanagementGeoInspiration2018}
\subsection{Das Realitätsmodell}
\subsubsection{Allgemeine Erklärung von envVisio}
\subsubsection{Vielleicht Aufteilung in mehrere Abschnitte?}
\subsubsection{was ist alles wichtig?}
\subsection{Laden und Konvertieren}
\subsubsection{auf Loader und insbesondere Converter eingehen}
\subsubsection{da in Converter Formular eingesetzt werden soll}
\subsubsection{wichtig, Prozesse zu verstehen}
\subsection{Szenarios}
\subsection{SimplexService}

\clearpage
\section{\acs{OGC} \acs{API} - Features}
\label{section:ogc}

Das \acf{OGC} entwickelt offene, raumbezogene Standards  \parencitealias{ogcOGC}. Der \textit{OGC API - Features}-Standard wird von Simplex4TwIS benutzt und durch den SimplexService erweitert, um Daten auszugeben. Der Standard wird durch mehrere Teile definiert:
\begin{itemize}
  \item \citetitle{ogcOGCAPI12022} \parencitealias{ogcOGCAPI12022}
  \item \citetitle{ogcOGCAPI22022} \parencitealias{ogcOGCAPI22022}
  \item Entwurf: \citetitle{ogcFiltering} \parencitealias{ogcFiltering}
  \item Entwurf: \citetitle{ogcOGCAPI4} \parencitealias{ogcOGCAPI4}
\end{itemize}

Der dritte Teil des Standards befindet sich zwar noch im Entwurfsstadium, beinhaltet jedoch Spezifikationen, die im Rahmen dieser Arbeit benutzt und von Simplex4TwIS umgesetzt werden. Dabei handelt es sich einerseits um die Abfragesprache \acf{CQL}, andererseits um Schnittstellen zum Auslesen von Queryables und verfügbaren Funktionen \parencitealias{ogcFiltering}. Im Folgenden wird auf diese drei Themen eingegangen.

\subsection{\acl{CQL}}

Bei der \acf{CQL} \footnote{Nicht zu verwechseln mit der \textit{Contextual} Query Language, welche zuerst auch als Common Query Language bezeichnet wurde \parencite{thelibraryofcongressCQLContextual2023, ZINGGentle2003}.} handelt es sich um eine Abfragesprache, welche durch das \ac{OGC} zum internationalen Standard erhoben wurde und von allen konformen \textit{\ac{OGC} Catalogue Services} unterstützt wird. \parencitealias{ogcCatalogue2016} In einem neuen Entwurf \parencitealias{ogcFiltering} wird eine Variation von \ac{CQL} eingeführt: Simple \ac{CQL}. Sie wird dazu benutzt um Filterbedingungen für Abfragen an \ac{OGC} API-Dienste zu erstellen und existiert in zwei Formaten: \textit{CQL\_TEXT} und \textit{CQL\_JSON} \parencitealias{ogcFiltering}. Im Weiteren wird sich auf Simple \ac{CQL} bezogen.

Der Syntax von \ac{CQL} ist inhaltlich analog zur \texttt{WHERE}-Klausel in \acs{SQL}, und unterstützt boolesche Prädikate, Textvergleiche, zeitliche Datentypen und raumbezogene Operatoren. Die Sprache kann um Prädikate, Operatoren und Datentypen erweitert werden \parencitealias{ogcCatalogue2016}.

Als typischer Anwendungsfall für \ac{CQL} dienen Geoinformationssysteme, mit denen sich die \ac{OGC} befasst. Daher sind im Gegensatz zu \ac{SQL} geometrische Funktionen wie \texttt{S\_EQUALS}, \texttt{S\_DISJOINT}, \texttt{S\_INTERSECTS} oder \texttt{S\_CONTAINS} bereits Teil der Sprache. Die in \ac{CQL} vorhandenen Datentypen umfassen sowohl primitive Datentypen, als auch zeitliche und raumbezogene Datentypen \parencitealias{ogcCatalogue2016}.

Der Entwurf des dritten Teil des OGC-API-Standards \parencitealias{ogcFiltering}, sieht die Möglichkeit vor, Abfragen über \ac{CQL} zu filtern. Laut Anforderung \texttt{/req/filter/filter-param} soll dies von Endpunkten wie \texttt{/collections/\{collectionId\}/items} unterstützt werden, welche Instanzlisten zurückgeben. Die Filterbedingungen, werden über den GET-Parameter \texttt{filter} spezifiziert und können mit \textit{CQL\_TEXT} und \textit{CQL\_JSON} umgesetzt werden \parencitealias{ogcFiltering}.

Während Filterprädikate boolesche Werte und \texttt{NULL} ergeben, können sie aus komplexeren Funktionen zusammengesetzt sein, welche auch andere Datentypen als Rückgabewert besitzen können. Über den Endpunkt \texttt{/functions} können Clients Informationen über die verfügbaren Funktionen abrufen.

\newcommand{\rfqsfn}{\footnote{\textcite{ogcFiltering}, \texttt{/rec/filter/queryables-schema}}}
\newcommand{\rfgqog}{\footnote{\textcite{ogcFiltering}, \texttt{/req/filter/get-queryables-op-global}}}
\newcommand{\rfgqol}{\footnote{\textcite{ogcFiltering}, \texttt{/req/filter/get-queryables-op-local}}}

\subsection{Queryables}
\label{sec:queryables}

\citetitle{ogcFiltering} beschreibt \textit{Queryables} als Token, welche ein eine Eigenschaft einer Ressource darstellen, die in einem Filterausdruck verwendet werden können \parencitealias{ogcFiltering}. Der vorläufige Standard sieht vor, dass ein \ac{API}-Endpunkt angeboten wird, der über die existierenden Queryables, deren Namen und Typen informiert.
\parencitealias{ogcFiltering}

Dabei existieren zwei Pfade. Unter \texttt{/queryables} werden die Attribute zurückgegeben , die für alle \textit{collections} existieren\rfgqog. Jede \textit{collection} verfügt des Weiteren über einen Unterpfad \texttt{/collections/\{collectionId\}/queryables}, welcher die spezifischen Attribute für die \textit{collection} angibt\rfgqol.
\parencitealias{ogcFiltering}

Der vorläufige Standard schlägt vor\rfqsfn, dass jedes ausgelieferte Attribut über einen Titel und, wenn nötig, über eine Beschreibung verfügt. Des Weiteren sollte jedes Attribut einen Typ angeben, außer es handelt sich um einen räumlichen Typ. Diese werden mittels einer Referenzierung des passenden \ac{JSON}-Schema des Geometrietyps gekennzeichnet. Zeitliche Typen sollen als \texttt{string} beschrieben werden und mit einem entsprechenden Format spezifiziert. Des Weiteren wird definiert, wie mit Aufzählungen und Wertebereichen umgegangen werden sollte.
\parencitealias{ogcFiltering}

\subsubsection{Queryables im SimplexService}

\lstinputlisting[
  label=lst:queryables,
  caption={Antwort auf \texttt{/queryables}-Anfrage für \textit{collection} "Gemeinden" (Auszug)},
  language=json,
  escapeinside=``
]{assets/queryables.json}

Im SimplexService werden die Daten im Realitätsmodell und in den SimplexSzenarios als "API - Features"-Dienst ausgegeben (\textcite{grossmannEnvVisioService2022}, Vgl. \ref{sec:simplex-service}). Jedes SimplexSzenario bildet einen eigenen solchen Dienst, wobei das erste Szenario dem Realitätsmodell entspricht. Objektklassen werden als collections mit der ID \texttt{\{sourceId\}-1-\{classId\}} ausgegeben. Der Pfad für eine typische Queryables-Anfrage im SimplexService sieht somit wie folgt aus:
\begin{quote}
  \texttt{/simplexservice/scenarios/1/collections/9-1-100/queryables}
\end{quote}
Abgefragt wird Klasse 100 der Quelle 9 im Realitätsmodell (SimplexSzenario 1). Eine Ausschnit aus der dazugehörigen Antwort ist in Quelltext \ref{lst:queryables} zu sehen. In diesem Fall handelt es sich um die Objektklasse "Gemeinde".

Jede Objektklasse in Simplex4Data verfügt über die gleichen Standardfelder (Titel, Schlüssel, Beschreibung, \dots). Diese können je nach Klasse per Metadaten unterschiedlich beschrieben werden, haben aber immer die gleichen Schlüssel. Außerdem werden für Objektklassen Attribute definiert, wie zum Beispiel Sachattribute (vom Typ Text, Ganzzahl, etc.) und Geometrieattribute (Geometrische Typen wie Fläche, Punkt, etc.). Sowohl die Standardfelder als auch die Attribute werden über den Queryables-Endpunkt ausgeliefert. Quelltext \ref{lst:queryables} enthält sowohl ein Standardfeld vom Typ string, als auch ein Geoemtrieattribut. Zu sehen ist, dass der Titel des Standardfelds gemäß der in der Objektklasse definierten Metadaten ausgeliefert wird und das Geometrieattribut nicht über \texttt{type}  definiert wird, sondern über \texttt{format} und eine Referenz zum zugehörigen \ac{JSON}-Schema.

Da in Simplex4Data Objektklassen miteinander verbunden sind, wurde der SimplexService um die Möglichkeit erweitert, Objektklasse-übergreifende Abfragen zu definieren. Diese geschieht über den \texttt{joins}-Parameter:
\begin{quote}
  \texttt{?joins=<9-1-100:9-3-106-obj-0:nn,>9-3-106-obj-0:9-1-102-obj-land}
\end{quote}
In diesem Fall wird die Verbindung 106 angesprochen, um mit der Abfrage außerdem die Informationen aus Klasse 102 abzufragen. Dabei handelt es sich um die Objektklasse "Bundesland". Wird der Parameter \texttt{joins} An eine Queryables-Abfrage angehangen, werden sowohl die Standardfelder und Attribute des der Basisklasse (in diesem Fall "Gemeinde"), als auch die der hinzugefügten Klassen zurückgegeben. Im zweiten Fall führt dies dazu, dass die Schlüssel der Elemente um einen Präfix erweitert werden, der angepasst werden kann. Im genannten Beispiel würde dann das Standardfeld der Klasse "Bundesland" \texttt{key} den Schlüssel \texttt{land.key} bekommen, während das gleiche Standardfelder der Klasse "Gemeinde" weiterhin \texttt{key} heißt.

\subsection{Funktionen in \acl{CQL}}
\label{sec:functions}

\citetitle{ogcFiltering} sieht Funktionen als ein Weg vor, um \ac{CQL} zu erweitern. Die Grammatik für Funktionen in \ac{CQL} werden durch die \ac{BNF} in Quelltext \ref{lst:function} definiert. Zu sehen ist, dass der Syntax der Funktionen ähnlich wie in Programmier- und Abfragesprachen wie Python, JavaScript oder \ac{SQL} definiert ist. Eine Funktion wird durch einen Funktionsname (\textit{identifier}) identifiziert, und kann eine Liste an Argumenten annehmen. Bei diesen Argumenten kann es sich um Literale, arithmetische und boolsche Ausdrücke, Arrays, Objektattribute oder weitere Funktionsaufrufe handeln. Die Attribute, die genutzt werden können, sollten über die in \ref{sec:queryables} beschrieben Schnittstelle ausgegeben werden \parencitealias{ogcFiltering} und stellen im SimplexService sowohl Standardfelder als auch Sachattribute dar (Vgl. \ref{sec:simplex-reality}).

\lstinputlisting[
  label=lst:function,
  caption={[Funktions-Definition in der \acs*{CQL}-Spezifikation in \acf*{BNF}]Funktions-Definition in der \acs{CQL}-Spezifikation in \acf{BNF} (Auszug). Dieser Vorschrift zufolge besteht eine Funktion in \textit{CQL\_TEXT} aus einem Funktionsname (\textit{identifier}), und einer Liste von Argumenten, die durch Klammern gekennzeichnet wird. Die List der Argumente wird durch Kommas getrennt, kann leer sein und aus verschiedenen Ausdrücken bestehen. \parencitealiass{ogcFiltering}},
  firstnumber=351,
  firstline=351,
  lastline=363,
  language=bnf
]{assets/cql2.bnf}

\textcitealias{ogcFiltering} definiert Funktionen als eine Erweiterung für \ac{CQL}, um komplexere Filterbedingungen zu erstellen. Diese Bedingungen evaluieren immer als boolscher Wert. Entweder hat die äußere Funktion von Filterbedingungen einen Wahrheitswert als Rückgabetyp, oder der äußere Ausdruck besteht aus einem boolschen Ausdruck \parencitealias{ogcFiltering}. Durch den rekursiven Charakter und den Zugriff auf Attribute aus Datensätzen lassen sich mit Hilfe der Funktionen jedoch auch beliebige komplexe Ausdrücke erstellen, welche auch zur Feldtransformation und Auswertung von Daten benutzt werden könnten.

Vorraussetzung dafür ist jedoch, dass alle benötigten Funktionen zur Verfügung stehen. Der vorläufige \ac{API}-Standard sieht vor, dass ein spezifischer Endpunkt über die auf dem Server verfügbaren Funktionen informiert. Unter \texttt{/functions} soll eine Liste von Funktionen zurückgegeben werden. Dazu gibt der Endpunkt für jede Funktion Auskunft über ihren Name, den Rückgabetyp und die möglichen Typen der Argumente. Eine beispielhafte Antwort auf eine Anfrage vom \texttt{/functions}-Endpunkt wird in Quelltext \ref{lst:functions} dargestellt. Dabei wird die Funktion \texttt{st\_MakePoint} aus \textit{PostGIS} und \texttt{date\_trunc} von \ac{SQL} beschrieben. Alle Argumente der beiden Funktionen nehmen nur einen Typ an - \citetitle{ogcFiltering} sieht jedoch auch die Möglichkeit vor, dass Argumente mehrere Datentypen annehmen können \parencitealias{ogcFiltering}.

\lstinputlisting[
  label=lst:functions,
  caption={[Antwort auf eine \texttt{/functions}-Anfrage im SimplexService]Antwort auf eine \texttt{/functions}-Anfrage im SimplexService (Auszug). Dargestellt ist die PostGIS-Funktion \texttt{ST\_MakePoint} und PostgreSQL-Funktion \texttt{DATE\_TRUNC}. Es werden nur die Pflichtinformationen über die Datentypen des Rückgabewertes und der Parameter angegeben.},
  language=json,
  escapeinside=``
]{assets/functions.json}

\citetitle{ogcFiltering} sieht für Argumente \texttt{type} als einzige Pflichtinformation vor \parencitealias{ogcFiltering}. Es kann jedoch auch ein Titel und eine Beschreibung für Argumente definiert werden. \parencitealias{ogcFiltering} Im SimplexService (siehe \ref{sec:simplex-service}) ist dies aktuell nicht der Fall. Es werden nur die Argumenttypen ausgegeben, ohne weitere Metadaten. Die verfügbaren Typen in Funktionsbeschreibungen lauten: \texttt{string}, \texttt{number}, \texttt{integer}, \texttt{datetime}, \texttt{geometry} und \texttt{boolean} \parencitealias{ogcFiltering}. Da keine Unterscheidung zwischen verschiedenen Datums- und Uhrzeit-Typen gemacht wird, ist es nicht möglich anzugeben, dass eine Funktion nur ein Datum, oder nur eine Uhrzeit annimmt. Das gleiche gilt auch für verschiedene Geometrietypen, die alle unter dem Typ \texttt{geometry} zusammengefasst werden.


% \clearpage
% \section{(Human-Centered Design)}
% \subsubsection{Begriffe und Konzepte aus Design of Everyday Things}
% \subsubsection{Abschnittstitel nicht final, vielleicht is HCD auch ein Unterpunkt auf gleicher Höhe wie Aff., Sign. und Map.}
% \subsection{Affordances, Signifiers und Mappings}
% \subsubsection{Kann gut bei Formularen angewendet werden}
% \subsection{Activity-Centered Design}
%
% \clearpage
% \section{(Formulare)}
% \cite{wroblewskiWebForm2008}
% \subsection{Funktionsweise}
% \subsubsection{Technische Fkt.-weise von F.}
% \subsection{Designanfordungen}
% \cite{normanDesignEveryday2013} (auf Affordances, Signifiers und Mappings eingehen)
% \subsubsection{Rückbezug auf HCD}
% \subsubsection{Anordnung}
% \subsubsection{Kontrollelemente}
% \subsubsection{Bezeichnungen und Platzhalter}
% \subsubsection{Validierung}

\clearpage
\section{Usability Testing}

Um auf \textit{Usability Testing} einzugehen, lohnt es sich, einen genaueren Blick auf \textit{Usability} zu werfen. \citefield{ISO924111}{shorttitle} definiert \textit{Usability} wie folgt:
\begin{quote}
  extent to which a system, product or service can be used by specified users to achieve specified goals with effectiveness, efficiency and satisfaction in a specified context of use
  \hspace*{\fill{}}\citetalias{ISO924111}
\end{quote}
Diese Definition ist kurz, umfasst jedoch trotzdem drei Hauptelemente, die \textit{Usability} ausmachen. Es sollte möglich sein, Aufgaben mit ausreichender Geschwindigkeit auszuführen (\textit{efficiency}) und das erwünschte Ziel zu erreichen (\textit{effectiveness}). Der dritte wichtige Faktor ist die Anwenderzufriedenheit (\textit{satisfaction}), welche mehr auf die subjektive Erfahrung von Nutzer:innen eingeht. So kann eine Aufgabe schnell und korrekt ausgeführt werden, aber dennoch nicht zu Zufriedenheit führen, da beispielsweise wichtige Informationen nicht angezeigt wurden, die über den Erfolg informieren. Der gleiche Effekt kann auch andersherum stattfinden - sollte die persönliche Einschätzung von Zufriedenheit sehr gut sein, können womögliche Probleme der Effizienz und Effektivität vernachlässigt werden, sodass das Produkt trotzdem als gut eingeschätzt wird.
\parencite{barnumUsabilityTesting2021}

Des Weiteren betont \citeauthor{barnumUsabilityTesting2021}, dass es sich um spezifische Nutzer:innen, Ziele und Nutzungskontexte handelt. Das bedeutet, dass \textit{Usability} nur für eine spezifische Gruppe an Menschen betrachtet werden kann, welche das Produkt benutzen werden und deren Ziele denen des Produkts entsprechen müssen. Außerdem kann die \textit{Usability} auch nur für den angedachten Nutzungskontext - die Umgebung in der ein Produkt genutzt wird - betrachtet werden.
\parencite{barnumUsabilityTesting2021}

\textcite{quesenberyDimensionsUsability2003} und \textcite{nielsenUsabilityEngineering1994} spezifizieren folgende 5 Attribute der \textit{Usability}:
\begin{itemize}
  \item \textbf{Effektivität}: wie umfassend und akkurat Aufgaben abgeschlossen werden,
  \item \textbf{Effizienz}: wie schnell dies erfolgt,
  \item \textbf{Ansprechende Gestaltung}: wie gut Interaktionen gesteuert werden und für Zufriedenheit gesorgt wird (von \citeauthor{nielsenUsabilityEngineering1994} vernachlässigt)
  \item \textbf{Fehlertoleranz}: wie das System Fehler verhindert und sich von Fehler erholt
  \item \textbf{Erlernbarkeit}: wie intuitiv das System ist, und wie Lernen im Laufe der Nutzung ermöglicht wird (von \citeauthor{nielsenUsabilityEngineering1994} in zwei Attribute geteilt)
\end{itemize}
Laut \textcite{barnumUsabilityTesting2021} hat der Faktor der Zufriedenheit in den letzten Jahren an Bedeutung zugenommen. Das wird auch davon unterstützt, dass \textcite{nielsenUsabilityEngineering1994} weniger Wert auf eine ansprechende Gestaltung legte als \textcite{quesenberyDimensionsUsability2003}.

\newpage

\textit{Usability Testing} ist ein Sammelbegriff für Methoden bei denen Benutzer:innen mit einem System interagieren, um ein Ziel zu erreichen oder Szenario umzusetzen. Dabei sind die Umstände der Interaktionen kontrolliert und Verhaltensdaten werden gesammelt \parencite{wichanskyUsabilityTesting2000}. Ziel ist es, die \textit{Usability} des Systems zu überprüfen und eventuelle Probleme zu identifizieren.

Dabei ist es laut \textcite{barnumUsabilityTesting2021} wichtig, dass die Nutzer:innen Aufgaben erfüllen, die als echt empfunden und von Bedeutung für sie sind. So werden Probleme, die bei der Nutzung in der realen Welt auftreten effizienter aufgedeckt.

Wichtige Aspekte des \textit{Usability Testing} werden im weiteren Verlaufe des dieses Kapitels beleuchtet.

% Quantitative vs Qualitative Tests
% Formative vs Summative Studien

\subsection{Testmethoden}
intro...
\subsubsection{Formative und Summative Studien}
\label{section:formative-summative}
Usability Tests können in zwei Typen unterteilt werden: \textbf{Formative} und \textbf{Summative Studien}.

\textbf{Formative Studien} werden während der Entwicklung durchgeführt, um Probleme zu diagnostizieren und zu lösen. Der Umfang der einzelnen Studien ist relativ klein, im Entwicklungszyklus können sie jedoch meist wiederholt durchgeführt werden. Die Ergebnisse von formativen Studien können direkt in die Entwicklung einfließen, wobei erneute Studien überprüfen können, ob die Probleme aus vergangen Durchläufen behoben wurden. Des Weiteren können formative Studien aufzeigen, was Nutzer*innen wichtig ist, und somit die Entwicklung in eine Richtung lenken, die die Usability des Endprodukts drastisch verbessert. Erkenntnisse sind in zeitigen Stadien der Produktentwicklung besonders wichtig, da es im Verlauf des Entwicklungszyklus immer schwieriger wird, grundlegend falsche Entscheidungen zu berichtigen.

Für die Zwecke dieser Arbeit ist eine formative Studie angemessen. Daher werden Schlüsselelemente von formativen Studien, wie zum Beispiel Benutzerprofile, Szenarien und das \acl{TA}, im weiteren Verlauf detailliert vorgestellt.
\todo{Link}
\todo{drauf eingehen wieso angemessen}

\textbf{Summative Studien} werden durchgeführt wenn ein Produkt (fast) fertiggestellt ist. Dabei soll unter Usability und Funktionalität auch die Zufriedenheit der Nutzer*innen bewertet werden. Diese Art von Studie findet in einem größeren Rahmen statt und erfordert in der Regel viele Teilnehmer, um statistische Relevanz zu erreichen. Ähnlich zu formativen Studien werden Teilnehmer*innen Szenarien und Aufgaben bereitgestellt, darüber hinaus wird jedoch nicht weiter mit ihnen interagiert. Somit können Metriken wie die Zeit für die Bearbeitung von Aufgaben und Abschlussquoten erhoben werden.

Die Ergebnisse von summativen Studien sind meistens numerisch, während Ergebnisse von formativen Studien konkrete Problembeschreibungen beinhalten. Es ist also einfacher, summativen Studien miteinander zu vergleichen, während formative Studien konkretere Probleme auflisten und in einer agilen Entwicklungsumgebung zeitig und schnell zu Verbesserungen führen können.
\todo{Agil - soll nicht so wirken als wäre agil dafür wichtig?}
\todo{Ergebnisse besser erklären}

\parencite{barnumUsabilityTesting2021}

\subsubsection{\acl{TA}}
Das \ac{TA} ist eine weit verbreite Methode, um die Usability von Applikationen einzuschätzen. Dabei sollen Nutzer*innen ihre Erfahrungen, Gedanken, Handlungen und Gefühle beim Interagieren mit der Anwendung ausdrücken. Auf diese Art gesammelte Erkenntnisse \todo{wording} bieten Einblick in Denkprozesse von Nutzer*innen und können somit Entwicklungsentscheidungen lenken.
\parencite{alhadretiRethinkingThinking2018}

Im Folgenden werden anhand von drei Studien verschiedene Funktionsweisen von \ac{TA} vorgestellt, und Auswirkungen auf den Usability Testing Prozess beleuchtet.

Typischerweise werden zwei Typen von \ac{TA} unterschieden: \ac{CTA} und \ac{RTA}. Bei \ac{CTA} teilen die Teilnehmer*innen ihre Gedanken sofort beim Ausführen der Aufgaben, während dies bei \ac{RTA} erst im Anschluss daran passiert. Einerseits kann \ac{CTA} es einfacher machen Problembereiche zu identifizieren, da die zusätzlichen Informationen in Echtzeit bereitgestellt werden, andererseits existieren folgende Bedenken: Das zum Handeln parallel Sprechen könnte ungewohnt und irritierend für Teilnehmer*innen sein und dadurch zu einer unangenehmen Situation führen. Des Weiteren könnte \ac{CTA} Denkprozesse aktiv beeinflussen und somit die Art und Weise in der Aufgaben abgearbeitet werden verändern. Auch die zusätzliche Zeit die dadurch oft in Anspruch genommen wird sollte bei Betrachtung der Aussagekraft der erhobenen Daten beachtet werden. \todo{zu verworren?} Im Gegensatz dazu beeinflusst \ac{RTA} die Denkprozesse von Teilnehmer*innen nicht, und leidet somit nicht unter den möglichen Problemen \ac{CTA}. Problematisch an dieser Methode ist jedoch, dass sich Teilnehmer*innen korrekt an die Ausführung der Aufgabe erinnern müssen. Außerdem besteht die Gefahr, dass spätere Handlungen und Teilschritte vorherige Gedanken beeinflussen oder relativieren. So könnten wichtige Erkenntnise verloren gehen. Vereinzelt wird auch auf einen hybriden Ansatz gesetzt, bei dem die Erkenntnise von \ac{CTA} durch eine Nachbesprechung angereichert werden. \cite{alhadretiRethinkingThinking2018}

\textcite{alhadretiRethinkingThinking2018} verglichen in einer Studie \ac{CTA}, \ac{RTA} und einen hybriden Ansatz miteinander. Dabei wurden folgende Metriken betrachtet: Allgemeine Leistung beim Ausführen von Aufgaben, Erfahrungen der Teilnehmer*innen, Menge und Qualität der gefundenen Usability-Probleme, sowie die Kosten zur Nutzung der Methoden. Am besten schnitt das \acl{CTA} ab, da es am meisten Probleme fand und das beste Feedback von Teilnehmer*innen erlangte \todo{erlangte??}. Es konnte auch kein Unterschied in der Leistung der Teilnehmer*innen im Vergleich zu Durchläufen ohne \ac{TA} festgestellt werden. Außerdem ist es vorteilhaft, \ac{CTA} gegenüber \ac{RTA} und einem hybriden Ansatz zu wählen, da diese doppelt so viel Test- und Analysierungszeit benötigten.
\parencite{alhadretiRethinkingThinking2018}

Abgesehen von den soeben betrachteten Typen des \acl{TA} existieren auch Unterschiede im Umfang der Interaktion zwischen Testadministrator*innen und Teilnehmer*innen. \textcite{olmsted-hawalaThinkaloudProtocols2010} testeten folgende auftretenden Kategorien, räumten jedoch ein dass die Realität oft nuancierter ist:
\begin{itemize}
  \item Traditionell: Außer "Bitte reden Sie weiter." werden keine weiteren Nachfragen angestellt.
  \item Sprach-Kommunikation: In Form von "um-hum" oder "un-hum" werden Teilnehmer*innen zum Weiterreden angeregt. Nach ausreichend langen Stillephasen wird mit dem zuletzt geäußerten Wort das \acl{TA} weiter angeregt. \footnote{Eine Interaktion könnte beispielsweise so aussehen:\\ Teilnehmer*in: "Das war komisch..." \\ -- 15 Sekunden Stille -- \\ Administrator*in: "Komisch?"} \todo{akzeptable Übersetzung von "Speech Communication"?}
  \item Coaching: Es wird aktiv in eingegriffen, indem direkte Fragen gestellt werden und assistiert wird, falls Teilnehmer*innen zu große Schwierigkeiten haben.
\end{itemize}
Die Ergebnisse der \citeyear{olmsted-hawalaThinkaloudProtocols2010} durchgeführten Studie zeigen, dass Teilnehmer*innen, bei denen Coaching angewendet wurde, erfolgreicher als andere waren und zufriedener mit der getesteten Website waren. Die Zeit zur Beendigung der Aufgaben unterschied sich nicht zwischen den verschiedenen Herangehensweisen und, genauso wie bei \citeauthor{alhadretiRethinkingThinking2018}, konnte kein Unterschied zur Kontrollgruppe (ohne \ac{TA}) festgestellt werden. Eine Beeinflussung der Ergebnisse scheint also nur beim Coaching aufzutreten - während die restlichen Variationen des \ac{TA} weder die Genauigkeit noch die Geschwindigkeit abändern. Es wird betont, dass eine genaue Beschreibung des verwendeten \acl{TA} wichtig ist, um einen hohen Grad an Reproduzier- und Vergleichbarkeit zu erreichen.

Eine Studie von \textcite{rheniusEvaluationConcurrent1990} kam zu leicht anderen Ergebnissen. Es wurde festgestellt, dass die Zeit zum Lösen von Aufgaben bei der Verwendung von \ac{TA} steigt, die Genauigkeit jedoch nicht davon beeinflusst wurde. \citeauthor{rheniusEvaluationConcurrent1990} kamen auch zu dem Schluss, dass der Einfluss von \ac{TA} nur in den frühen Phasen der Gewöhnung an die Aufgaben auftritt.
\parencite{rheniusEvaluationConcurrent1990}

Somit kann also festgehalten werden, dass die Verwendung von \ac{TA} wenn überhaupt nur eine minimale Auswirkung auf die Zeit zur Fertigstellung von Aufgaben hat. Eine Beeinflussung auf die Testergebnisse konnte in keiner der Studien festgestellt werden, außer bei der Verwendung von Coaching. Das \acl{TA} stellt sich somit als gutes Werkzeug zur Durchführung von Usability Studien heraus. Insbesondere eine Verwendung von \acl{CTA} und eine Beschränkung auf minimale Interaktionen mit den Teilnehmer*innen scheint vorteilhaft.

\subsection{Studiengröße}
\label{section:study-size}

Wie in \ref{section:formative-summative} erwähnt, existieren Usability Tests in unterschiedlichen
Umfängen. Formative Studien sind klein und werden meist während der Entwicklung durchgeführt,
während summative Studien mehr Teilnehmer:innen umfassen und gegen Ende der Entwicklung stattfinden.

Für diese Arbeit werden formative Studien von Relevanz sein. Dabei sollte die Anzahl der
Teilnehmer:innen so klein wie möglich zu halten, während so viele Usability Probleme wie möglich
entdeckt werden, denn eine höhere Anzahl an Testdurchläufen steigert die Kosten und den Zeitaufwand.
\parencites{faulknerFiveuserAssumption2003, nielsenWhyYou2000}

\textcite{nielsenMathematicalModel1993} beschreiben den Anteil an gefundenen Usability Problemen bei
$n$ Teilnehmer:innen als wie folgt:
\begin{equation}
  \label{equation:finding-usability-problems}
  N(1-(1-\lambda{})^n),
\end{equation}
wobei $N$ die Gesamtanzahl der Probleme und $\lambda{}$ der Anteil der Probleme die bei einem
einzigen Testdurchlauf gefunden werden. Mit dem von ihm über mehrere Projekte beobachtete Wert von
$\lambda{}=31\%$ kommt \textcite{nielsenWhyYou2000} zum Resultat, dass mit fünf Teilnehmer:innen
85\% aller Probleme festgestellt werden können. \cite{nielsenWhyYou2000} Weitere Testdurchläufe sind
weniger lohnenswert, da nicht genug neue Probleme aufgedeckt werden. Um alle Probleme mit einer
Studie zu finden, wären 15 Teilnehmer:innen nötig, die gegen Ende jedoch nur noch vereinzelt neue
Erkenntnisse liefern. Da Teilnehmer:innen oft die gleichen, oberflächlichen Fehler finden und
dadurch abgelenkt werden, ist es von Vorteil nach den ersten fünf Durchläufen die gefundenen 85\% zu
beheben, und dann eine weitere Studie durchzuführen. Eine dreifache Wiederholung der Tests mit fünf
Durchläufen und einem verbesserten Produkt führt zu umfangreicheren Resultat als ein einziger Test
mit 15 Durchläufen.
\parencite{nielsenWhyYou2000}
\todo{vielleicht update von 2012 mit einbeziehen? \cite{nielsenHowMany2012}}

Die als Faustregel adaptierte Zahl Fünf wurde in mehreren Publikationen untersucht. Während fünf
Teilnehmer:innen durchschnittlich 85\% der Probleme (einer Studie) entdecken können, kann dieser
Prozentsatz laut \textcite{faulknerFiveuserAssumption2003} jedoch bis auf 55\% sinken. Während die
durchschnittliche Rate langsamer steigt, verringert sich die Varianz mit mehr Testdurchläufen.
Außerdem ist anzumerken, dass die Frage, wie viele Teilnehmer:innen benötigt werden schwer zu
beantworten ist, und von vielen Faktoren abhängt. \footnote{genannt werden: Art und Erfahrung der
Teilnehmer:innen, Wichtigkeit des Systems, mögliche Folgen von Usability Problemen}
\parencite{faulknerFiveuserAssumption2003}

\textcite{spoolTestingWeb2001} fanden einen deutlich geringeren Wert von $\lambda{}=10\%$ bei
unstrukturiertem Testen von E-Commerce-Webseiten. Somit würden mit 5 Testdurchläufen nur 35\% der
Probleme gefunden werden. Als Gründe dafür kann die Komplexität der Webseite und die persönlich
getroffenen Entscheidungen der Teilnehmer:innen genannt werden.
\parencite{spoolTestingWeb2001}

Laut \textcite{woolrychWhyWhen2001} ist die Wahrscheinlichkeit, dass ein Problem entdeckt wird,
nicht konstant sondern hängt von Schweregrad, Benutzermerkmalen, Produkttyp und Strukturierungsgrad
des Testes ab. Sie erweitern deshalb Gleichung \ref{equation:finding-usability-problems} um
$\lambda{}_j$ für jedes Usability Problem, welches die Wahrscheinlichkeit darstellt, dass Problem
von einer zufällig ausgewählten Teilnehmer:in gefunden wird. Die erwartete Nummer an gefundenen
Problemen beträgt dann laut \textcite{woolrychWhyWhen2001}:
\begin{equation}
  \sum_{j=1}^n 1-(1-\lambda{}_j)^n
\end{equation}
Um Gleichung \ref{equation:finding-usability-problems} zu reproduzieren, müsste $\lambda{}_j =
\lambda{}$ für alle $j$ gelten. Das bedeutet, dass die Wahrscheinlichkeiten, einzelner Probleme
gefunden zu werden, eine geringe Variabilität besitzen. Die von
\textcite{nielsenMathematicalModel1993} eingeführte Gleichung
\ref{equation:finding-usability-problems} wird \textcite{woolrychWhyWhen2001} zufolge
unzuverlässiger, sobald eine größere Varianz im System und bei den Teilnehmer:innen existiert. Sich
unterscheidende Interaktionen mit dem System und die Auswahl der Aufgaben beinflussen die
Zuverlässigkeit eines einzigen Wertes $\lambda{}$.
\parencite{woolrychWhyWhen2001}

\todo{Fazit für den Abschnitt? Auf iteratives Testen (\textcite{barnumUsabilityTesting2021}, S.21)
eingehen! \textcite{barnumUsabilityTesting2021} stimmt eher
\textcite{nielsenMathematicalModel1993} zu, aber mit Abhandlung von vorher passt das nicht mehr
zusammen}

% \subsection{Personas}
\todo{find more sources? basically only citing \textcite{tomlinUXOptimization2018}}

Das Konzept von Personas wurde zuerst von \textcite{cooperInmatesAre1999}
eingeführt. Er erstellte sie, um eine fiktive Person als eine Zusammenfassung von gemeinsamen
Bedürfnissen, Hintergründen und Vorwissen zu erstellen und somit zu definieren, was eine "typische"
Nutzer:in tun müsste, um Erfolg mit einer Anwendung zu haben. Auch er gab dieser Persona bereits
eine Hintergrundgeschichte, Ziele und Gründe zum Benutzen der Applikation. Sie vereinfachen das
Design und die Entwicklung indem sie den Fokus von den Funktionalitäten auf die Endnutzer:innen
lenken und die Frage stellen, was sie benötigen, um erfolgreich zu sein. Somit sind Personas ein
wichtiges Werkzeug des \textit{"user-centered design"}. \parencite{cooperInmatesAre1999,
  tomlinUXOptimization2018}

Für Usability Testing stellen Personas eine wichtige Rolle dar, da sie helfen, die richtigen
Teilnehmer:innen zu finden.
\parencite{tomlinUXOptimization2018}

Laut \textcite{tomlinUXOptimization2018} werden gute Personas durch mehrere Merkmale geprägt. Sie
basieren auf Feldforschung, bei der Informationen von potenziellen Nutzer:innen gesammelt werden, um
gemeinsame Verhaltensmuster zu identifizieren. Personas sollten sich auf den gegenwärtigen Zustand
fokussieren und die typische Umgebung bei der Verwendung der Anwendung beinhalten, sowie die
benutzten Geräte. Des Weiteren werden ein Bild, Name und eine kurze Geschichte dazu verwendet, die
beschriebene Person zu vermenschlichen. Dabei sollte ein typisches Problem oder eine Aufgabe
beschrieben werden, die es gilt zu überwinden. Die zwei oder drei wichtigsten Aufgaben können auch
in einer Reihenfolge priorisiert werden. Zu guter Letzt betont \textcite{tomlinUXOptimization2018},
dass gute Personas auch Details über den typischen Grad der Expertise beinhalten sollten. Diese
Informationen sollten detailliert auf die Anwendung und die Problemstellung angepasst werden.
\parencite{tomlinUXOptimization2018}

\subsubsection{Persona-Arten} \todo{besserer Abschnittsname?}

\textcite{tomlinUXOptimization2018} identfiziert drei verschiedene Arten von Personas: Design und
Marketing Personas, sowie Proto-Personas. Im Folgenden sollen diese vorgestellt werden.

\textbf{Design Personas} stellen die Nutzer:innen, ihre Ziele und Aufgaben in den Mittelpunkt. Laut
\textcite{tomlinUXOptimization2018} ist es am Wichtigsten, dass Design Personas auf Feldforschung
und Beobachtung von Nutzer:innen in ihrer eigenen Umgebung basieren. Diese Personas stellen Kontext
für Designentscheidungen dar und können den Einfluss von Einzelpersonen einschränken, indem sie eine
fundierte Basis bilden. \todo{Formulierung} Da sie sich auf wichtige Aufgaben konzentrieren, kann
durch Design Personas der Entwicklungsprozess optimiert und in die richtige Richtung gelenkt werden.
Das betrifft sowohl Designentscheidungen, als auch Usability Studien. Abschließend ist anzumerken,
dass eine gute Persona dieser Art ein wichtiges Werkzeug für \textit{user-centered design}
darstellt.
\parencite{tomlinUXOptimization2018}

\textbf{Marketing Personas} unterscheiden sich von Design Personas, da sie den Fokus auf das
Kaufverhalten, Meinungen und Einstellungen der potentiellen der potentiellen Kundschaft legen. Daher
beinhalten sie typischerweise nicht die wichtigsten Aufgaben, sondern Informationen über
Kaufmotivationen und -bedürfnisse. Anstatt von Feldforschung werden oft quantitative Datensätze
basierend auf Primär- oder Sekundärforschung verwendet. \textcite{tomlinUXOptimization2018}
beschreibt dass es selten möglich ist, Marketing Personas anstelle von Design Personas zu benutzen,
da sie wichtige Aufgaben beschreiben und auf Feldforschung basieren sollten.
\parencite{tomlinUXOptimization2018}

\textbf{Proto-Personas} werden von \textcite{tomlinUXOptimization2018} als am Weitesten verbreitet
angesehen, da sie kostengünstig sind und nicht viel Zeit benötigt wird, um sie zu erstellen.
Proto-Personas basieren oft auf Sekundärforschung, oder sogar nur auf der Intuition der
Designer:innen. Obwohl die Verwendung einer Proto-Persona einem Vorgehen ohne Personas vorzuziehen
ist, ist es empfehlenswert, sie im Laufe der Zeit zu validieren und zu verbessern. Sie können somit
einen guten Startpunkt für agile Prozesse in Design und Entwicklung darstellen und nach einiger
Zeit, mit gesammelten Erkenntnissen, zu Design Personas aufgewertet werden.
\parencite{tomlinUXOptimization2018}

Im Rahmen dieser Arbeit sind \textbf{Design Personas} und \textbf{Proto-Personas} von Interesse.

\todo{Überlegen ob dem Persona Abschnitt nochwas fehlt}
\todo{Oder auch, ob nach Personas noch ein Abschnitt fehlt (vor Aufgaben und Scenarios)}

\subsection{Szenarios}
\label{sec:scenarios}

Damit die Benutzung einer Anwendung bei Usability Tests beobachtet und bewertet werden kann, müssen den Teilnehmer:innen spezifische und realistische Aufgaben gegeben werde. Dies betonen \textcite{mccloskeyTaskScenarios2014} und \textcite{barnumUsabilityTesting2021}. Szenarios werden verwendet, um die Aufgaben mit einer Erklärung und Kontext anzureichern \parencite{mccloskeyTaskScenarios2014}. Sie bilden somit einen Weg den Nutzer:innen zu beschreiben, welche Aktionen sie in der Anwendung ausführen sollen, ohne ihnen konkret zu beschreiben, was sie tun sollen.

\textcite{mccloskeyTaskScenarios2014} beschreibt drei Eigenschaften von guten Aufgaben: Sie sollten realistisch sein, als Handlung ausführbar sein und es vermeiden Tipps zu geben oder Unterschritte zu beschreiben.

Eine realistische Aufgabe macht es einfacher für Nutzer:innen sich in das Szenario zu versetzen und die Aufgabe auszuführen, während sie die Oberfläche benutzen. Es kann besser sein, den Teilnehmer:innen etwas Spielraum in den Details der Aufgabe zu geben, damit sie so vorgehen können, wie es für sie realistisch ist \parencite{mccloskeyTaskScenarios2014}. Damit Szenarios realistischer wirken, empfiehlt \textcite{barnumUsabilityTesting2021}, die Sprache der Nutzer:innen statt der des Produktes zu verwenden. Zweiteres kann auch die unerwünschte Folge haben, dass bereits Informationen über den vorgesehenen Lösungsweg offenbart werden \parencite{barnumUsabilityTesting2021}.

Außerdem ist es besser, nach einer Handlung zu fragen, statt sie beschreiben zu lassen. \textcite{nielsenFirstRule2001} geht darauf ein, dass dies immer besser ist, da eine Einschätzung nicht so genaue Ergebnisse birgt wie die Beobachtung der Handlung. Grund für diese Annahme ist die Forschungsarbeit von \textcite{nielsenMeasuringUsability1994}, aus der hervorgeht, dass Nutzer:innen bei zwei unterschiedlichen Designs eine Meinung darüber abgeben können, welches besser ist, aber diese Information wird immer genauer sein, wenn sie die Möglichkeit hatten, die Oberfläche selbst zu benutzen. Die Interaktion mit der Anwendung kann Einblicke in Problembereiche und Frustrationen bieten, die durch eine einfache Beschreibung nicht verfügbar gewesen wären \parencite{mccloskeyTaskScenarios2014}.

Zu guter Letzt sollte es vermieden werden, den Teilnehmer:innen in der Aufgabenbeschreibung Tipps zu geben oder den Lösungsweg zu detailliert auszuführen. Ausdrücke, die in der Benutzeroberfläche verwendet werden, sollten vermieden werden, da so das Verhalten der Teilnehmer:innen beeinflusst wird \parencite{mccloskeyTaskScenarios2014, barnumUsabilityTesting2021}. Beschreibungen von Unterschritten enthalten oft Tipps über den Aufbau und die vorgesehene Art und Weise der Anwedungsbenutzung. Dies kann dazu führen, dass weniger hilfreiche Informationen über das Verhalten der Nutzer:innen gewonnen werden. \textcite{barnumUsabilityTesting2021} betont deshalb, dass es besser ist ein Ziel anzugeben, statt einer Reihenfolge an Schritten um dieses Ziel zu erreichen \parencite{mccloskeyTaskScenarios2014}.

Eine Usability Studie besteht typischerweise aus mehreren Szenarios. \textcite{barnumUsabilityTesting2021} geht darauf ein, in welcher Reihenfolge Szenarios organisiert werden können. Das erste Szenario kann kurz gehalten werden, und auf den allgemeinen ersten Eindruck der Anwendung eingehen. Ein kurzes erstes Szenario gibt den Teilnehmer:innen die Möglichkeit sich mit dem Prozess und den Methoden der Studie vertraut zu machen, und unvorhergesehene Probleme technischer oder logistischer Art können gelöst werden \parencite{barnumUsabilityTesting2021}. Außerdem empfiehlt es sich Missverständnisse nach dem ersten, kurzen Szenario aus dem Weg zu räumen, damit während einem längeren Szenario keine Unterbrechung entsteht, die den Gedankenfluss und die Ergebnisse stören könnte \parencite{barnumUsabilityTesting2021}. Die nachfolgenden Szenarios können aufeinander aufbauen, nach Häufigkeit der Ausführung beziehungsweise Wichtigkeit der Aufgabe sortiert werden, oder einem vorgesehenen Aufgabenfluss folgen, der somit gleichzeitig überprüft wird \parencite{barnumUsabilityTesting2021}.

Laut \textcite{barnumUsabilityTesting2021} ist das Ende von Szenarios auch von Bedeutung für die Ergebnisse einer Usability Studie. Unsicherheit bei den Teilnehmer:innen, ob eine Aufgabe abgeschlossen ist, ob sie korrekt durchgeführt wurde oder noch nicht beendet ist, kann auf Probleme im Design der Anwendung hindeuten. Deshalb ist es wichtig diese Informationen zu erheben, darauf zu achten wie die Aufgaben beendet werden, und den Teilnehmer:innen die Möglichkeit zu geben, zu entscheiden, wann sie eine Aufgabe absolviert haben \parencite{barnumUsabilityTesting2021}.

Gute gewählte und formulierte Aufgaben haben den Vorteil, dass Verhaltensmuster, und häufig auftretende Muster erkannt werden können \parencite{barnumUsabilityTesting2021}. Des Weiteren besteht die Möglichkeit, die Erfolgsrate als grobes Maß für die Usability zu benutzen \parencite{nielsenSuccessRate2001}. Andere Maße und Informationen sollten jedoch weiterhin gesammelt werden, da Erfolg nur von einem Mindestmaß an Usability spricht \parencite{nielsenSuccessRate2001}.

\subsection{Feedback}
\todo{Performance Data wurde als Leistungsdaten übersetzt, und Preference data als
Präferenzinformationen - ok so?}

Feedback-Methoden können abhängig vom Typ der Studie variieren. Formative Studien profitieren von
qualitativen Methoden, während formative Studien quantitatives Feedback einsetzen. Laut
\textcite{barnumUsabilityTesting2021} kann in beiden Fällen eine Benutzung der jeweils anderen
Herangehensweise die Erkenntnisse erweitern. Wie bereits in \ref{section:formative-summative}
festgestellt, ist es in zeitigeren Entwicklungsstadien vorteilhafter, formative Studien
durchzuführen. \todo{Überprüfen ob a) stimmig und b) so beschrieben} Qualitative Aussagen von
Teilnehmer:innen können wertvolle Einblicke in die größten Problembereiche geben und auf die
nächsten Entwicklungsschritte hindeuten.
\parencite{barnumUsabilityTesting2021}

\textcite{barnumUsabilityTesting2021} nennt Leistungsdaten und Präferenzinformationen als
quantitative Maße für Usability. Als Leistungsdaten gelten zum Beispiel die Zeit zum Absolvieren von
Aufgaben, Fehlerquote oder Erfolgsrate. Wie bereits in \ref{section:scenarios} angeschnitten,
spricht eine hohe Erfolgsrate jedoch nur von einem Mindestmaß an Usability und sollte durch weitere
Werte unterstützt werden. Des Weiteren weißt \textcite{barnumUsabilityTesting2021} darauf hin, dass
auch unterschiedliche Grade von Erfolg beim Absolvieren von Aufgaben bestehen. So könnte eine
Nutzer:in den schnellsten, vom Design vorgesehenen Weg zum Ziel benutzen, oder aber auch einen
indirekten Pfad nehmen. Auch verschiedene in Anspruch genommene Hilfstellungen können von Usability
Problemen sprechen. Das Bewerten einer Aufgabe als Misserfolg kann auch verschiedene Gründe haben:
Aufgeben, Abruch durch die Testleitung, oder die Annahme, dass die Aufgabe beendet sei, ohne dass
sie das wirklich ist. Eine weitere Metrik die einfach zu erheben ist, ist die Zeit zum Vollenden der
Aufgaben. In der ersten Studie können diese Werte als Ausgangsbasis gesammelt werden, und in
zukünftigen Studien zum Vergleich benutzt werden, so \textcite{barnumUsabilityTesting2021}. Zu
beachten ist, dass diese einfach zu erhebenden Metriken nicht die gesamte Usability Erfahrung
beschreiben können.
\parencite{barnumUsabilityTesting2021}

Als Präferenzinformationen beschreibt \textcite{barnumUsabilityTesting2021} Antworten auf
Fragebögen, welche nach Aufgaben und nach dem kompletten Test erhoben werden. Befinden sie sich auf
Skalen (1 bis 5 oder 1 bis 10), können sie als quantitative Daten benutzt werden. Fragen mit offenem
Ende lieferen qualitative Informationen über die Erlebnisse der Teilnehmer:innen. Zusätzlich sollten
auch über Thinking-Aloud gewonnene Eindrücke berücksichtigt werden und, wenn verfügbar,
nonverbales Feedback wie Körperspräche oder nonverbale Ausdrücke gesammelt werden.
\parencite{barnumUsabilityTesting2021}

\subsubsection{Fragebögen}
Nach jedem Szenario Feedback über die Aufgabe zu sammeln, hat laut
\textcite{barnumUsabilityTesting2021} den Vorteil, dass die Erinnerungen noch frisch sind. Dabei
kann es sich auch um eine einzige Frage handeln. \textcite{sauroIfYou2010} nennt als Eigenschaften
eines guten Fragebogens, dass er zuverlässig, sensibel, valide und kurz ist, sowie einfach zu
beantworten, zu handhaben und zu bewerten sein sollte. \textcite{barnumUsabilityTesting2021} schlägt
vor eins oder mehr der folgenden Themen abzufragen: Schwierigkeit der Durchführung, benötigte Zeit
(von "weniger als erwartet" bis "mehr als erwartet"), die Wahrscheinlichkeit, dass dieses Feature
erneut benutzt wird, und das Vertrauen in die erfolgreiche Bewältigung der Aufgabe.

\textcite{sauroIfYou2010} listet folgende Standard-Fragebögen für das Einholen von Feedback nach
Aufgaben auf: \ac{ASQ}, \ac{NASA-TLX}, \ac{SMEQ}, \ac{UME} und \ac{SEQ}. \todo{bessere Überleitung?
maybe selbsterschließend?}

Der \textbf{\ac{ASQ}} wurde von \textcite{lewisPsychometricEvaluation1991} eingeführt und beinhaltet
drei Fragen bezüglich Schwierigkeit der Aufgabe, Bearbeitungszeit und der Menge der unterstützenden
Informationen (Dokumentation, Online-Hilfe, etc.). Die Antworten werden auf einer Skala von 1
(stimme voll zu) bis 7 (stimme überhaupt nicht zu) angegeben.

Der \textbf{\ac{NASA-TLX}} wurde von \textcite{hartDevelopmentNASATLX1988} entwickelt und berechnet
eine Gesamtbewertung der Arbeitsbelastung basierend auf, geistiger, physischer und zeitlicher
Anforderung, sowie Leistung, Aufwand und Frustration.
\parencite{nasaNASATLX}

Der \textbf{\ac{SMEQ}}, zuerst von \textcite{zijlstraConstructionScale1985} beschrieben, besteht aus
einer einzelnen Skala von "gar nicht anstrengend" bis "extrem anstrengend".
\todo{schwer vs schwierig?}

Bei \textbf{\ac{UME}} handelt es sich um eine Methode, bei der nach jeder Aufgabe ein numerischer
Wert von den Teilnehmer:innen erfragt wird, welcher sich auf die Aufgabe bezieht. Die erste Antwort
kann arbiträr sein, die restlichen orientieren sich dann aber an den vorherigen
Antworten - somit können die Aufgaben untereinander verglichen werden. Zu
betonen ist, dass die Skala im Vorhinein nicht festgelegt ist, und von den
Teilnehmer:innne selbst gewählt wird.
\parencite{mcgeeUsabilityMagnitude2003}

Die \textbf{\ac{SEQ}} besteht aus einer einzigen Frage: "Insgesamt war diese Aufgabe...",
wobei von einer Skala von 1 (sehr schwierig) bis 5 (sehr einfach) ausgewählt werden kann.
\textcite{tedescoComparisonMethods2006} stellte fest, dass diese Methode bei einer kleinen
Stichprobengröße die beständigsten Ergebnisse liefert \footnote{Es fand ein Vergleich mit 4 anderen
Methoden statt: \ac{ASQ}, \ac{UME}, eine Variation von \ac{SEQ} und die Erwartungs-Bewertung von
\textcite{albertThisWhat2003}}. \textcite{sauroComparisonThree2009} verglich die \ac{SEQ} mit
\ac{SMEQ} und \ac{UME}. Dabei punktete \ac{SEQ} mit einfacher Bedienbarkeit und Erlernbarkeit,
während alle Methoden eine ähnliche Zuverlässigkeit aufwiesen.

\textcite{barnumUsabilityTesting2021} erklärt, dass Fragebögen nach Szenarios an die soeben
absolvierte Aufgabe angepasst werden können. Während das gleiche für Fragebögen nach dem gesamten
Test gilt, existieren auch da Standard-Fragebögen, wie \ac{SUS}, \ac{CSUQ} und \ac{NPS}.

\textbf{\ac{SUS}} wurde von \textcite{brookeSUSQuick1996} \todo{nochmal nachschauen ob das wirklich
erste Publikation ist, weil \textcite{barnumUsabilityTesting2021} (S.233) von 1986 schreibt (auch in
Quellen)} vorgestellt und besteht aus 10 Fragen, die auf einer Likert-Skala bewertet werden.
\todo{Muss Likert-Skala erklärt werden?} Um den Gesamtwert-Wert zu berechnen, werden die einzelnen
Antworten miteinander addiert und mit 2,5 multipliziert, um einen \ac{SUS}-Wert zwischen 0 und 100
zu erhalten. \textcite{sauroMeasuringUsability2011} führte eine Meta-Analyse von 500 Studien durch
und errechnete einen durchschnittlichen \ac{SUS}-Wert von 68. Dieser Werte könnte laut
\textcite{barnumUsabilityTesting2021} als Basiswert für iterative Studien benutzt werden.
\todo{überdurschnittlich und so erklären?}

\textcite{brookeSUSQuick1996} beschreibt folgende Vorgehensweise zur Anwendung von \ac{SUS}: Die
Fragen sollten direkt nach dem Test gestellt werden, bevor etwaige andere Aktivitäten durchgeführt
werden. Des Weiteren sollte Teilnehmer:innen ihre direkte Antwort geben, statt zu viel darüber
nachzudenken. Schlussendlich betont \textcite{brookeSUSQuick1996}, dass optimalerweise eine Antwort
auf jede Frage gegeben wird - im Falle von Meinungslosigkeit kann das der Mittelpunkt sein. Laut
\textcite{barnumUsabilityTesting2021} ist die \ac{SUS} sehr weit verbreitet, da sie schnell
anzuwenden ist, unabhäng von der Technologie des getesteten Produktes ist und sich über die Zeit als
valide Methode für Studien ab fünf Teilnehmer:innen erwiesen hat.

\textbf{\ac{CSUQ}} ist sehr ähnlich zum \ac{PSSUQ}. Beide beinhalten 16 positive Fragen, welche auf
einer Skala mit 7 Punkten bewertet werden. Im Kontrast zu \ac{SUS} wird auch eine Bewertung von
"nicht anwendbar" erlaubt. Die Fragebögen ergeben einen Gesamtwert und drei Unterbwertungen von
Systemqualität, Informationsqualität und Oberflächenqualität. \parencite{barnumUsabilityTesting2021}
\todo{vielleicht auch noch Originalquelle für \ac{CSUQ}/\ac{PSSUQ} finden?}

Beim \textbf{\ac{NPS}} handelt es sich um eine einzige Frage, die ungefähr wie folgt lautet: "Wie
wahrscheinlich ist es, dass Sie [Unternehmen/Produkt] weiterempfehlen?". Dabei wird die Antwort auf
einer Skala zwischen 0 und 10 gegeben. Teilnehmer:innen, die einen Wert unter 7 angeben, werden als
\textit{Detractors} bezeichnet, bei einer Antwort über 8 werden sie als \textit{Promoters}
eingestuft. Der \ac{NPS}-Wert wird aus einer Subtraktion des Prozentsatzes der \textit{Detractors}
vom Prozentsatz der \textit{Promoters} gewonnen und kann somit zwischen -100 und 100 liegen.
\todo{explizites Plus?} Aufgrund einer Untersuchung der Korrelation zwischen \ac{SUS} und
\ac{NPS} von \textcite{sauroDoesBetter2010} stellt \textcite{barnumUsabilityTesting2021} fest, dass
eine Daumenregel zum errechnen des \ac{NPS}-Wertes ist, den \ac{SUS}-Wert durch 10 zu teilen.
\todo{Vielleicht noch Kritik am NPS finden?}


\clearpage
\section{Verwandte Arbeiten}

In diesem Abschnitt wird ein Blick auf für die Projekte geworfen, die für die vorliegende Arbeit von Relevanz sind. Zum einen wurde \textit{Home Assistant} ausgewählt, da die darin enthaltenen visuellen Editoren eine Inspiration für die blockbasierte Herangehensweise darstellen, die im entwickelten Prototyp verwendet wurde. Des Weiteren wird die blockbasierte Programmierumgebung \textit{Scratch} untersucht, \textit{Snap!}, welches starke Ähnlichkeiten zu \textit{Scratch} aufzeigt, sich jedoch mit einem erweiterten Funktionsumfang an ein erfahreneres Publikum richtet. Außerdem wird \textit{DataSnap}, eine Erweiterung für \textit{Snap!}, die es zu einem Werkzeug für Fachexpert:innen machen soll.

Weitere Projekte, die ähnliche Themen behandeln beinhalten:
\begin{itemize}
  \item \textit{\citetitle{pazosr.InterfaceComposing2015}}: Entwicklung einer Abfrageoberfläche für Datenbanken, die von unerfahrenen Nutzer:innen benutzt werden kann, die kein Wissen über den Inhalt und das Schema der Datenbank besitzen \parencite{pazosr.InterfaceComposing2015}.
  \item \textit{DBSnap}: Eine Anwendung zum Erstellen von Datenbankabfragen, die es Nutzer:innen erlaubt, Blöcke nach Regeln der relationalen Algebra miteinander zu verknüpfen und somit Abfragebäume und Datenabfragen zu erlernen \parencite{silvaDBSnapLearning2015}. Diese Anwendung wurde durch \textit{DBSnap++} \parencite{silvaDBSnapCreating2018} und \textit{DBSnap 2} \parencite{silvaDBSnapNew2022} weiterentwickelt.
  \item \textit{SQheLper}: Diese Webanwendung ermöglicht es, durch das Zusammensetzen von Blöcken \acs{SQL}-Abfragen zu generieren, und so die Anzahl der Syntax-Fehler zu reduzieren \parencite{jacobsSQheLperBlockbased2021}.
\end{itemize}

\subsection{Home Assistant}
\subsection{DBSnap}
\parencite{silvaDBSnapLearning2015}
\url{https://www.semanticscholar.org/paper/DBSnap\%3A-Learning-Database-Queries-by-Snapping-Silva-Chon/aabf7791575f8cc62432ca5e6289862eec0f6845}

\url{https://www.semanticscholar.org/paper/DBSnap-2\%3A-New-Features-to-Construct-Database-by-Silva-Loza/61c95757d7d4151f51ea848623c6839816756385}

\url{https://www.semanticscholar.org/paper/DBSnap\%2B\%2B\%3A-creating-data-driven-programs-by-snapping-Silva-Nieuwenhuyse/fdd5ae1bc33337f9f0f7818509008705ffa2dc23}

\subsection{DataSnap}
\parencite{hellmannDataSnapEnabling2015}
\url{https://www.semanticscholar.org/paper/DataSnap\%3A-Enabling-Domain-Experts-and-Introductory-Hellmann/999eda4ba4b9bb1c2320eaf9a0053e37e2bf0c51}

\subsection{SQheLper}
\parencite{jacobsSQheLperBlockbased2021}
\url{https://www.semanticscholar.org/paper/SQheLper\%3A-A-block-based-syntax-support-for-SQL-Jacobs-Jaschke/b4f6023a4ee72d0103ee521c4f9b35768d38810b}
\subsection{Scratch}


\chapter{Studie}
\section{Prototyp}
\subsection{Zielformulierung}
\subsubsection{Genaue Formulierung der Problemstellung, was ist das Ziel?}
\subsection{Lösungsvorstellung}
Eingehen auf Problematik: functions typen / queryables typen -> Informationsverlust?
\subsubsection{Erklärung des Resultats}

\clearpage
\section{Studienaufbau}
\subsection{Problemformulierung und Ziele}
\subsection{Auswahl von Teilnehmer*innen}
\subsubsection{Auf \ref{section:study-size} eingehen, wieso dürfte Größe reichen (wenig persönliche Entscheidungen, konkrete Aufgaben, etc.)}
\subsection{Szenarios}
\subsection{Evaluation}
\subsection{Studienablauf}

\clearpage
\section{Erkenntnisse}
\label{sec:results}

Im Folgenden wird die Usability-Studie ausgewertet. Darunter fallen sowohl die durch \acs{CTA} gewonnenen Erkenntnisse, als auch die Ergebnisse von Befragungen qualitativer und quantitativer Art.

Das \ac{CTA} wurde in den Testsitzungen auf zufriedenstellende Art und Weise durchgeführt. Einige Teilnehmer:innen stachen heraus, indem sie ihre Gedanken besonders oft teilten und aktiv Lösungsvorschläge präsentierten. Anderen Testpersonen fiel es schwerer, kontinuierlich ihre Gedanken zu teilen und mussten häufiger ermuntert werden. In diesen Fällen wurde sich während der Testsitzung auf minimale Fragen beschränkt, die den Redefluss anregen sollten. Falls auf Schlüsselelemente während der Benutzung nicht eingegangen wurde, wurden im Nachhinein Fragen dazu gestellt. Dies war jedoch nur in Einzelfällen notwendig und nie für komplette Testsitzungen der Fall. Die Handlungsabsicht war meist klar und Usability-Probleme konnten häufig sitzungsübergreifend beobachtet werden.

In Abschnitt \ref{sec:impressions} erfolgt zunächst eine Zusammenfassung der ersten Eindrücke der Teilnehmer:innen, die in den Vorgesprächen gewonnen wurden. Im Anschluss werden in Abschnitt \ref{sec:qualitative} die Erkenntnisse präsentiert, die im Laufe der Szenarios gewonnen wurden. Abschnitt \ref{sec:quantitative} widmet sich schließlich der Darstellung der quantitativen Resultate.


\clearpage
\section{Schlussfolgerungen und Zukunftsaussicht}

\chapter{Fazit}

%
% Anhang
%

\appendix
\part*{Anhang}
\markboth{}{}
\addcontentsline{toc}{part}{Anhang}

\ifdefined\STANDALONE
  \section*{Einladung zur Usability-Studie}
\else
  \chapter{Einladung zur Usability-Studie}
\fi

\begin{tabular}{ l l }
  \textbf{Zeitaufwand} & 45 Minuten                                                                   \\
  \textbf{Ort}         & Online-Konferenzraum von Simplex4Data                                        \\
  \textbf{Termin}      & flexibel nach Absprache, jeder Wochentag möglich                             \\
  \textbf{Kontakt}     & \href{mailto:joshua.jeschek@simplex4data.de}{joshua.jeschek@simplex4data.de}
\end{tabular}

\vspace{2\baselineskip}

\noindent
Im Rahmen meiner Bachelorarbeit arbeite ich an der Entwicklung einer Low-Code Anwendung, die das
System Simplex4Data erweitern soll. Diese Anwendung zielt darauf ab, die Konvertierung und
Bearbeitung von Datensätzen durch einen blockbasierten Ansatz zu vereinfachen. Dabei soll ein
Tool erschafft werden, welches effiziente und benutzerfreundliche Datenbearbeitung für Fachleute
ermöglicht - unabhängig davon, ob sie bereits Erfahrung mit dem Simplex-Ansatz und Datenbanken haben
oder nicht.

Für die Weiterentwicklung und Optimierung dieses Projekts bin ich auf der Suche nach Fachleuten,
die im Arbeitsalltag mit (Geo)-Datensätzen umgehen. Ein tiefgehendes Verständnis von Datenbanken ist
für die Teilnahme an diesem Projekt nicht erforderlich. Ihr Fachwissen und Ihre praktischen
Erfahrungen sind von unschätzbarem Wert, um die Anwendung so zu gestalten, dass sie den realen
Anforderungen und Herausforderungen in der Praxis gerecht wird.

Um dies zu erreichen, plane ich, eine qualitative Studie durchzuführen, bei der Sie die Möglichkeit
haben, die Anwendung in einer Videokonferenz und über Bildschirmfreigabe auszutesten. Dabei
werden praxisnahe Problemstellungen bearbeitet, um ein tiefgehendes Verständnis für die
Funktionsweise und den Nutzen der Anwendung zu entwickeln. Dies bietet auch die Gelegenheit,
eventuelle Fehler zu identifizieren und durch Ihr Feedback und Ihre Verbesserungsvorschläge
direkt zur Weiterentwicklung der Anwendung beizutragen.

Mit Ihrer Teilnahme können Sie die Entwicklung der Anwendung beeinflussen, und somit dafür sorgen
dass es einfacher zu benutzen ist und den Anforderungen der Realität gerechnet wird.

Für weitere Informationen und bei Interesse an einer Teilnahme kontaktieren Sie mich bitte unter
\texttt{\href{mailto:joshua.jeschek@simplex4data.de}{joshua.jeschek@simplex4data.de}}. Ich freue mich auf
ihre Teilnahme und Feedback, um gemeinsam die Anwendung zu verbessern und auf Ihre Bedürfnisse
abzustimmen.

\begin{flushright}
  Joshua Jeschek

  Simplex4Data GmbH

  März 2024
\end{flushright}

\ifdefined\STANDALONE\else
  \chapter{Handout Usability Studie}
  \label{app:handout}
\fi

\subsection*{Szenario 1: Import von Bundesländern}
\href[pdfnewwindow=true]{https://db01.simplex4data.de:444/develop/joshua/cql/?functions=/develop/joshua/simplexservice/functions&queryables=/develop/joshua/simplexservice/scenarios/1/collections/14-1-103/queryables&output_class=/develop/joshua/simplexservice/scenarios/1/collections/14-1-104/queryables}{Startpunkt Szenario 1}

\vspace{\baselineskip}\noindent
Gegeben ist eine Importtabelle mit Daten zu den Bundesländern. Sie wollen diese in das Simplex4Data
System importieren. Dazu existiert bereits eine Objektklasse mit den nötigen Attributen.

\vspace{\baselineskip}\noindent
Die Importtabelle sieht in etwa so aus:

\begin{flushleft}
  \begin{tabular}{||c | c | c | c | c | c | c | c | c ||}
    \hline
    bezeichnung & name      & ars    & nuts   & nbd    & ibz    & bemerkung & wirksamkeit & geom       \\ [0.5ex]
    \hline\hline
    'Freistaat' & 'Sachsen' & '14'   & 'DED'  & 'ja'   & 20     & '--'      & 2014-02-01  & <geometry> \\
    \hline
    \vdots      & \vdots    & \vdots & \vdots & \vdots & \vdots & \vdots    & \vdots      & \vdots     \\
    \hline
  \end{tabular}
\end{flushleft}

\vspace{\baselineskip}\noindent
Die Objektklasse enthält folgende Attribute:
\begin{flushleft}
  \begin{tabular}{ || l || }
    \hline
    Bundesland            \\
    \hline
    key - ARS             \\
    typ - Bezeichnung     \\
    title - Name          \\
    cmt - Bemerkung       \\
    beg - Wirksamkeit     \\
    nuts - NUTS           \\
    nbd - NBD             \\
    ibz - IBZ             \\
    fl - Flächengeometrie \\
    \hline
  \end{tabular}
\end{flushleft}

\vspace{\baselineskip}\noindent
Weisen Sie die Daten aus der Importtabelle den korrespondierenden Attributen aus der Objektklasse
zu, und überprüfen Sie das Resultat.

\clearpage

\subsection*{Szenario 2: Import von Gemeinden}
\href[pdfnewwindow=true]{https://db01.simplex4data.de:444/develop/joshua/cql/?functions=/develop/joshua/simplexservice/functions&queryables=/develop/joshua/simplexservice/scenarios/1/collections/14-1-100/queryables&output_class=/develop/joshua/simplexservice/scenarios/1/collections/14-1-105/queryables&filter=on}{Startpunkt Szenario 2}

\vspace{\baselineskip}\noindent
Analog zu den Bundesländern sollen Gemeinden importiert werden. Die Daten in der Quelltabelle liegen
nicht einfach zur Zuordnung bereit, sondern müssen teilweise umgeformt werden.

\noindent Die Quelltabelle enthält auch Einträge, die sich nicht auf Gemeinden beziehen. Diese
können daran erkannt werden, dass \texttt{gemeinde\_id} nicht angegeben ist, und sollen nicht
importiert werden.

\vspace{\baselineskip}\noindent
Hier ein Auszug aus der Quelltabelle:
\begin{flushleft}
  \begin{tabular}{|| c | c | c | c | c | c | c | c ||}
    \hline
    gemeindename  & breitengrad & \dots  & bevoelkerung & flaeche & gemeinde\_id & \dots  & land\_id \\ [0.5ex]
    \hline\hline
    'Saarbrücken' & 49.236608   & \dots  & 179634       & 167.52  & '10'         & \dots  & '10'     \\
    \hline
    \vdots        & \vdots      & \vdots & \vdots       & \vdots  & \vdots       & \vdots & \vdots   \\
    \hline
  \end{tabular}
\end{flushleft}

\vspace{\baselineskip}\noindent
Die Objektklasse enthält folgende Attribute:
\begin{flushleft}
  \begin{tabular}{ || l || }
    \hline
    Gemeinde                                        \\
    \hline
    title - Name                                    \\
    key - ARS                                       \\
    bevoelkerung\_je\_km2 - Bevölkerung je km2      \\
    bevoelkerung\_weiblich - Bevölkerung weiblich   \\
    bevoelkerung\_männlich - Bevölkerung männlich   \\
    bevoelkerung\_insgesamt - Bevölkerung insgesamt \\
    flaeche\_km2 - Fläche in km2                    \\
    pt - Punktgeometrie                             \\
    \hline
  \end{tabular}
\end{flushleft}

\vspace{\baselineskip}\noindent
Der ARS (Amtlicher Regionalschlüssel) setzt sich wie folgt zusammen:
\[
  \underbrace{\scalebox{3}{10}}_{\texttt{land\_id}}\overbrace{\scalebox{3}{0}}^{\texttt{bezirk\_id}}\underbrace{\scalebox{3}{41}}_{\texttt{kreis\_id}}\overbrace{\scalebox{3}{0100}}^{\texttt{verband\_id}}\underbrace{\scalebox{3}{10}}_{\texttt{gemeinde\_id}}
\]

\clearpage

\subsection*{Szenario 3: Auflistung der Adressen in einem Leipziger Ortsteil}
\href[pdfnewwindow=true]{https://db01.simplex4data.de:444/develop/joshua/cql/?functions=/develop/joshua/simplexservice/functions&queryables=/develop/joshua/simplexservice/scenarios/1/collections/1-1-119/queryables?joins=%3E1-1-119:1-3-121-obj-0:nn,%3C1-3-121-obj-0:1-1-108-obj-strasse,%3E1-1-119:1-3-120-obj-1:nn,%3C1-3-120-obj-1:1-1-103-obj-ortsteil&filter=on}{Startpunkt Szenario 3}

\vspace{\baselineskip}\noindent
In diesem Szenario sind die Objektklassen bereits importiert. Dabei handelt es sich um:
\begin{itemize}
  \item Adressen (Hausnummer, PLZ, Geometrie)
  \item Straßen (Straßenname)
  \item Ortsteile (Ortsteil-Name)
\end{itemize}

\vspace{\baselineskip}\noindent
Die Klassen sind miteinander verknüpft und sollen nun zu einer Übersicht aller Adressen \textbf{im Ortsteil
  "Lindenau"} zusammengestellt werden.

\vspace{\baselineskip}\noindent
Die Liste sollte am Ende folgende Informationen enthalten:
\begin{itemize}
  \item Straßenname
  \item Hausnummer
  \item Adresszusatz
  \item Postleitzahl
  \item Ortsteil-Name
  \item Punktgeometrie(n)
\end{itemize}

\ifdefined\STANDALONE
  \section*{Skript für Moderation der Usability-Studie}
\else
  \chapter{Skript für die Moderation der Usability-Studie}
  \label{app:moderation}
\fi

\subsection*{Begrüßung}
\subsection*{Motivation und Zielstellung erklären}
\begin{itemize}
  \item Anwendung greift an zwei Stellen des Simplex Systems an
        \begin{itemize}
          \item Import \textrightarrow{} Rohdaten in die interne Struktur (Importtabellen \textrightarrow{} Objektklassen)
          \item Erstellen von SimplexSzenarios, neue Sicht auf bereits importierte Daten (Dabei können mehrere Objektklassen miteinander kombiniert werden, insofern verknüpft) (Objektklassen \textrightarrow{} Szenarios)
        \end{itemize}
  \item Ziel ist es, den Prozess ohne Tippfehler und häufiges Nachschlagen zu bewältigen
  \item Auch für Personen die nicht bewandert in SQL etc. sind
  \item Auch mit komplexeren Operationen an den Daten, als 1:1 Übernahme von Quelle zu Ziel
\end{itemize}

\subsection*{Erklärung des Vorgehens}
\begin{itemize}
  \item schon mal an Usability-Studie teilgenommen?
  \item Szenarios erklären (erkläre Situation, Zielstellung, Sie versuchen diese zu erreichen)
  \item Handout weniger als Aufgabenblatt, sondern mehr als Gedankenstütze für uns beide, dort sind die wichtigsten Fakten nochmal vermerkt. Aber an sich erzähle ich alles. (Dort befinden sich auch die Links zu den Startpunkten der einzelnen Szenarios)
  \item Screen-Sharing
  \item Thinking Aloud / lautes Nachdenken erklären
        \begin{itemize}
          \item Während der Benutzung der Anwendung so viele Gedanken wie möglich teilen
          \item weniger Spekulation meinerseits
          \item Sehr hilfreich gleich zu hören, was Sie frustriert oder verwirrt, aber auch was sie gut finden oder was sie erwarten bevor sie etwas tun / erwartet hätten danach
          \item fühlt sich vielleicht nicht sofort natürlich an, aber so können wir auch mehr Informationen sammeln als wenn wir im Nachinein darüber reden.
          \item Beispiele: "Ich mag das, weil...", "Das hab ich nicht als Reaktion auf meinen Klick erwartet, sondern..."
          \item Kann auch sein dass ich an bestimmten Stellen nachhake
        \end{itemize}
  \item Szenarios sind dafür gedacht die Anwendung zu testen nicht Sie.
\end{itemize}

Fragen?

\ifdefined\STANDALONE
  \clearpage
\fi
\subsection*{Vor Beginn des ersten Szenarios}
\begin{itemize}
  \item Was ist ihr Eindruck?
  \item Können Sie beschreiben was Sie über die Oberfläche in diesem Zustand denken? So können wir auch schon das Thinking Aloud austesten.
  \item Benennung von Bereichen in der Oberfläche
\end{itemize}

\subsection*{Szenarios}
\begin{enumerate}
  \item Import Bundesländer
  \item Import Gemeinden
  \item[] DATENBANK UMSTELLEN
  \item SimplexSzenario Adressen Leipzig
\end{enumerate}

\subsection*{Nach jedem Szenario}
\begin{itemize}
  \item SEQ: Auf einer Skala von 1 bis 7, wie einfach oder schwer fanden Sie es, die Aufgabe zu absolvieren? (1 sehr schwer, 7 sehr einfach)
  \item darauf basierend Möglichkeit auf Probleme einzugehen
  \item Feedback zu Thinking Aloud
  \item Liegt noch was auf dem Herzen?
\end{itemize}

\subsection*{Nach den Szenarios}
\begin{itemize}
  \item Haben Sie sich von der Anwendung in der Erfüllung der Aufgaben unterstützt gefühlt?
  \item Was fanden Sie am Schlechtesten?
  \item Was fanden Sie am Besten?
  \item Bewerten Sie die Einfachheit der Benutzung auf einer Skala von 1 bis 10. (1 ist sehr schwer, 10 ist sehr einfach.)
  \item Falls das alte System bekannt ist, wie würden Sie dieses auf der gleichen Skala einschätzen?
  \item Fühlen Sie sich produktiver mit dieser Oberfläche, oder finden Sie diese zufriedenstellender?
  \item Halten Sie diese Oberfläche, sobald sie in Simplex4TwIS integriert ist, als eine sinnvolle und nützliche Erweiterung?
  \item Haben Sie noch weitere Vorschläge für die Erweiterung und Verbesserung der Anwendung, wie können wir sie nützlicher gestalten?
  \item Wie würden Sie ähnliche Aufgaben wie die hier betrachteten Aufgaben sonst angehen?
\end{itemize}

\subsection*{Vielen Dank für Ihre Zeit und nützliche Hinweise}



\nocite{*}
\chapter*{Literaturverzeichnis}
\begingroup
\let\clearpage\relax
\printbibliography[filter=articles, title=Artikel, heading=subbibliography]
\printbibliography[filter=books, title=Bücher, heading=subbibliography]
\printbibliography[filter=other, title=sonstige, heading=subbibliography]
\endgroup

% TODO: Index
% TODO: Glossar

\cleardoublepage
\vspace*{0pt plus 1fill}
\begin{tucsimplesection}{Danksagung}
  Mein großer Dank gilt allen Teilnehmer:innen der Usability-Studie. Sie haben durchschnittlich 90 Minuten ihrer Zeit aufgeopfert und mich mit vielen interessanten Einblicken in ihr Nutzungsverhalten belohnt.

  Ich möchte mich bei allen Kolleg:innen von Simplex4Data bedanken, die mich bei der Durchführung dieses Projekts unterstützt haben und die Arbeit korrekturgelesen haben. Insbesondere möchte ich mich bei Werner bedanken, ohne den das Projekt bereits vor der Konzeption an einem fehlenden Backend gescheitert wäre.

  Ein weiterer Werner, dem ich danken will ist Werner vom \textit{Tex - LaTeX - Stack Exchange}\footnote{\url{https://tex.stackexchange.com/users/5764/werner}}, da ich überdurschnittlich oft auf seine Antworten gestoßen bin, um Probleme mit meinem \LaTeX{}-Dokument zu lösen.

  Zu guter Letzt möchte ich mich bei Sarah bedanken, die mir immer wieder aus Schreibblockaden geholfen hat und mich mit Bildern von ihrer Katze Shanti angespornt hat.
\end{tucsimplesection}
\vspace*{0pt plus 2.5fill}

\cleardoublepage
\begin{tucerklaerung}
  Ich erkläre gegenüber der Technischen Universität Chemnitz, dass ich die
  vorliegende XXX\todo{Art der Arbeit}\ selbstständig und ohne Benutzung anderer
  als der angegebenen Quellen und Hilfsmittel angefertigt habe.

  Die vorliegende Arbeit ist frei von Plagiaten. Alle Ausführungen, die wörtlich
  oder inhaltlich aus anderen Schriften entnommen sind, habe ich als solche
  kenntlich gemacht.

  Diese Arbeit wurde in gleicher oder ähnlicher Form noch bei keinem anderen
  Prüfer als Prüfungsleistung eingereicht und ist auch noch nicht veröffentlicht.

  \tucsignature{Autor}\todo{Autor}
  \tucsignature{Autor 2}
\end{tucerklaerung}

\end{document}
