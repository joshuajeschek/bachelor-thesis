\documentclass[a4paper, 12pt, oneside, BCOR=1cm,toc=chapterentrywithdots]{scrbook}

\usepackage{graphicx}                   % use for pdfLatex
\usepackage{makeidx}                    % \printindex
\usepackage[colorlinks=false]{hyperref}
% \usepackage{tocbibind}
\usepackage{blindtext}
\usepackage{subfigure}
%\usepackage[acronym,toc,nomain]{glossaries}
\usepackage{acro}
\usepackage[shorthands=off,ngerman]{babel}
\usepackage[utf8]{inputenc}
\usepackage[T1]{fontenc}
\usepackage{csquotes}
\usepackage[style=apa,backend=biber]{biblatex}
\usepackage{xcolor}
\addbibresource{bibliography.bib}

% https://tex.stackexchange.com/a/302313
\makeatletter
\newcommand{\newstarcommand}[1]{%
  \DeclareRobustCommand#1{%
    \@ifstar{\csname s\string#1\endcsname}{\csname n\string#1\endcsname}%
  }%
  \edef\meta@def@name{\string#1}%
  \meta@def
}
\newcommand\meta@def[3][0]{%
  \expandafter\newcommand\csname n\meta@def@name\endcsname[#1]{#2}%
  \expandafter\newcommand\csname s\meta@def@name\endcsname[#1]{#3}%
}
\makeatother

% todo footnotes
\newcounter{todocounter}
\setcounter{todocounter}{100}
\newcommand{\todo}[1]{\refstepcounter{todocounter}\textcolor{red}{\footnote[\thetodocounter]{\textcolor{gray}{TODO: #1}}}}

\defbibfilter{articles}{
  type=article or
  type=inproceedings or
  type=report
}

\defbibfilter{books}{
  type=book or
  type=inbook or
  type=incollection
}

\defbibfilter{other}{
  type=misc or
  type=thesis or
  type=online
}

\DefineBibliographyStrings{ngerman}{
  nodate = {{}o\adddot \space D\adddot},
  mathesis = {Masterarbeit}
}

\makeatletter
\DeclareCiteCommand{\textcitealias}
{\usebibmacro{prenote}}
{\usebibmacro{citeindex}%
  \printtext[bibhyperref]{\@citealias{\thefield{entrykey}} (\citeyear{\thefield{entrykey}})}}
{\multicitedelim}
{\usebibmacro{postnote}}

\DeclareCiteCommand{\parencitealias}[\mkbibparens]
{\usebibmacro{prenote}}
{\usebibmacro{citeindex}%
  \printtext[bibhyperref]{\@citealias{\thefield{entrykey}}, \citeyear{\thefield{entrykey}}}}
{\multicitedelim}
{\usebibmacro{postnote}}
\makeatother

\defcitealias{ogcCatalogue2016}{\ac{ogc}}
\defcitealias{ogcFiltering}{\ac{ogc}}
\defcitealias{ISO924111}{ISO/TC 159 Ergonomics SC 4}
\defcitealias{unisd2021}{UNIS-D}


\begin{acronym}
  \acro{TA}{Think-Aloud-Protokoll}
  \acro{CTA}{Concurrent Think-Aloud-Protokoll}
  \acro{RTA}{Retrospective Think-Aloud-Protokoll}
\end{acronym}


\hypersetup{bookmarksnumbered=true,bookmarksopen=true,bookmarksopenlevel=1,pdfborder=0 0 0}

% remove this later
%\setcounter{tocdepth}{4}

% könnte zu Problemen führen
\MakeOuterQuote{"}

\begin{document}
\begin{titlepage}

  \begin{center}
    \raisebox{-1ex}{\includegraphics[scale=1.5]{assets/tuc.pdf}}\\
  \end{center}
  \vspace{0.5cm}

  \begin{center}

    \LARGE{\textbf{Title of the bachelor thesis}}\\
    \vspace{1cm}


    \Large{\textbf{Bachelor Thesis}}\\
    \vspace{1cm}
    Submitted in Fulfilment of the\\
    Requirements for the Academic Degree\\
    B.Sc.\\
    \vspace{0.5cm}
    Dept. of Computer Science\\
    Chair of Computer Engineering
  \end{center}
  \vspace{3cm}
  Eingereicht von: Joshua Jeschek\\
  Matrikelnummer: 652272\\
  Datum: 24.12.2023\\
  \vspace{0.3cm}\\
  Dr. Thomas Wilhelm-Stein \\
  Dr. Heino Rudolf

\end{titlepage}

\addchap*{Abstract}
\vspace*{0pt plus 1fill}
\begin{tucsimplesection}{Abstract}
  \todo[inline]{\blindtext}
\end{tucsimplesection}
\vspace*{0pt plus 2.5fill}


\tableofcontents

\begingroup
\cleardoublepage
\addcontentsline{toc}{chapter}{\listfigurename}
\listoffigures

\let\clearpage\relax

\cleardoublepage
\addcontentsline{toc}{chapter}{\listtablename}
\listoftables
\endgroup

\twocolumn
%\addchap{Abkürzungsverzeichnis}
%\printglossary[type=\acronymtype,style=long]
\printacronyms
\onecolumn

\chapter{Einleitung}

\section{Hintergrund und Motivation}
\section{Ziele und Umfang}
\section{Aufbau der Arbeit}

\chapter{Theoretische Grundlagen}

\section{envVisio}
\subsection{Das Realitätsmodell}
\subsubsection{Allgemeine Erklärung von envVisio}
\subsubsection{Vielleicht Aufteilung in mehrere Abschnitte?}
\subsubsection{was ist alles wichtig?}
\subsection{Laden und Konvertieren}
\subsubsection{auf Loader und insbesondere Converter eingehen}
\subsubsection{da in Converter Formular eingesetzt werden soll}
\subsubsection{wichtig, Prozesse zu verstehen}

\section{\acl{cql}}

Bei der \ac{cql} handelt es sich um eine Abfragesprache, welche durch das
\shorthandtextcite{opengeospatialconsortiumOGCAPI2020} eingeführt wurde.


\clearpage
\section{Human-Centered Design}
\subsubsection{Begriffe und Konzepte aus Design of Everyday Things}
\subsubsection{Abschnittstitel nicht final, vielleicht is HCD auch ein Unterpunkt auf gleicher Höhe wie Aff., Sign. und Map.}
\subsection{Affordances, Signifiers und Mappings}
\subsubsection{Kann gut bei 2.3.2 angewendet werden}
\subsection{Activity-Centered Design}

\clearpage
\section{Formulare}
\cite{wroblewskiWebForm2008}
\subsection{Funktionsweise}
\subsubsection{Technische Fkt.-weise von F.}
\subsection{Designanfordungen}
\cite{normanDesignEveryday2013} (auf Affordances, Signifiers und Mappings eingehen)
\subsubsection{Rückbezug auf 2.2}
\subsubsection{Anordnung}
\subsubsection{Kontrollelemente}
\subsubsection{Bezeichnungen und Platzhalter}
\subsubsection{Validierung}

\clearpage
\section{Usability Testing}

Um auf \textit{Usability Testing} einzugehen, lohnt es sich, einen genaueren Blick auf
\textit{Usability} zu werfen. \citefield{ISO924111}{shorttitle} definiert \textit{Usability} wie
folgt:
\begin{quote}
  extent to which a system, product or service can be used by specified users to achieve specified
  goals with effectiveness, efficiency and satisfaction in a specified context of use
  \hspace*{\fill{}}\shorthandtextcite*{ISO924111}
\end{quote}
Diese Definition ist kurz, umfasst jedoch trotzdem drei Hauptelemente, die \textit{Usability}
ausmachen. Es sollte möglich sein, Aufgaben mit ausreichender Geschwindigkeit auszuführen
(\textit{efficiency}) und das erwünschte Ziel zu erreichen (\textit{effectiveness}). Der dritte
wichtige Faktor ist die Anwenderzufriedenheit (\textit{satisfaction}), welche mehr auf die
subjektive Erfahrung von Nutzer:innen eingeht. So kann eine Aufgabe schnell und korrekt ausgeführt
werden, aber dennoch nicht zu Zufriedenheit führen, da beispielsweise wichtige Informationen nicht
angezeigt wurden, die über den Erfolg informieren. Der gleiche Effekt kann auch andersherum
stattfinden - sollte die persönliche Einschätzung von Zufriedenheit sehr gut sein, können womögliche
Probleme der Effizienz und Effektivität vernachlässigt werden, sodass das Produkt trotzdem als gut
eingeschätzt wird. \parencite{barnumUsabilityTesting2021}

Des Weiteren betont \citeauthor{barnumUsabilityTesting2021} \cite{barnumUsabilityTesting2021}, dass
es sich um spezifische Nutzer:innen, Ziele und Nutzungskontexte handelt. Das bedeutet, dass
\textit{Usability} nur für eine spezifische Gruppe an Menschen betrachtet werden kann, welche das
Produkt benutzen werden und deren Ziele denen des Produkts entsprechen müssen. Außerdem kann die
\textit{Usability} auch nur für den angedachten Nutzungskontext - die Umgebung in der ein Produkt
genutzt wird - betrachtet werden. \parencite{barnumUsabilityTesting2021}

\textcite{quesenberyDimensionsUsability2003} und \textcite{nielsenUsabilityEngineering1994}
spezifizieren folgende 5 Attribute der \textit{Usability}:
\begin{itemize}
  \item \textbf{Effektivität}: wie umfassend und akkurat Aufgaben abgeschlossen werden,
  \item \textbf{Effizienz}: wie schnell dies erfolgt,
  \item \textbf{Ansprechende Gestaltung}: wie gut Interaktionen gesteuert werden und für
    Zufriedenheit gesorgt wird (von \textcite{nielsenUsabilityEngineering1994} vernachlässigt)
  \item \textbf{Fehlertoleranz}: wie das System Fehler verhindert und sich von Fehler erholt
  \item \textbf{Erlernbarkeit}: wie intuitiv das System ist, und wie Lernen im Laufe der Nutzung
    ermöglicht wird (von \textcite{nielsenUsabilityEngineering1994} in zwei Attribute geteilt)
\end{itemize}
Laut \textcite{barnumUsabilityTesting2021} hat der Faktor der Zufriedenheit in den letzten Jahren an
Bedeutung zugenommen. Das wird auch davon unterstützt, dass
\textcite{nielsenUsabilityEngineering1994} weniger Wert auf eine ansprechende Gestaltung legte als
\textcite{quesenberyDimensionsUsability2003}.

\newpage

\textit{Usability Testing} ist ein Sammelbegriff für Methoden bei denen Benutzer:innen mit einem
System interagieren, um ein Ziel zu erreichen oder Szenario umzusetzen. Dabei sind die Umstände der
Interaktionen kontrolliert und Verhaltensdaten werden gesammelt
\parencite{wichanskyUsabilityTesting2000}. Ziel ist es, die \textit{Usability} des Systems zu
überprüfen und eventuelle Probleme zu identifizieren.

Dabei ist es laut \textcite{barnumUsabilityTesting2021} wichtig, dass die Nutzer:innen Aufgaben
erfüllen, die echt sind und eine Bedeutung für sie beinhalten. So werden Probleme, die bei der
Nutzung in der realen Welt auftreten effizienter aufgedeckt.

Wichtige Aspekte des \textit{Usability Testing} werden im weiteren Verlaufe des dieses Kapitels
beleuchtet.

\subsection{Testmethoden}
\subsubsection{Formative und Summative Studien}
\label{sec:formative-summative}
Usability Tests können in zwei Typen unterteilt werden: \textbf{Formative} und \textbf{Summative
  Studien}.

\vspace{\baselineskip}

\textbf{Formative Studien} werden während der Entwicklung durchgeführt, um Probleme zu diagnostizieren und zu lösen. Der Umfang der einzelnen Studien ist relativ klein, im Entwicklungszyklus können sie jedoch meist wiederholt durchgeführt werden \parencite{barnumUsabilityTesting2021}. Die Ergebnisse von formativen Studien können direkt in die Entwicklung einfließen, wobei erneute Studien überprüfen können, ob die Probleme aus vergangen Durchläufen behoben wurden. Des Weiteren können formative Studien aufzeigen, was Nutzer:innen wichtig ist, und somit die Entwicklung in eine Richtung lenken, die die Usability des Endprodukts drastisch verbessert \parencite{barnumUsabilityTesting2021}. Erkenntnisse sind in zeitigen Stadien der Produktentwicklung besonders wichtig, da es im Verlauf des Entwicklungszyklus immer schwieriger wird, grundlegend falsche Entscheidungen zu berichtigen \parencite{barnumUsabilityTesting2021}.

Für die Zwecke dieser Arbeit ist eine formative Studie angemessen (Vgl. \ref{sec:study-methods}). Daher werden Schlüsselelemente von formativen Studien, wie zum Beispiel Benutzerprofile, Szenarios und das \acl{TA}, im weiteren Verlauf detailliert vorgestellt.

\vspace{\baselineskip}

\textbf{Summative Studien} werden durchgeführt wenn ein Produkt fertiggestellt ist oder kurz vor der Fertigstellung steht \parencite{barnumUsabilityTesting2021}. Dabei soll unter Usability und Funktionalität auch die Zufriedenheit der Nutzer:innen bewertet werden. Diese Art von Studie findet in einem größeren Rahmen statt und erfordert in der Regel viele Teilnehmer, um statistische Relevanz zu erreichen \parencite{barnumUsabilityTesting2021}. Ähnlich zu formativen Studien werden Teilnehmer:innen Szenarios und Aufgaben bereitgestellt, darüber hinaus wird jedoch nicht weiter mit ihnen interagiert. Somit können Metriken wie die Zeit für die Bearbeitung von Aufgaben und Abschlussquoten erhoben werden.

Die Ergebnisse von summativen Studien sind meistens numerisch, während Erkenntnisse von formativen Studien konkrete Problembeschreibungen beinhalten. Es ist also einfacher, summativen Studien miteinander zu vergleichen, während formative Studien konkretere Probleme auflisten und frühzeitig und schnell zu Verbesserungen führen können \parencite{barnumUsabilityTesting2021}.

\subsubsection{\acl{TA}}
\label{sec:think-aloud}

Das \acf{TA} ist eine weit verbreite Methode, um die Usability von Applikationen einzuschätzen. Dabei sollen Nutzer:innen ihre Erfahrungen, Gedanken, Handlungen und Gefühle beim Interagieren mit der Anwendung ausdrücken \parencite{barnumUsabilityTesting2021}. Erkenntnisse, die auf diese Art gesammelt wurden, bieten Einblick in Denkprozesse von Nutzer:innen und können somit Entwicklungsentscheidungen lenken \parencite{alhadretiRethinkingThinking2018}.

Im Folgenden werden anhand von drei Studien verschiedene Funktionsweisen von \ac{TA} vorgestellt, und Auswirkungen auf den Usability Testing Prozess beleuchtet.

Typischerweise werden zwei Typen von \ac{TA} unterschieden: \ac{CTA} und \ac{RTA}. Bei \ac{CTA} teilen die Teilnehmer:innen ihre Gedanken sofort beim Ausführen der Aufgaben, während dies bei \ac{RTA} erst im Anschluss daran passiert \parencite{alhadretiRethinkingThinking2018}. Einerseits kann \ac{CTA} es einfacher machen Problembereiche zu identifizieren, da die zusätzlichen Informationen in Echtzeit bereitgestellt werden, andererseits existieren folgende Bedenken: Das zum Handeln parallel Sprechen könnte ungewohnt und irritierend für Teilnehmer:innen sein und dadurch zu einer unangenehmen Situation führen \parencite{alhadretiRethinkingThinking2018}. Des Weiteren könnte \ac{CTA} Denkprozesse aktiv beeinflussen und somit die Art und Weise in der Aufgaben abgearbeitet werden verändern \parencite{alhadretiRethinkingThinking2018}. Laut \citeauthor{alhadretiRethinkingThinking2018} sollte bei der Betrachtung der Aussagekraft der erhobenen Daten auch beachtet werden, dass durch die Verwendung von \ac{CTA} oft zusätzlich Zeit in Anspruch genommen wird. Im Gegensatz dazu beeinflusst \ac{RTA} die Denkprozesse von Teilnehmer:innen nicht, und leidet somit nicht unter den möglichen Problemen des \ac{CTA}. Problematisch an dieser Methode ist jedoch, dass sich Teilnehmer:innen korrekt an die Ausführung der Aufgabe erinnern müssen \parencite{alhadretiRethinkingThinking2018}. Außerdem besteht die Gefahr, dass spätere Handlungen und Teilschritte vorherige Gedanken beeinflussen oder relativieren \parencite{alhadretiRethinkingThinking2018}. So könnten wichtige Erkenntnise verloren gehen. Vereinzelt wird auch auf einen hybriden Ansatz gesetzt, bei dem die Erkenntnise von \ac{CTA} durch eine Nachbesprechung
angereichert werden \parencite{alhadretiRethinkingThinking2018}.

\textcite{alhadretiRethinkingThinking2018} verglichen in einer Studie \ac{CTA}, \ac{RTA} und einen hybriden Ansatz miteinander. Dabei wurden folgende Metriken betrachtet: Allgemeine Leistung beim Ausführen von Aufgaben, Erfahrungen der Teilnehmer:innen, Menge und Qualität der gefundenen Usability-Probleme, sowie die Kosten zur Nutzung der Methoden. Am besten schnitt das \acl{CTA} ab, da es am meisten Probleme fand und das beste Feedback von Teilnehmer:innen hervorrief \parencite{alhadretiRethinkingThinking2018}. Es konnte auch kein Unterschied in der Leistung der Teilnehmer:innen im Vergleich zu Durchläufen ohne \ac{TA} festgestellt werden. Außerdem ist es vorteilhaft, \ac{CTA} gegenüber \ac{RTA} und einem hybriden Ansatz zu wählen, da diese doppelt so viel Test- und Analysierungszeit benötigten \parencite{alhadretiRethinkingThinking2018}.

Abgesehen von den soeben betrachteten Typen des \acl{TA} existieren auch Unterschiede im Umfang der Interaktion zwischen Testadministrator:innen und Teilnehmer:innen. \textcite{olmsted-hawalaThinkaloudProtocols2010} testeten folgende auftretenden Kategorien, räumten jedoch ein dass die Realität oft nuancierter ist:
\begin{itemize}
  \item Traditionell: Außer "Bitte reden Sie weiter." werden keine weiteren Nachfragen angestellt.
  \item Sprachkommunikation: In Form von "um-hum" oder "un-hum" werden Teilnehmer:innen zum Weiterreden angeregt. Nach ausreichend langen Stillephasen wird mit dem zuletzt geäußerten Wort das \acl{TA} weiter angeregt. \footnote{Eine Interaktion könnte beispielsweise so aussehen:\\ Teilnehmer:in: "Das war komisch..." \\ -- 15 Sekunden Stille -- \\ Administrator:in: "Komisch?"}
  \item Coaching: Es wird aktiv in eingegriffen, indem direkte Fragen gestellt werden und assistiert wird, falls Teilnehmer:innen zu große Schwierigkeiten haben.
\end{itemize}
Die Ergebnisse der \citeyear{olmsted-hawalaThinkaloudProtocols2010} durchgeführten Studie zeigen, dass Teilnehmer:innen, bei denen Coaching angewendet wurde, erfolgreicher als andere waren und zufriedener mit der getesteten Website waren. Die Zeit zur Beendigung der Aufgaben unterschied sich nicht zwischen den verschiedenen Herangehensweisen und, genauso wie bei \citeauthor{alhadretiRethinkingThinking2018}, konnte kein Unterschied zur Kontrollgruppe (ohne \ac{TA}) festgestellt werden. Eine Beeinflussung der Ergebnisse scheint also nur beim Coaching aufzutreten - während die restlichen Variationen des \ac{TA} weder die Genauigkeit noch die Geschwindigkeit abändern. Es wird betont, dass eine genaue Beschreibung des verwendeten \acl{TA} wichtig ist, um einen hohen Grad an Reproduzier- und Vergleichbarkeit zu erreichen \parencite{alhadretiRethinkingThinking2018}.

Eine Studie von \textcite{rheniusEvaluationConcurrent1990} kam zu leicht anderen Ergebnissen. Es wurde festgestellt, dass die Zeit zum Lösen von Aufgaben bei der Verwendung von \ac{TA} steigt, die Genauigkeit jedoch nicht davon beeinflusst wurde. \citeauthor{rheniusEvaluationConcurrent1990} kamen auch zu dem Schluss, dass der Einfluss von \ac{TA} nur in den frühen Phasen der Gewöhnung an die Aufgaben auftritt \parencite{rheniusEvaluationConcurrent1990}.

Somit kann also festgehalten werden, dass die Verwendung von \ac{TA} wenn überhaupt nur eine minimale Auswirkung auf die Zeit zur Fertigstellung von Aufgaben hat. Eine Beeinflussung auf die Testergebnisse konnte in keiner der Studien festgestellt werden, außer bei der Verwendung von Coaching. Das \acl{TA} stellt sich somit als gutes Werkzeug zur Durchführung von Usability Studien heraus. Insbesondere eine Verwendung von \acl{CTA} und eine Beschränkung auf minimale Interaktionen mit den Teilnehmer:innen scheint vorteilhaft.

\subsection{Studiengröße}
\label{section:study-size}

Wie in \ref{section:formative-summative} erwähnt, existieren Usability Tests in unterschiedlichen Umfängen. Formative Studien sind klein und werden meist während der Entwicklung durchgeführt, während summative Studien mehr Teilnehmer*innen umfassen und gegen Ende der Entwicklung stattfinden.

Für diese Arbeit werden formative Studien von Relevanz sein. Dabei sollte die Anzahl der Teilnehmer*innen so klein wie möglich zu halten, während so viele Usability Probleme wie möglich entdeckt werden, denn eine höhere Anzahl an Testdurchläufen steigert die Kosten und den Zeitaufwand.
\parencites{faulknerFiveuserAssumption2003, nielsenWhyYou2000}

\textcite{nielsenMathematicalModel1993} beschreiben den Anteil an gefundenen Usability Problemen bei $n$ Teilnehmer*innen als wie folgt:
\begin{equation}
  \label{equation:finding-usability-problems}
  N(1-(1-\lambda{})^n),
\end{equation}
wobei $N$ die Gesamtanzahl der Probleme und $\lambda{}$ der Anteil der Probleme die bei einem einzigen Testdurchlauf gefunden werden. Mit dem von ihm über mehrere Projekte beobachtete Wert von $\lambda{}=31\%$ kommt \textcite{nielsenWhyYou2000} zum Resultat, dass mit fünf Teilnehmer*innen 85\% aller Probleme festgestellt werden können. \cite{nielsenWhyYou2000} Weitere Testdurchläufe sind weniger lohnenswert, da nicht genug neue Probleme aufgedeckt werden. Um alle Probleme mit einer Studie zu finden, wären 15 Teilnehmer*innen nötig, die gegen Ende jedoch nur noch vereinzelt neue Erkenntnisse liefern. Da Teilnehmer*innen oft die gleichen, oberflächlichen Fehler finden und dadurch abgelenkt werden, ist es von Vorteil nach den ersten fünf Durchläufen die gefundenen 85\% zu beheben, und dann eine weitere Studie durchzuführen. Eine dreifache Wiederholung der Tests mit fünf Durchläufen und einem verbesserten Produkt führt zu umfangreicheren Resultat als ein einziger Test mit 15 Durchläufen.
\parencite{nielsenWhyYou2000}
\todo{vielleicht update von 2012 mit einbeziehen? \cite{nielsenHowMany2012}}

Die als Faustregel adaptierte Zahl Fünf wurde in mehreren Publikationen untersucht. Während fünf Teilnehmer*innen durchschnittlich 85\% der Probleme (einer Studie) entdecken können, kann dieser Prozentsatz laut \textcite{faulknerFiveuserAssumption2003} jedoch bis auf 55\% sinken. Während die durchschnittliche Rate langsamer steigt, verringert sich die Varianz mit mehr Testdurchläufen. Außerdem ist anzumerken, dass die Frage, wie viele Teilnehmer*innen benötigt werden schwer zu  beantworten ist, und von vielen Faktoren abhängt. \footnote{genannt werden: Art und Erfahrung der Teilnehmer*innen, Wichtigkeit des Systems, mögliche Folgen von Usability Problemen}
\parencite{faulknerFiveuserAssumption2003}

\textcite{spoolTestingWeb2001} fanden einen deutlich geringeren Wert von $\lambda{}=10\%$ bei unstrukturiertem Testen von E-Commerce-Webseiten. Somit würden mit 5 Testdurchläufen nur 35\% der Probleme gefunden werden. Als Gründe dafür kann die Komplexität der Webseite und die persönlich getroffenen Entscheidungen der Teilnehmer*innen genannt werden.
\parencite{spoolTestingWeb2001}

Laut \textcite{woolrychWhyWhen2001} ist die Wahrscheinlichkeit, dass ein Problem entdeckt wird, nicht konstant sondern hängt von Schweregrad, Benutzermerkmalen, Produkttyp und Strukturierungsgrad des Testes ab. Sie erweitern deshalb Gleichung \ref{equation:finding-usability-problems} um $\lambda{}_j$ für jedes Usability Problem, welches die Wahrscheinlichkeit darstellt, dass Problem von einer zufällig ausgewählten Teilnehmer*in gefunden wird. Die erwartete Nummer an gefundenen Problemen beträgt dann laut \textcite{woolrychWhyWhen2001}:
\begin{equation}
  \sum_{j=1}^n 1-(1-\lambda{}_j)^n
\end{equation}
Um Gleichung \ref{equation:finding-usability-problems} zu reproduzieren, müsste $\lambda{}_j = \lambda{}$ für alle $j$ gelten. Das bedeutet, dass die Wahrscheinlichkeiten, einzelner Probleme gefunden zu werden, eine geringe Variabilität besitzen. Die von \textcite{nielsenMathematicalModel1993} eingeführte Gleichung \ref{equation:finding-usability-problems} wird \textcite{woolrychWhyWhen2001} zufolge unzuverlässiger, sobald eine größere Varianz im System und bei den Teilnehmer*innen existiert. Sich unterscheidende Interaktionen mit dem System und die Auswahl der Aufgaben beinflussen die Zuverlässigkeit eines einzigen Wertes $\lambda{}$.
\parencite{woolrychWhyWhen2001}

\todo{Fazit für den Abschnitt? \textcite{barnumUsabilityTesting2021} stimmt eher \textcite{nielsenMathematicalModel1993} zu, aber mit Abhandlung von vorher passt das nicht mehr zusammen}

\subsection{Personas}
\todo{find more sources? basically only citing \textcite{tomlinUXOptimization2018}}

Das Konzept von Personas wurde zuerst von \textcite{cooperInmatesAre1999} eingeführt. Er erstellte sie, um eine fiktive Person als eine Zusammenfassung von gemeinsamen Bedürfnissen, Hintergründen und Vorwissen zu erstellen und somit zu definieren, was eine "typische" Nutzer:in tun müsste, um Erfolg mit einer Anwendung zu haben. Auch er gab dieser Persona bereits eine Hintergrundgeschichte, Ziele und Gründe zum Benutzen der Applikation. Sie vereinfachen das Design und die Entwicklung indem sie den Fokus von den Funktionalitäten auf die Endnutzer:innen lenken und die Frage stellen, was sie benötigen, um erfolgreich zu sein. Somit sind Personas ein wichtiges Werkzeug des \textit{"user-centered design"}.
\parencite{cooperInmatesAre1999, tomlinUXOptimization2018}

Für Usability Testing stellen Personas eine wichtige Rolle dar, da sie helfen, die richtigen Teilnehmer:innen zu finden.
\parencite{tomlinUXOptimization2018}

Laut \textcite{tomlinUXOptimization2018} werden gute Personas durch mehrere Merkmale geprägt. Sie basieren auf Feldforschung, bei der Informationen von potenziellen Nutzer:innen gesammelt werden, um gemeinsame Verhaltensmuster zu identifizieren. Personas sollten sich auf den gegenwärtigen Zustand fokussieren und die typische Umgebung bei der Verwendung der Anwendung beinhalten, sowie die benutzten Geräte. Des Weiteren werden ein Bild, Name und eine kurze Geschichte dazu verwendet, die beschriebene Person zu vermenschlichen. Dabei sollte ein typisches Problem oder eine Aufgabe beschrieben werden, die es gilt zu überwinden. Die zwei oder drei wichtigsten Aufgaben können auch in einer Reihenfolge priorisiert werden. Zu guter Letzt betont \textcite{tomlinUXOptimization2018}, dass gute Personas auch Details über den typischen Grad der Expertise beinhalten sollten. Diese Informationen sollten detailliert auf die Anwendung und die Problemstellung angepasst werden.
\parencite{tomlinUXOptimization2018}

\subsubsection{Persona-Arten} \todo{besserer Abschnittsname?}

\textcite{tomlinUXOptimization2018} identfiziert drei verschiedene Arten von Personas: Design und Marketing Personas, sowie Proto-Personas. Im Folgenden sollen diese vorgestellt werden.

\textbf{Design Personas} stellen die Nutzer:innen, ihre Ziele und Aufgaben in den Mittelpunkt. Laut \textcite{tomlinUXOptimization2018} ist es am Wichtigsten, dass Design Personas auf Feldforschung und Beobachtung von Nutzer:innen in ihrer eigenen Umgebung basieren. Diese Personas stellen Kontext für Designentscheidungen dar und können den Einfluss von Einzelpersonen einschränken, indem sie eine fundierte Basis bilden. \todo{Formulierung} Da sie sich auf wichtige Aufgaben konzentrieren, kann durch Design Personas der Entwicklungsprozess optimiert und in die richtige Richtung gelenkt werden. Das betrifft sowohl Designentscheidungen, als auch Usability-Studien. Abschließend ist anzumerken, dass eine gute Persona dieser Art ein wichtiges Werkzeug für \textit{user-centered design} darstellt.
\parencite{tomlinUXOptimization2018}

\textbf{Marketing Personas} unterscheiden sich von Design Personas, da sie den Fokus auf das Kaufverhalten, Meinungen und Einstellungen der potentiellen der potentiellen Kundschaft legen. Daher beinhalten sie typischerweise nicht die wichtigsten Aufgaben, sondern Informationen über Kaufmotivationen und -bedürfnisse. Anstatt von Feldforschung werden oft quantitative Datensätze basierend auf Primär- oder Sekundärforschung verwendet. \textcite{tomlinUXOptimization2018} beschreibt dass es selten möglich ist, Marketing Personas anstelle von Design Personas zu benutzen, da sie wichtige Aufgaben beschreiben und auf Feldforschung basieren sollten.
\parencite{tomlinUXOptimization2018}

\textbf{Proto-Personas} werden von \textcite{tomlinUXOptimization2018} als am Weitesten verbreitet angesehen, da sie kostengünstig sind und nicht viel Zeit benötigt wird, um sie zu erstellen. Proto-Personas basieren oft auf Sekundärforschung, oder sogar nur auf der Intuition der Designer:innen. Obwohl die Verwendung einer Proto-Persona einem Vorgehen ohne Personas vorzuziehen ist, ist es empfehlenswert, sie im Laufe der Zeit zu validieren und zu verbessern. Sie können somit einen guten Startpunkt für agile Prozesse in Design und Entwicklung darstellen und nach einiger Zeit, mit gesammelten Erkenntnissen, zu Design Personas aufgewertet werden.
\parencite{tomlinUXOptimization2018}

Im Rahmen dieser Arbeit sind \textbf{Design Personas} und \textbf{Proto-Personas} von Interesse.

\todo{Überlegen ob dem Persona Abschnitt nochwas fehlt}
\todo{Oder auch, ob nach Personas noch ein Abschnitt fehlt (vor Aufgaben und Scenarios)}

\subsection{Szenarios}

Damit die Benutzung einer Anwendung bei Usability Tests beobachtet und bewertet werden kann, müssen
den Teilnehmer:innen spezifische und realistische Aufgaben gegeben werde. Dies betonen
\textcite{mccloskeyTaskScenarios2014} und \textcite{barnumUsabilityTesting2021}. Szenarios werden
verwendet, um die Aufgaben mit einer Erklärung und Kontext anzureichern
\parencite{mccloskeyTaskScenarios2014}. Sie bilden somit einen Weg den Nutzer:innen zu beschreiben,
welche Aktionen sie in der Anwendung ausführen sollen, ohne ihnen konkret zu beschreiben, was sie
tun sollen.

\textcite{mccloskeyTaskScenarios2014} beschreibt drei Eigenschaften von guten Aufgaben: Sie sollten
realistisch sein, als Handlung ausführbar sein und es vermeiden Tipps zu geben oder Unterschritte zu
beschreiben.

Eine realistische Aufgabe macht es einfacher für Nutzer:innen sich in das Szenario zu versetzen und
die Aufgabe auszuführen, während sie die Oberfläche benutzen. Es kann einfacher sein, den
Teilnehmer:innen etwas Spielraum in den Details der Aufgabe zu geben, damit sie so vorgehen können,
wie es für sie realistisch ist.
\parencite{mccloskeyTaskScenarios2014}

Außerdem ist es besser, nach einer Handlung zu fragen, statt sie beschreiben zu lassen.
\textcite{nielsenFirstRule2001} geht darauf ein, dass dies immer besser ist, da eine Einschätzung
nicht so genaue Ergebnisse birgt wie die Beobachtung der Handlung. Grund für diese Annahme ist die
Forschungsarbeit von \textcite{nielsenMeasuringUsability1994}, aus der hervorgeht, dass Nutzer:innen
bei zwei unterschiedlichen Designs eine Meinung darüber abgeben können, welches besser ist, aber
diese Information wird immer genauer sein, wenn sie die Möglichkeit hatten, die Oberfläche selbst zu
benutzen. Die Interaktion mit der Anwendung kann Einblicke in Problembereiche und Frustrationen
bieten, die durch eine einfache Beschreibung nicht verfügbar gewesen wären.
\parencite{mccloskeyTaskScenarios2014} \todo{Zitierung in diesem Paragraph überprüfen (alles stammt
von \textcite{mccloskeyTaskScenarios2014}, außer die Sätze die extra mit Zitationen versehen sind)}

Zu guter Letzt sollte es vermieden werden, den Teilnehmer:innen in der Aufgabenbeschreibung Tipps zu
geben oder den Lösungsweg zu detailliert auszuführen. Ausdrücke, die in der Benutzeroberfläche
verwendet werden, sollten vermieden werden, da so das Verhalten der Teilnehmer:innen beeinflusst
wird \parencite{mccloskeyTaskScenarios2014, barnumUsabilityTesting2021}. Beschreibungen von
Unterschritten enthalten oft Tipps über den Aufbau und die vorgesehene Art und Weise der
Anwedungsbenutzung. Dies kann dazu führen, dass weniger hilfreiche Informationen über das Verhalten
der Nutzer:innen gewonnen werden. \textcite{barnumUsabilityTesting2021} betont deshalb dass es
besser ist ein Ziel anzugeben, statt einer Reihenfolge an Schritten um dieses Ziel zu erreichen.
\parencite{mccloskeyTaskScenarios2014}

\todo{schauen, wohin dieser Paragraph passt}
Gute gewählte und formulierte Aufgaben haben den Vorteil, dass Verhaltensmuster, und häufig
auftretende Muster erkannt werden können \parencite{barnumUsabilityTesting2021}. Des Weiteren
besteht die Möglichkeit, die Erfolgsrate als grobes Maß für die Usability zu benutzen
\parencite{nielsenSuccessRate2001}. Andere Maße und Informationen sollten jedoch weiterhin gesammelt
werden, da Erfolg nur von einem Mindestmaß an Usability spricht \parencite{nielsenSuccessRate2001}.

\subsection{Feedback}
\todo{Performance Data wurde als Leistungsdaten übersetzt, und Preference data als
  Präferenzinformationen - ok so?}

Feedback-Methoden können abhängig vom Typ der Studie variieren. Formative Studien profitieren von
qualitativen Methoden, während formative Studien quantitatives Feedback einsetzen. Laut
\textcite{barnumUsabilityTesting2021} kann in beiden Fällen eine Benutzung der jeweils anderen
Herangehensweise die Erkenntnisse erweitern. Wie bereits in \ref{sec:formative-summative}
festgestellt, ist es in zeitigeren Entwicklungsstadien vorteilhafter, formative Studien
durchzuführen. \todo{Überprüfen ob a) stimmig und b) so beschrieben} Qualitative Aussagen von
Teilnehmer:innen können wertvolle Einblicke in die größten Problembereiche geben und auf die
nächsten Entwicklungsschritte hindeuten.
\parencite{barnumUsabilityTesting2021}

\textcite{barnumUsabilityTesting2021} nennt Leistungsdaten und Präferenzinformationen als
quantitative Maße für Usability. Als Leistungsdaten gelten zum Beispiel die Zeit zum Absolvieren von
Aufgaben, Fehlerquote oder Erfolgsrate. Wie bereits in \ref{sec:scenarios} angeschnitten,
spricht eine hohe Erfolgsrate jedoch nur von einem Mindestmaß an Usability und sollte durch weitere
Werte unterstützt werden. Des Weiteren weißt \textcite{barnumUsabilityTesting2021} darauf hin, dass
auch unterschiedliche Grade von Erfolg beim Absolvieren von Aufgaben bestehen. So könnte eine
Nutzer:in den schnellsten, vom Design vorgesehenen Weg zum Ziel benutzen, oder aber auch einen
indirekten Pfad nehmen. Auch verschiedene in Anspruch genommene Hilfstellungen können von Usability
Problemen sprechen. Das Bewerten einer Aufgabe als Misserfolg kann auch verschiedene Gründe haben:
Aufgeben, Abruch durch die Testleitung, oder die Annahme, dass die Aufgabe beendet sei, ohne dass
sie das wirklich ist. Eine weitere Metrik die einfach zu erheben ist, ist die Zeit zum Vollenden der
Aufgaben. In der ersten Studie können diese Werte als Ausgangsbasis gesammelt werden, und in
zukünftigen Studien zum Vergleich benutzt werden, so \textcite{barnumUsabilityTesting2021}. Zu
beachten ist, dass diese einfach zu erhebenden Metriken nicht die gesamte Usability Erfahrung
beschreiben können.
\parencite{barnumUsabilityTesting2021}

Als Präferenzinformationen beschreibt \textcite{barnumUsabilityTesting2021} Antworten auf
Fragebögen, welche nach Aufgaben und nach dem kompletten Test erhoben werden. Befinden sie sich auf
Skalen (1 bis 5 oder 1 bis 10), können sie als quantitative Daten benutzt werden. Fragen mit offenem
Ende lieferen qualitative Informationen über die Erlebnisse der Teilnehmer:innen. Zusätzlich sollten
auch über Thinking-Aloud gewonnene Eindrücke berücksichtigt werden und, wenn verfügbar,
nonverbales Feedback wie Körperspräche oder nonverbale Ausdrücke gesammelt werden.
\parencite{barnumUsabilityTesting2021}

\subsubsection{Fragebögen}
Nach jedem Szenario Feedback über die Aufgabe zu sammeln, hat laut
\textcite{barnumUsabilityTesting2021} den Vorteil, dass die Erinnerungen noch frisch sind. Dabei
kann es sich auch um eine einzige Frage handeln. \textcite{sauroIfYou2010} nennt als Eigenschaften
eines guten Fragebogens, dass er zuverlässig, sensibel, valide und kurz ist, sowie einfach zu
beantworten, zu handhaben und zu bewerten sein sollte. \textcite{barnumUsabilityTesting2021} schlägt
vor eins oder mehr der folgenden Themen abzufragen: Schwierigkeit der Durchführung, benötigte Zeit
(von "weniger als erwartet" bis "mehr als erwartet"), die Wahrscheinlichkeit, dass dieses Feature
erneut benutzt wird, und das Vertrauen in die erfolgreiche Bewältigung der Aufgabe.

\textcite{sauroIfYou2010} listet folgende Standard-Fragebögen für das Einholen von Feedback nach
Aufgaben auf: \ac{asq}, \ac{nasa-tlx}, \ac{smeq}, \ac{ume} und \ac{seq}. \todo{bessere Überleitung?
  maybe selbsterschließend?}

Der \textbf{\ac{asq}} wurde von \textcite{lewisPsychometricEvaluation1991} eingeführt und beinhaltet
drei Fragen bezüglich Schwierigkeit der Aufgabe, Bearbeitungszeit und der Menge der unterstützenden
Informationen (Dokumentation, Online-Hilfe, etc.). Die Antworten werden auf einer Skala von 1
(stimme voll zu) bis 7 (stimme überhaupt nicht zu) angegeben.

Der \textbf{\ac{nasa-tlx}} wurde von \textcite{hartDevelopmentNASATLX1988} entwickelt und berechnet
eine Gesamtbewertung der Arbeitsbelastung basierend auf, geistiger, physischer und zeitlicher
Anforderung, sowie Leistung, Aufwand und Frustration.
\parencite{nasaNASATLX}

Der \textbf{\ac{smeq}}, zuerst von \textcite{zijlstraConstructionScale1985} beschrieben, besteht aus
einer einzelnen Skala von "gar nicht anstrengend" bis "extrem anstrengend".
\todo{schwer vs schwierig?}

Bei \textbf{\ac{ume}} handelt es sich um eine Methode, bei der nach jeder Aufgabe ein numerischer
Wert von den Teilnehmer:innen erfragt wird, welcher sich auf die Aufgabe bezieht. Die erste Antwort
kann arbiträr sein, die restlichen orientieren sich dann aber an den vorherigen
Antworten - somit können die Aufgaben untereinander verglichen werden. Zu
betonen ist, dass die Skala im Vorhinein nicht festgelegt ist, und von den
Teilnehmer:innne selbst gewählt wird.
\parencite{mcgeeUsabilityMagnitude2003}

Die \textbf{\ac{seq}} besteht aus einer einzigen Frage: "Insgesamt war diese Aufgabe...",
wobei von einer Skala von 1 (sehr schwierig) bis 5 (sehr einfach) ausgewählt werden kann.
\textcite{tedescoComparisonMethods2006} stellte fest, dass diese Methode bei einer kleinen
Stichprobengröße die beständigsten Ergebnisse liefert \footnote{Es fand ein Vergleich mit 4 anderen
  Methoden statt: \ac{asq}, \ac{ume}, eine Variation von \ac{seq} und die Erwartungs-Bewertung von
  \textcite{albertThisWhat2003}}. \textcite{sauroComparisonThree2009} verglich die \ac{seq} mit
\ac{smeq} und \ac{ume}. Dabei punktete \ac{seq} mit einfacher Bedienbarkeit und Erlernbarkeit,
während alle Methoden eine ähnliche Zuverlässigkeit aufwiesen.

\textcite{barnumUsabilityTesting2021} erklärt, dass Fragebögen nach Szenarios an die soeben
absolvierte Aufgabe angepasst werden können. Während das gleiche für Fragebögen nach dem gesamten
Test gilt, existieren auch da Standard-Fragebögen, wie \ac{sus}, \ac{csuq} und \ac{nps}.

\textbf{\ac{sus}} wurde von \textcite{brookeSUSQuick1996} \todo{nochmal nachschauen ob das wirklich
  erste Publikation ist, weil \textcite{barnumUsabilityTesting2021} (S.233) von 1986 schreibt (auch in
  Quellen)} vorgestellt und besteht aus 10 Fragen, die auf einer Likert-Skala bewertet werden.
\todo{Muss Likert-Skala erklärt werden?} Um den Gesamtwert-Wert zu berechnen, werden die einzelnen
Antworten miteinander addiert und mit 2,5 multipliziert, um einen \ac{sus}-Wert zwischen 0 und 100
zu erhalten. \textcite{sauroMeasuringUsability2011} führte eine Meta-Analyse von 500 Studien durch
und errechnete einen durchschnittlichen \ac{sus}-Wert von 68. Dieser Werte könnte laut
\textcite{barnumUsabilityTesting2021} als Basiswert für iterative Studien benutzt werden.
\todo{überdurschnittlich und so erklären?}

\textcite{brookeSUSQuick1996} beschreibt folgende Vorgehensweise zur Anwendung von \ac{sus}: Die
Fragen sollten direkt nach dem Test gestellt werden, bevor etwaige andere Aktivitäten durchgeführt
werden. Des Weiteren sollte Teilnehmer:innen ihre direkte Antwort geben, statt zu viel darüber
nachzudenken. Schlussendlich betont \textcite{brookeSUSQuick1996}, dass optimalerweise eine Antwort
auf jede Frage gegeben wird - im Falle von Meinungslosigkeit kann das der Mittelpunkt sein. Laut
\textcite{barnumUsabilityTesting2021} ist die \ac{sus} sehr weit verbreitet, da sie schnell
anzuwenden ist, unabhäng von der Technologie des getesteten Produktes ist und sich über die Zeit als
valide Methode für Studien ab fünf Teilnehmer:innen erwiesen hat.

\textbf{\ac{csuq}} ist sehr ähnlich zum \ac{pssuq}. Beide beinhalten 16 positive Fragen, welche auf
einer Skala mit 7 Punkten bewertet werden. Im Kontrast zu \ac{sus} wird auch eine Bewertung von
"nicht anwendbar" erlaubt. Die Fragebögen ergeben einen Gesamtwert und drei Unterbwertungen von
Systemqualität, Informationsqualität und Oberflächenqualität. \parencite{barnumUsabilityTesting2021}
\todo{vielleicht auch noch Originalquelle für \ac{csuq}/\ac{pssuq} finden?}

Beim \textbf{\ac{nps}} handelt es sich um eine einzige Frage, die ungefähr wie folgt lautet: "Wie
wahrscheinlich ist es, dass Sie [Unternehmen/Produkt] weiterempfehlen?". Dabei wird die Antwort auf
einer Skala zwischen 0 und 10 gegeben. Teilnehmer:innen, die einen Wert unter 7 angeben, werden als
\textit{Detractors} bezeichnet, bei einer Antwort über 8 werden sie als \textit{Promoters}
eingestuft. Der \ac{nps}-Wert wird aus einer Subtraktion des Prozentsatzes der \textit{Detractors}
vom Prozentsatz der \textit{Promoters} gewonnen und kann somit zwischen -100 und 100 liegen.
\todo{explizites Plus?} Aufgrund einer Untersuchung der Korrelation zwischen \ac{sus} und
\ac{nps} von \textcite{sauroDoesBetter2010} stellt \textcite{barnumUsabilityTesting2021} fest, dass
eine Daumenregel zum errechnen des \ac{nps}-Wertes ist, den \ac{sus}-Wert durch 10 zu teilen.
\todo{Vielleicht noch Kritik am NPS finden?}


\clearpage
\section{Analyse und Vergleich von existierenden Projekten}
\subsection{Home Assistant}
\subsection{DBSnap}
\url{https://www.semanticscholar.org/paper/DBSnap\%3A-Learning-Database-Queries-by-Snapping-Silva-Chon/aabf7791575f8cc62432ca5e6289862eec0f6845}
\url{https://www.semanticscholar.org/paper/DBSnap-2\%3A-New-Features-to-Construct-Database-by-Silva-Loza/61c95757d7d4151f51ea848623c6839816756385}
\subsection{SQheLper}
\url{https://www.semanticscholar.org/paper/SQheLper\%3A-A-block-based-syntax-support-for-SQL-Jacobs-Jaschke/b4f6023a4ee72d0103ee521c4f9b35768d38810b}
\subsection{Scratch}


\chapter{Implementierung}
\section{Zielformulierung}
\subsubsection{Genaue Formulierung der Problemstellung, was ist das Ziel?}
\section{Lösungsvorstellung}
\subsubsection{Erklärung des Resultats}

\chapter{Studie}
\section{Studienaufbau}
\subsection{Problemformulierung und Ziele}
\subsection{Personas}
\subsubsection{Personas aus \shorthandtextcite{unisd2021}}
\subsubsection{Shorthand: \shorthandtextcite*{unisd2021}}
\subsection{Auswahl von Teilnehmer*innen}
\subsubsection{Auf \ref{section:study-size} eingehen, wieso dürfte Größe reichen (wenig persönliche Entscheidungen, konkrete Aufgaben, etc.)}
\subsection{Aufgaben}
\subsection{Szenarios}
\subsection{Evaluation}
\subsection{Studienablauf}
\section{Ergebnisse und Analyse}
\section{Schlussfolgerungen und Zukunftsaussicht}

\chapter{Fazit}


\nocite{*}
\chapter*{Quellen}
\begingroup
\let\clearpage\relax
\printbibliography[filter=articles, title=Artikel, heading=subbibliography]
\printbibliography[filter=books, title=Bücher, heading=subbibliography]
\printbibliography[filter=other, title=sonstige, heading=subbibliography]
\endgroup

\printindex

\end{document}
