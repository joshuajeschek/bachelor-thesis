\subsection{Studiengröße}
\label{sec:study-size}

Wie in \ref{sec:formative-summative} erwähnt, existieren Usability Tests in unterschiedlichen Umfängen. Formative Studien sind klein und werden meist während der Entwicklung durchgeführt, während summative Studien mehr Teilnehmer:innen umfassen und gegen Ende der Entwicklung stattfinden.

Für diese Arbeit werden formative Studien von Relevanz sein. Dabei sollte die Anzahl der Teilnehmer:innen so klein wie möglich zu halten, während so viele Usability Probleme wie möglich entdeckt werden, denn eine höhere Anzahl an Testdurchläufen steigert die Kosten und den Zeitaufwand \parencites{faulknerFiveuserAssumption2003, nielsenWhyYou2000}.

\textcite{nielsenMathematicalModel1993} beschreiben den Anteil an gefundenen Usability Problemen bei
$n$ Teilnehmer:innen als wie folgt:
\begin{equation}
  \label{equ:finding-usability-problems}
  N(1-(1-\lambda{})^n),
\end{equation}
wobei $N$ die Gesamtanzahl der Probleme und $\lambda{}$ der Anteil der Probleme die bei einem einzigen Testdurchlauf gefunden werden. Mit dem von ihm über mehrere Projekte beobachtete Wert von $\lambda{}=31\%$ kommt \textcite{nielsenWhyYou2000} zum Resultat, dass mit fünf Teilnehmer:innen 85\% aller Probleme festgestellt werden können. \cite{nielsenWhyYou2000} Weitere Testdurchläufe sind weniger lohnenswert, da nicht genug neue Probleme aufgedeckt werden. Um alle Probleme mit einer Studie zu finden, wären 15 Teilnehmer:innen nötig, die gegen Ende jedoch nur noch vereinzelt neue Erkenntnisse liefern. Da Teilnehmer:innen oft die gleichen, oberflächlichen Fehler finden und dadurch abgelenkt werden, ist es von Vorteil nach den ersten fünf Durchläufen die gefundenen 85\% zu beheben, und dann eine weitere Studie durchzuführen. Eine dreifache Wiederholung der Tests mit fünf Durchläufen und einem verbesserten Produkt führt zu umfangreicheren Resultat als ein einziger Test mit 15 Durchläufen \parencite{nielsenWhyYou2000}.

Die als Faustregel adaptierte Zahl Fünf wurde in mehreren Publikationen untersucht. Während fünf Teilnehmer:innen durchschnittlich 85\% der Probleme (einer Studie) entdecken können, kann dieser Prozentsatz laut \textcite{faulknerFiveuserAssumption2003} jedoch bis auf 55\% sinken. Während die durchschnittliche Rate langsamer steigt, verringert sich die Varianz mit mehr Testdurchläufen. Außerdem ist anzumerken, dass die Frage, wie viele Teilnehmer:innen benötigt werden schwer zu beantworten ist, und von vielen Faktoren abhängt\footnote{genannt werden: Art und Erfahrung der Teilnehmer:innen, Wichtigkeit des Systems, mögliche Folgen von Usability Problemen} \parencite{faulknerFiveuserAssumption2003}.


\textcite{spoolTestingWeb2001} fanden einen deutlich geringeren Wert von $\lambda{}=10\%$ bei unstrukturiertem Testen von E-Commerce-Webseiten. Somit würden mit 5 Testdurchläufen nur 35\% der Probleme gefunden werden. Als Gründe dafür kann die Komplexität der Webseite und die persönlich getroffenen Entscheidungen der Teilnehmer:innen genannt werden \parencite{spoolTestingWeb2001}.

Laut \textcite{woolrychWhyWhen2001} ist die Wahrscheinlichkeit, dass ein Problem entdeckt wird, nicht konstant sondern hängt von Schweregrad, Benutzermerkmalen, Produkttyp und Strukturierungsgrad des Testes ab. Sie erweitern deshalb Gleichung \ref{equ:finding-usability-problems} um $\lambda{}_j$, welches die Wahrscheinlichkeit für jedes Usability Problem darstellt, dass das Problem von einer zufällig ausgewählten Teilnehmer:in gefunden wird. Die erwartete Nummer an gefundenen Problemen beträgt dann laut \textcite{woolrychWhyWhen2001}:
\begin{equation}
  \sum_{j=1}^n 1-(1-\lambda{}_j)^n
\end{equation}
Um Gleichung \ref{equ:finding-usability-problems} zu reproduzieren, müsste $\lambda{}_j = \lambda{}$ für alle $j$ gelten. Das bedeutet, dass die Wahrscheinlichkeiten, einzelner Probleme gefunden zu werden, eine geringe Variabilität besitzen. Die von \textcite{nielsenMathematicalModel1993} eingeführte Gleichung \ref{equ:finding-usability-problems} wird \textcite{woolrychWhyWhen2001} zufolge unzuverlässiger, sobald eine größere Varianz im System und bei den Teilnehmer:innen existiert. Sich unterscheidende Interaktionen mit dem System und die Auswahl der Aufgaben beinflussen die Zuverlässigkeit eines einzigen Wertes $\lambda{}$ \parencite{woolrychWhyWhen2001}.

Meinungen darüber, wie viele Teilnehmer:innen für eine Usability-Studie angemessen sind, varrieren in der Literatur. \textcite{nielsenHowMany2012} räumt ein, dass summative Studien mehr Nutzer:innen benötigen, um aussagekräftige Zahlen zu generieren. Außerdem hängt es von den Umständen einzelner Projekte ab, wie viele Teilnehmer:innen sinnvoll sind. Dazu gehört die Art des Projektes, der Umfang der Usability-Studie, sowie die verfügbare Zeit und finanziellen Mittel \parencite{nielsenHowMany2012}.
