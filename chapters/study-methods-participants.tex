\subsection{Auswahl von Teilnehmer:innen}

Für die Usability-Studie wurden Personen aus zwei unterschiedlichen Gruppen ausgewählt, die beide zu den typischen Nutzer:innen von Simplex4TwIS gehören.

Einerseits handelt es sich dabei um Angestellte von Simplex4Data, die im Arbeitsalltag Daten von Kund:innen bearbeiten und das System bedienen. Viele von ihnen haben bereits Erfahrung mit dem alten Editor für Konvertierungen gemacht und sind mit dem von Simplex4TwIS genutzten Modellierungsansatz vertraut. Sechs Angestellte von Simplex4Data nahmen an der Studie teil.

Des Weiteren wurden Kund:innen von Simplex4Data persönlich angefragt, oder mittels eines Anschreiben kontaktiert (Anhang \ref{app:invitation}). Zum Teil verwalten sie bereits aktiv Umweltdaten im System Simplex4TwIS, oder wollen es in Zukunft vermehrt anwenden. Auf diese Weise wurden vier weitere Testpersonen gewonnen.

Die Teilnehmer:innen wiesen unterschiedliche Erfahrungsniveaus hinsichtlich ihrer \ac{SQL}-Kenntnisse und ihrer Programmiererfahrung auf. Drei von ihnen, darunter zwei Kund:innen von Simplex4Data, verfügten über keine oder nur grundlegende Kenntnisse in diesen Bereichen. Die Erfahrung der übrigen Teilnehmer:innen reichte von grundlegenden Kenntnissen bis hin zu ausgereifter Expertise.

Wie von \textcite{nielsenWhyYou2000} beschrieben, können bereits mit wenigen Nutzer:innen viele Usability Probleme aufgedeckt werden. Dies ist insbesondere für die vorliegende Studie der Fall, da nicht das gesamte System getestet werden soll \parencite[Vgl. \ref{sec:formative-summative}]{spoolTestingWeb2001}. Getestet wird nur die vorliegende Teilanwendung, die sich auf eine einzelne Oberfläche beschränkt. Aufgrund der gegebenen Aufgaben können nur wenige persönliche Entscheidungen getroffen werden. Korrekte Lösungen können nur auf wenige unterschiedliche Weisen umgesetzt werden. Deshalb wird davon ausgegangen, dass die wenigen Testsitzungen den Großteil der Nutzungsmöglichkeiten innerhalb der Szenarios abdecken. Da \ac{CTA} verwendet wurde und jede Testsitzung eine Länge von 90 Minuten hatte, beschäftigte sich jede Person ausführlich mit der Anwendung.
