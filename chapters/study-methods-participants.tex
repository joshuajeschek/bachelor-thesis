\subsection{Auswahl von Teilnehmer:innen}

Für die Usability-Studie wurden Personen aus zwei unterschiedlichen Gruppen ausgewählt, die beide zu den typischer Nutzer:innen von Simplex4Data gehören.

Einerseits handelt es sich dabei um Angestellte von Simplex4Data, die im Arbeitsalltag Daten von Kund:innen bearbeiten und das System bedienen. Viele von ihnen haben bereits Erfahrung mit dem alten Editor für Konvertierungen gemacht und sind mit dem von Simplex4Data genutzten Modellierungsansatz vertraut. Es wurden sechs Angestellte von Simplex4Data getestet.

Des Weiteren wurden Kund:innen von Simplex4Data mittels eines Anschreiben oder persönlich angefragt (Anhang \ref{app:invitation}). Sie verwalten zum Teil bereits aktiv Umweltdaten im System Simplex4Data, oder wollen es in Zukunft vermehrt anwenden. Auf diese Weise wurden vier weitere Testpersonen gewonnen.

\todo[noline]{passt das so oder wirkt das zu sehr nach Schönreden?}
Wie von \textcite{nielsenWhyYou2000} beschrieben, können bereits mit wenigen Nutzer:innen viele Usability Probleme aufgedeckt werden. Dies ist insbesondere für die vorliegende Studie der Fall, da nicht das gesamte System getestet werden soll, sondern nur die vorliegende Teilanwendung, die sich auf eine einzelne Oberfläche beschränkt. Aufgrund der gegebenen Aufgaben können nur wenige persönliche Entscheidungen getroffen werden, korrekte Lösungen werden können nur auf wenige unterschiedliche Weisen umgesetzt werden. Deshalb wird davon ausgegangen, dass der Großteil der Nutzungsszenarien durch die Testsitzungen abgedeckt wurden. Außerdem beschäftigte sich jede Testperson ausführlich mit der Anwendung, aufgrund der Verwendung von \ac{CTA} und der Länge der Testsitzungen von 90 Minuten.
