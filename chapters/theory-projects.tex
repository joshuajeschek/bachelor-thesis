\section{Verwandte Arbeiten}
\label{sec:projects}

In diesem Abschnitt wird ein Blick auf für die Projekte geworfen, die für die vorliegende Arbeit von Relevanz sind. Zum einen wurde \textit{Home Assistant} ausgewählt, da die darin enthaltenen visuellen Editoren eine Inspiration für die blockbasierte Herangehensweise darstellen, die im entwickelten Prototyp verwendet wurde. Des Weiteren wird die blockbasierte Programmierumgebung \textit{Scratch} untersucht, \textit{Snap!}, welches starke Ähnlichkeiten zu \textit{Scratch} aufzeigt, sich jedoch mit einem erweiterten Funktionsumfang an ein erfahreneres Publikum richtet. Außerdem wird \textit{DataSnap}, eine Erweiterung für \textit{Snap!}, die es zu einem Werkzeug für Fachexpert:innen machen soll.

Weitere Projekte, die ähnliche Themen behandeln beinhalten:
\begin{itemize}
  \item \textit{\citetitle{pazosr.InterfaceComposing2015}}: Entwicklung einer Abfrageoberfläche für Datenbanken, die von unerfahrenen Nutzer:innen benutzt werden kann, die kein Wissen über den Inhalt und das Schema der Datenbank besitzen \parencite{pazosr.InterfaceComposing2015}.
  \item \textit{DBSnap}: Eine Anwendung zum Erstellen von Datenbankabfragen, die es Nutzer:innen erlaubt, Blöcke nach Regeln der relationalen Algebra miteinander zu verknüpfen und somit Abfragebäume und Datenabfragen zu erlernen \parencite{silvaDBSnapLearning2015}. Diese Anwendung wurde durch \textit{DBSnap++} \parencite{silvaDBSnapCreating2018} und \textit{DBSnap 2} \parencite{silvaDBSnapNew2022} weiterentwickelt.
  \item \textit{SQheLper}: Diese Webanwendung ermöglicht es, durch das Zusammensetzen von Blöcken \acs{SQL}-Abfragen zu generieren, und so die Anzahl der Syntax-Fehler zu reduzieren \parencite{jacobsSQheLperBlockbased2021}.
\end{itemize}
