\subsection{Umsetzung in dieser Arbeit}

Da sich der entwickelte Block-Editor in einem zeitigen Entwicklungsstadium befindet, wurde eine formative Studie durchgeführt. Die gesammelten Ergebnisse können somit direkt in die Entwicklung einfließen. Die Umstände dieser Arbeit erlauben nur eine geringe Anzahl an Teilnehmer:innen. Wie in \ref{sec:study-size} jedoch gezeigt wurde, kann auch eine geringe Anzahl an Teilnehmer:innen für zielführende Ergebnisse sorgen. Des Weiteren wurden Szenarios genutzt, um realistische Aufgaben zu vermitteln. Abschnitt \ref{sec:study-szenarios} geht tiefer auf die gewählten Szenarios ein. Zur Erhebung von qualitativen Daten wurde das \ac{CTA} genutzt, welches von \textcite{alhadretiRethinkingThinking2018} empfohlen wird. Dem Vorschlag von \textcite{barnumUsabilityTesting2021} folgend, wurden die so gesammelten qualitativen Daten durch quantitative Erhebungsmethoden angereichert. Nach jedem Szenario sollte die Einfachheit der Szenarios bewertet werden. Aufgrund der durch \textcite{sauroComparisonThree2009} beschriebenen einfachen Bedienbarkeit, wurde dafür die \ac{SEQ} gewählt. Am Ende der Testsitzungen wurde außerdem die Einfachheit der Benutzung der gesamten Anwendung bewertet. Abschnitt \ref{sec:study-methods} geht näher auf die eingesetzte Methodik ein.
