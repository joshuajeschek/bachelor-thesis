\subsection{Quantitative Auswertung}
\label{sec:quantitative}

\subsubsection{Bewertung der Szenarios}

Im Anschluss an jedes Szenario wurde die \ac{SEQ} gestellt. Damit sollte überprüft werden, wie die Teilnehmer:innen die Schwierigkeit der einzelnen Aufgaben ansahen und im Vergleich zueinander einschätzen. Es war explizit erwünscht, die drei Einschätzungen zueinander stimmig abzugeben. Die Frage wurden von allen Teilnehmer:innen beantwortet, das heißt für jedes Szenario wurden insgesamt 10 Werte erfragt. Abbildung \ref{fig:seq} zeigt die gesammelten Antworten für jedes Szenario, dargestellt als Box-Plot. Gezeigt werden die Extremwerte (obere und untere Antenne), das obere und untere Quartil (Box) und der Median (dicke Linie).

\begin{figure}[!ht]
  \pgfplotstableread[col sep=comma]{assets/study-results.csv}\datatable
  \centering
  \begin{tikzpicture}
    \begin{axis}[
        boxplot/draw direction=y,
        boxplot/every median/.style={ultra thick},
        xtick={1,2,3},
        xticklabels={Szenario 1, Szenario 2, Szenario 3},
        ytick={1,...,7},
        ymin=1,
        ylabel={\acl{SEQ}},
      ]
      \addplot+ [boxplot, plot1] table [y=a] {\datatable};
      \addplot+ [boxplot, plot2] table [y=b] {\datatable};
      \addplot+ [boxplot, plot3] table [y=c] {\datatable};
    \end{axis}
  \end{tikzpicture}
  \caption{Antworten auf die \acs{SEQ} nach jedem Szenario (1: sehr schwer, 7: sehr einfach).}
  \label{fig:seq}
\end{figure}

In allen Testsitzungen wurde das erste Szenario als am Einfachsten bewertet. Alle Teilnehmer:innen bewerteten es zwischen 5 und 7. Abzüge von Punkten wurde zumeist mit Startschwierigkeiten begründet, die sich im Laufe des ersten Szenarios gelöst haben - nach einer initialen Einfindungsphase bestand die Aufgabe in diesem Szenario daraus, sich wiederholende Zuordnungen durchzuführen.

Das zweite Szenario wurde als etwas schwieriger bewertet, wobei sich die Gesamtheit der Bewertungen über einen größeren Bereich (3,5-7) verteilte. Die gesteigerte Komplexität kann in diesem Szenario größtenteils auf die Verwendung von Funktionen zurückgeführt werden, dies wurde von einigen Teilnehmer:innen als Grund genannt. Personen, die das Szenario als schwieriger bewerteten, hatten in der Regel weniger \ac{SQL}- und Programmiererfahrung, während der Großteil der höheren Bewertungen von internen Testpersonen abgegeben wurde.

Die größte Streuung der Bewertungen kann im dritten Szenario beobachtet werden. Die angegebenen Werte liegen zwischen 3 und 7, wobei das untere und obere Quartil einen relativ großen Bereich abdeckt. Die unterschiedlichen Bewertungen und die Art der aufgetretenen Probleme zeigt, dass die Teilnehmer:innen variierende Erfahrungen während des letzten Szenarios gesammelt haben. Im Vergleich zum zweiten Szenario ist festzustellen, dass sich interne und externe Testpersonen im Wertebereich gleichmäßig verteilen. Vorwissen über Simplex4Data scheint einerseits weniger Wichtigkeit für ein erfolgreiches Abschließen des Szenarios zu haben, andererseits hatten Mitarbeiter:innen von Simplex4Data bisher nur wenige Möglichkeiten Szenarios zu erstellen (im Gegensatz zu Konvertierungen).

Indem in jedem Szenario mehr Funktionalitäten hinzugefügt wurden, sollte die Komplexität im Laufe einer Testsitzung steigen. Die Antworten auf die \ac{SEQ} bestätigen diese Erwartungshaltung nur begrenzt. Wie in Abbildung \ref{fig:seq} zu sehen, fielen die Angaben von Szenario 1 zu Szenario 2 im Median von 6 auf 5, stiegen jedoch im Szenario 3 wieder auf 6,5. Somit kann eine Komplexitätssteigerung zwischen den ersten beiden Szenarios festgestellt werden. Im dritten Szenario kann kein klarer Trend in eine der beiden Richtungen festgestellt werden, größtenteils bedingt durch die starke Streuung der Antworten.

\subsubsection{Bewertung des Block-Editors}

Abbildung \ref{fig:ges} zeigt die Einschätzung der Einfachheit der Benutzung der Anwendung, abgefragt am Ende der Testsitzungen. Während alle Teilnehmer:innen den Block-Editor bewerteten, konnten zwei Personen keine Aussage zur Vorversion treffen, da sie diese nie benutzt haben.

Die Teilnehmer:innen sagten aus, dass sie sich mit dem Block-Editor effizienter fühlen und die Bedienung durch Funktionalitäten wie die Typ-Filterung vereinfacht wird (Vgl. \ref{sec:qualitative}). Die Bewertungen bestätigen diese Aussagen.

\begin{figure}[!ht]
  \pgfplotstableread[col sep=comma]{assets/study-results.csv}\datatable
  \centering
  \begin{tikzpicture}
    \begin{axis}[
        boxplot/draw direction=y,
        boxplot/every median/.style={ultra thick},
        xtick={1,2},
        xticklabels={Vorversion, Block-Editor},
        ytick={1,...,10},
        ymin=1,
        ylabel={Einfachheit der Benutzung},
      ]
      \addplot+ [boxplot, plot1] table [y=prev] {\datatable};
      \addplot+ [boxplot, plot2] table [y=ges] {\datatable};
    \end{axis}
  \end{tikzpicture}
  \caption{Einschätzung der Einfachheit der Benutzung nach dem Usability
    Test auf einer Skala von 1 (sehr schwer) bis 10 (sehr einfach)}
  \label{fig:ges}
\end{figure}

Kein:e Teilnehmer:in bewertete die Vorversion einfacher als den Block-Editor, alle Personen, die bereits zuvor Erfahrung mit dem existierenden System gesammelt hatten konnten ein eindeutige Verbesserung der Einfachheit der Benutzung feststellen. Die Teilnehmer:innen waren sich einig, dass der Block-Editor einfach in der Benutzung ist, mit dem Großteil der Bewertungen über 8. Die schlechtesten Bewertungen für den Block-Editor wurden von Personen abgegeben, die wenig \ac{SQL}- und Programmiererfahrung besitzen, und Schwierigkeiten hatten, das Prinzip der Funktionen korrekt anzuwenden. Im Allgemeinen wurden die Abstriche von der Gesamtpunktzahl damit begründet, dass sich die Benutzer:innen zuerst in das Konzept des Block-Editors einfinden müssen, um produktiv zu sein, und dass ein gewisses Vorwissen von Nöten ist, um fachlich korrekte Abfragen zu erstellen.

Die Antworten bezüglich der Vorversion waren ein wenig gestreuter - den darauffolgenden Aussagen zufolge kann dies mit unterschiedlichen Erfahrungsgraden begründet werden. Die höchsten Bewertungen für die Vorversion wurden von internen Testpersonen getätigt, die äußerst vertraut mit der Funktionsweise von Simplex4Data und dies auch selbst zu ihrer Bewertung aussagten. Teilnehmer:innen mit weniger Erfahrung bewerteten die Vorversion hingegen als schwerer. Als Gründe für die niedrigere Benutzbarkeit der Vorversion wurde Umständlichkeit, Fehleranfälligkeit und die fehlende Sicht auf die Quelldaten genannt.
