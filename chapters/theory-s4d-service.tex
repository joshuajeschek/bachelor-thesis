\subsection{SimplexService}
\label{sec:simplex-service}

Der SimplexService gibt die Daten, die in Simplex4TwIS abgelegt wurden, als \textit{OGC API - Features}-Dienst aus. Dabei stellt jede Objektklasse eine \textit{collection} dar, die darin enthaltenen Objekte sind \textit{items}. Aufgrund der wiederkehrenden Strukturen werden sowohl Nutz- als auch Metadaten in diesem Format ausgeliefert \parencite{grossmannEnvVisioService2022}. Der Dienst erlaubt es, Daten in verschiedenen Formaten wie \acs{CSV}-, \acs{JSON}- oder Shape-Dateien auszugeben.

Eine Liste von wichtigen Endpunkten des SimplexService wird in Quelltext \ref{lst:simplex-service} gezeigt. Die in blau dargestellten Teile entspringen nicht dem \ogcapi-Standard. Diese zusätzlichen Endpunkte werden dafür genutzt, für Simplex4TwIS spezifische Sachverhalte darzustellen \parencite{grossmannEnvVisioService2022}. Dabei handelt es sich einerseits um die SimplexSzenarios: Jedes SimplexSzenario bildet einen eigenen \ogcapi-Dienst. Das Realitätsmodell wird als Szenario mit der \texttt{scenarioID} 1 ausgeliefert. Außerdem beinhaltet der SimplexService weitere Schnittstellen, welche Verbindungen, Metadaten über Verbindungen und verbundene Objekte ausgeben.

\begin{lstlisting}[float=!ht,language=simplexservice,caption={[Ausgewählte Endpunkte des SimplexService]Ausgewählte Endpunkte des SimplexService. Schwarz dargestellte Abschnitte sind Teil des \ogcapi-Standard, während die blauen Teile vom SimplexService hinzugefügt wurden. \parencite{simplex4datagmbhSimplexServiceDokumenation}},label=lst:simplex-service]
/scenarios
/scenarios/{scenarioId}/functions
/scenarios/{scenarioId}/collections
/scenarios/{scenarioId}/collections/{collectionId}
/scenarios/{scenarioId}/collections/{collectionId}/queryables
/scenarios/{scenarioId}/collections/{collectionId}/items
/scenarios/{scenarioId}/collections/{collectionId}/items/{featureId}
\end{lstlisting}

Der \ogcapi-Dienst wird weiter im nächsten Abschnitt erläutert, da er für die Entwicklung des Block-Editors relevant ist. Es wird insbesondere auf \ac{CQL} eingegangen. Da \ac{CQL} den ohnehin unterstützen \ac{OGC}-Standards entspringt, bietet es sich als Abfragesprache an, um Feld-Zuweisungen und -Transformationen anzugeben. Außerdem wird auf die in Quelltext \ref{lst:simplex-service} aufgelisteten \texttt{/functions}- und \texttt{/queryables}-Endpunkte eingegangen, die benutzt werden können, um vom Server unterstützte Funktionen, sowie Standardfelder und Attribute von Objektklassen aufzulisten.
