\label{sec:theory}
Das Datenverwaltungsystem Simplex4TwIS beinhaltet Prozesse zum Importieren, Auswerten und Exportieren von Daten. In Form des Imports und dem Erstellen von Auswertungen werden dabei zwei Schritte benötigt, in denen Datenfelder ausgewählt und miteinander kombiniert werden können. Diese Schritte sollen verbessert werden. Um Daten aus Simplex4TwIS zu exportieren, wird ein \ogcapi-Dienst angeboten. Der Dienst definiert die Abfragesprache \ac{CQL}, mit der Filterausdrücke angegeben werden können. Es bietet sich an, die gleiche Sprache zu nutzen, um Felder beim Importieren und Auswerten von Daten auszuwählen und zu kombinieren. Durch den Dienst können die \ac{CQL}-Ausdrücke dann als die nötigen Datenbankoperationen umgesetzt werden.

In Abschnitt \ref{sec:simplex} wird das System Simplex4TwIS beschrieben, und wie es Daten importiert, verwaltet, ausgewertet und exportiert. Anschließend wird in Abschnitt \ref{sec:ogc} auf den \ogcapi-Dienst eingegangen, und die Abfragesprache \ac{CQL} erläutert. Im darauffolgenden Abschnitt \ref{sec:projects} werden Projekte diskutiert, die ähnliche Probleme zu lösen versuchen oder vergleichbare Ansätze verfolgen wie der Block-Editor, der in dieser Arbeit entstanden ist. Um den Editor zu überprüfen, wurde eine Usability-Studie durchgeführt. Die dafür benötigten Grundlagen des Usability Testings werden in Abschnitt \ref{sec:usability} beschrieben.
