\subsection{Laden und Konvertieren}
\label{sec:simplex-importer}

Der Import von Fachdaten in das Realitätsmodell findet in drei Schritten statt. Zuerst werden die Quelldaten oder deren Ablageort auf dem Server abgespeichert. Es ist möglich, Dateien hochzuladen, vom Server auszuwählen, eine externe \acs{URL} anzugeben oder eine \acs{API}-Schnittstelle zu spezifizieren. Unterstütze Dateiformate beinhalten: Shape-, \acs{CSV}-, \acs{JSON}-, sowie \acs{XML}-Dateien, \textit{OGC API - Features}-Dienste und Datenbanktabellen \parencite{simplex4datagmbhSimplex4TwIS}.

Der zweite Schritt lädt die vorhandenen Daten in Datenbanktabellen, deren Zeilen die jeweils kleinsten Informationselemente der Ausgangsdaten beinhalten \parencite{grossmannFachsystemeSchemaevolution2024}. Dies passiert je nach Ausgangsformat unterschiedlich. In einigen Fällen kann dieser Prozess automatisiert werden. Das Ergebnis des zweiten Schritts stellen eine Datenbanktabellen dar, welche als Importtabellen bezeichnet werden . Zusätzlich werden Metadaten über den erfolgten Importprozess abgelegt, welche die Importtabellen beschreiben, und es ermöglichen den Ladeprozess erneut auszuführen \parencite{grossmannFachsystemeSchemaevolution2024}. Dies ist von Relevanz, um bereits eingepflegte Informationen auf einen neuen Stand zu bringen, falls es beispielsweise im Fachsystem zu neuen Datenerhebungen gekommen ist.

Der letzte Schritt des Imports stellt die Konvertierung von Importtabellen zu Objekten im Realitätsmodell dar. Für diesen Schritt muss definiert werden, welche Informationen aus den Importtabellen, zu welchen Standardfeldern oder Attributen konvertiert werden sollen. Außerdem können basierend auf eventuellen Primär- und Sekundärschlüsseln Verbindungen angelegt werden. Dies kann auch auf komplexeren Wegen passieren, beispielsweise über die geometrische Überschneidung von Geodaten. Um diesen Schritt zu bewältigen, werden zuerst die benötigen Objektklassen mit ihren Attributen und Verbindungsklassen definiert. Danach werden Objektkonvertierungen zum Erstellen der Objekte, Attributkonvertierungen zur Erstellung der dazugehörigen Attribute und Verbindungskonvertierungen zum Verbinden der Objekte definiert. Dabei wird \acs{SQL} verwendet, um die Transformation von Datenbankzeilen zu Entitäten im Realitätsmodell festzulegen \parencite{grossmannFachsystemeSchemaevolution2024}. Für jedes Standardfeld und Attribut, welches aus einer Importtabelle übernommen werden soll, wird eine Bildungsvorschrift festgelegt. Um Verbindungen anzulegen, muss eine \acs{SQL}-Join-Bedingung angegeben werden. Außerdem ist es möglich, Filter zu definieren, und somit die Anzahl der Objekte einzuschränken, die durch eine Konvertierung erstellt werden \parencite{grossmannFachsystemeSchemaevolution2024}.
