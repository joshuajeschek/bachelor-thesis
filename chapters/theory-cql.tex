\section{\acl{cql}}

Bei der \acf{cql} \footnote{Nicht zu verwechseln mit der \textit{Contextual} Query Language, welche
zuerst auch als Common Query Language bezeichnet wurde.
\parencite{thelibraryofcongressCQLContextual2023, ZINGGentle2003}} handelt es sich um eine
Abfragesprache, welche durch das \ac{ogc} zum internationalen Standard erhoben wurde und von allen
konformen \textit{\ac{ogc} Catalogue Services} untersützt wird. \citepalias{ogcCatalogue2016} In
einem neuen Entwurf \citepalias{ogcFiltering} wird eine Variation von \ac{cql} eingeführt: Simple
\ac{cql}. Sie wird dazu benutzt um Filterbedingungen für Abfragen an \ac{ogc} API-Dienste zu
erstellen und existiert in zwei Formaten: \textit{CQL\_TEXT} und \textit{CQL\_JSON}. \citepalias{ogcFiltering} Im
Weiteren wird sich auf Simple \ac{cql} bezogen.

Der Syntax von \ac{cql} basiert auf der \texttt{WHERE}-Klausel in \acs{sql}, und unterstützt
boolesche Prädikate, Textvergleiche, zeitliche Datentypen und raumbezogene Operatoren. Die Sprache
kann um Prädikate, Operatoren und Datentypen erweitert werden.
\citepalias{ogcCatalogue2016}

Als typischer Anwendungsfall für \ac{cql} dienen Geoinformationssysteme, mit denen sich die \ac{ogc}
befasst. Daher sind im Gegensatz zu \ac{sql} geometrische Operatoren wie \texttt{EQUALS},
\texttt{DISJOINT}, \texttt{INTERSECT} oder \texttt{CONTAINS} bereits Teil der Sprache. Die in
\ac{cql} vorhandenen Datentypen umfassen sowohl primitive Datentypen, als auch zeitliche und
raumbezogene Datentypen.
\citepalias{ogcCatalogue2016}

= Funktionen
- Selbstauskunft über /functions
- einfache Operatoren, spatial and temporal functions
- information über inputs gegeben, output typ wird von /functions bestimmt

= Verwendung zum Erstellen von Nicht-Filter Ausdrücken
- Filter bestehen aus komplexen Ausdrücken
- Kombination aus Funktionen kann Transformationen an Queryables darstellen
