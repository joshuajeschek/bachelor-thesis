\section{\acl{cql}}

Bei der \acf{cql} \footnote{Nicht zu verwechseln mit der \textit{Contextual} Query Language, welche
zuerst auch als Common Query Language bezeichnet wurde.
\parencite{thelibraryofcongressCQLContextual2023, ZINGGentle2003}} handelt es sich um eine
Abfragesprache, welche durch den Standard \citetitle{opengeospatialconsortiumOGCAPI} beschrieben
wird. Zum Zeitpunkt des Verfassens dieser Arbeit handelt es sich dabei noch um einen Entwurf.
\shorthandparencite{opengeospatialconsortiumOGCAPI}

ACHTUNG eigentlich \shorthandtextcite*{opengeospatialconsortiumOGCCatalogue2016}

= Anwendungsfall
- GIS
- Filtern
- GIS, dadurch Augenmerk auf spacial und temporal functions

= Datentypen und Typsystem
- primitive datentypen
- spatial und temporal datentypen

= Funktionen
- Selbstauskunft über /functions
- einfache Operatoren, spatial and temporal functions
- information über inputs gegeben, output typ wird von /functions bestimmt

= Verwendung zum Erstellen von Nicht-Filter Ausdrücken
- Filter bestehen aus komplexen Ausdrücken
- Kombination aus Funktionen kann Transformationen an Queryables darstellen
