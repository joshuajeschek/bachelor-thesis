\newcommand{\SimplexData}{\textit{Simplex4Data}}
\section{Simplex4Data}
% \parencite{rudolfUmweltdatenmanagementGeoInspiration2018}
% \parencite{rudolfUmweltdatenIntelligenz2021}
% \parencite{grossmannEnvVisioUniverselle2021}
% \parencite{grossmannEnvVisioService2022}
% \parencite{grossmannFachsystemeSchemaevolution2024}

Bei \SimplexData{} handelt es sich um ein Data Warehouse, welches den von \textcite{rudolfUmweltdatenmanagementGeoInspiration2018} für Umweltdaten vorgestellten Modellierungsansatz umsetzt \parencite{grossmannFachsystemeSchemaevolution2024}. Die allgemeine Struktur des Systems wird in Abbildung \ref{fig:s4d} dargestellt. Dabei besteht der Kern der Datenhaltungskomponente aus simplen Strukturen, die objektorientiert Datensätze abbilden. Datenobjekte werden harmonisiert und atomar verwaltet. Dieses Modell wird unter dem Name Realitätsmodell in \ref{sec:simplex-reality} vorgestellt. \SimplexData{} bildet ein Sekundärsystem, in das Daten aus Fachapplikationen eingespielt werden. Dieser Schritt wird Import genannt und in \ref{sec:simplex-importer} erläutert. Durch SimplexSzenarios können die gesammelten Datenobjekte ausgewertet und zu fachspezifischen Sichten zusammengestellt werden (\ref{sec:simplex-scenarios}). Sowohl die Daten im Realitätsmodell, als auch die SimplexSzenarios werden über den SimplexService als \acs{API}-Schnittstelle ausgegeben (\ref{sec:simplex-service}).

\begin{figure}
  \tikzstyle{s4dArrow}=[->, line width=1.5mm, s4d-blue]
  \resizebox{.95\textwidth}{!}{%
    \begin{tikzpicture}
      \node (s4d) at (8, -6) [inner sep=0, outer sep=0]
      {\includegraphics[height=4cm]{assets/s4d.png}};
      \coordinate (btm) at ($(s4d.south west)!0.58!(s4d.south east)$);
      \node (s4dT) at (s4d.north)
      [regular polygon, regular polygon sides=3, minimum size=5.35cm,
        anchor=south, rotate around={180:(s4d.north)}]
      {};
      \node (s4dR) at (btm)
      [regular polygon, regular polygon sides=3, minimum size=4.5cm,
        anchor=corner 3, rotate around={304:(btm)}]
      {};
      \node (s4dL) at (btm)
      [regular polygon, regular polygon sides=3, minimum size=4.5cm,
        anchor=corner 2, rotate around={64:(btm)}]
      {};

      \node (s4dImport) at ($(s4dL.center) - (8,0)$) [align=center]
      {\huge Import};

      \node (importCircle) at (s4dImport.center)
      [shape=circle, minimum width=7cm]
      {};

      \node (i1) at (importCircle.140) {\Large{}.shp};
      \node (i2) at (importCircle.160) {\Large{}.json};
      \node (i3) at (importCircle.180) {\Large{}.csv};
      \node (i4) at (importCircle.200) {\Large{}API};
      \node (i5) at (importCircle.220) {\Large{}\dots};

      \draw [s4dArrow, line width=.5mm]
      (i1) edge (s4dImport)
      (i2) edge (s4dImport)
      (i3) edge (s4dImport)
      (i4) edge (s4dImport)
      (i5) edge (s4dImport);

      \node (s4dScenarios) at ($(s4dR.center) + (8,0)$) [align=center]
      {\huge Simplex\\\huge Szenarios};

      \draw [s4dArrow]
      (s4dImport) edge node (importArrow) {} (s4dL)
      (s4dR) edge node (scenarioArrow) {} (s4dScenarios);

      \node (s4dReality) at (s4dT.south) [anchor=south, align=center]
      {\huge Realitäts-\\\huge modell};

      \node (block1) at ($(importArrow.center) - (0,2)$) [align=center]
      {\Large Konvertierung zu\\\Large Realitätsmodell};

      \node (block2) at ($(scenarioArrow.center) - (0,2)$) [align=center]
      {\Large Zusammenstellung von\\\Large Anwendungsfällen};

      \node (s4dService) at (s4dScenarios |- s4dReality.base)
      [anchor=base, align=center]
      {\huge Simplex\\\huge Service};

      \draw[s4dArrow]
      (s4dReality) edge (s4dService)  (s4dScenarios) edge (s4dService);

    \end{tikzpicture}
  }
  \caption{Datenverarbeitung in \SimplexData{}. Aus verschiedenen Formaten werden Daten in das Realitätsmodell importiert, und dort harmonisiert verwaltet. Durch SimplexSzenarios können sie ausgewertet, und mit dem SimplexService ausgelesen werden.}
  \label{fig:s4d}
\end{figure}
