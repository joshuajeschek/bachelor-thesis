\subsection{Funktionen in \ac{cql}}

\citetitle{ogcFiltering} sieht Funktionen als ein Weg vor, um \ac{cql} zu erweitern. Die Grammatik
für Funktionen in \ac{cql} werden durch die \ac{bnf} in Quelltext \ref{lst:function} definiert.
Zu sehen ist, dass die Funktionen ähnlich wie in \ac{sql} oder verschiedenen Programmiersprachen
funktionieren\todo{Formulierung}. Eine Funktion wird durch einen Funktionsname (\textit{identifier})
identifiziert, und kann eine Liste an Argumenten annehmen. Bei diesen Argumenten kann es sich um
Literale, arithmetische und boolsche Ausdrücke, Arrays, Attribute\todo{mit Queryables in
Verbindung setzen} oder weitere Funktionsaufrufe handeln.

\lstinputlisting[
  label=lst:function,
  caption={\ac{bnf} für \ac{cql}-Funktionen laut \textcite{ogcFiltering}},
  captionpos=b,
  firstnumber=351,
  firstline=351,
  lastline=363,
  language=bnf
]{assets/cql2.bnf}
\todo{filename, margin}

Durch den rekursiven Charakter und den Zugriff auf Attribute aus Datensätzen lassen sich beliebig
komplexe\todo{Beweis?} Ausdrücke erstellen, welche zur Transformation und Auswertung von Daten
benutzt werden kann.

Vorraussetzung dafür ist jedoch, dass alle benötigten Funktionen zur Verfügung stehen. Der
vorläufige \ac{api}-Standard sieht vor, dass ein spezifischer Endpunkt über die auf dem Server
verfügbaren Funktionen informiert. Unter \texttt{/functions} soll eine Liste von Funktionen
zurückgegeben werden. Dabei soll jede Funktion über ihren Name, den Rückgabetyp und die möglichen
Typen der Argumente Auskunft geben. Eine beispielhafte Antwort auf eine Anfrage vom
\texttt{/functions}-Endpunkt wird in Quelltext \ref{lst:functions} dargestellt. Dabei wird die
Funktion \texttt{st\_makepoint} aus \textit{PostGIS} und \texttt{date\_trunc} von \ac{sql}
beschrieben. Alle Argumente der beiden Funktionen nehmen nur einen Typ an - \citetitle{ogcFiltering}
sieht jedoch auch die Möglichkeit vor, dass Argumente mehrere Datentypen annehmen können.

\lstinputlisting[
  label=lst:functions,
  caption={Antwort auf \texttt{/functions}-Anfrage (Auszug)},
  captionpos=b,
  language=json,
  escapeinside=``
]{assets/functions.json}
\todo{filename, margin}
