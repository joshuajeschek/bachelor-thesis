\subsection{Personas}
\todo{find more sources? basically only citing \textcite{tomlinUXOptimization2018}}

Das Konzept von Personas wurde zuerst von \textcite{cooperInmatesAre1999} eingeführt. Er erstellte sie, um eine fiktive Person als eine Zusammenfassung von gemeinsamen Bedürfnissen, Hintergründen und Vorwissen zu erstellen und somit zu definieren, was eine "typische" Nutzer:in tun müsste, um Erfolg mit einer Anwendung zu haben. Auch er gab dieser Persona bereits eine Hintergrundgeschichte, Ziele und Gründe zum Benutzen der Applikation. Sie vereinfachen das Design und die Entwicklung indem sie den Fokus von den Funktionalitäten auf die Endnutzer:innen lenken und die Frage stellen, was sie benötigen, um erfolgreich zu sein. Somit sind Personas ein wichtiges Werkzeug des \textit{"user-centered design"}.
\parencite{cooperInmatesAre1999, tomlinUXOptimization2018}

Für Usability Testing stellen Personas eine wichtige Rolle dar, da sie helfen, die richtigen Teilnehmer:innen zu finden.
\parencite{tomlinUXOptimization2018}

Laut \textcite{tomlinUXOptimization2018} werden gute Personas durch mehrere Merkmale geprägt. Sie basieren auf Feldforschung, bei der Informationen von potenziellen Nutzer:innen gesammelt werden, um gemeinsame Verhaltensmuster zu identifizieren. Personas sollten sich auf den gegenwärtigen Zustand fokussieren und die typische Umgebung bei der Verwendung der Anwendung beinhalten, sowie die benutzten Geräte. Des Weiteren werden ein Bild, Name und eine kurze Geschichte dazu verwendet, die beschriebene Person zu vermenschlichen. Dabei sollte ein typisches Problem oder eine Aufgabe beschrieben werden, die es gilt zu überwinden. Die zwei oder drei wichtigsten Aufgaben können auch in einer Reihenfolge priorisiert werden. Zu guter Letzt betont \textcite{tomlinUXOptimization2018}, dass gute Personas auch Details über den typischen Grad der Expertise beinhalten sollten. Diese Informationen sollten detailliert auf die Anwendung und die Problemstellung angepasst werden.
\parencite{tomlinUXOptimization2018}

\subsubsection{Persona-Arten} \todo{besserer Abschnittsname?}

\textcite{tomlinUXOptimization2018} identfiziert drei verschiedene Arten von Personas: Design und Marketing Personas, sowie Proto-Personas. Im Folgenden sollen diese vorgestellt werden.

\textbf{Design Personas} stellen die Nutzer:innen, ihre Ziele und Aufgaben in den Mittelpunkt. Laut \textcite{tomlinUXOptimization2018} ist es am Wichtigsten, dass Design Personas auf Feldforschung und Beobachtung von Nutzer:innen in ihrer eigenen Umgebung basieren. Diese Personas stellen Kontext für Designentscheidungen dar und können den Einfluss von Einzelpersonen einschränken, indem sie eine fundierte Basis bilden. \todo{Formulierung} Da sie sich auf wichtige Aufgaben konzentrieren, kann durch Design Personas der Entwicklungsprozess optimiert und in die richtige Richtung gelenkt werden. Das betrifft sowohl Designentscheidungen, als auch Usability-Studien. Abschließend ist anzumerken, dass eine gute Persona dieser Art ein wichtiges Werkzeug für \textit{user-centered design} darstellt.
\parencite{tomlinUXOptimization2018}

\textbf{Marketing Personas} unterscheiden sich von Design Personas, da sie den Fokus auf das Kaufverhalten, Meinungen und Einstellungen der potentiellen der potentiellen Kundschaft legen. Daher beinhalten sie typischerweise nicht die wichtigsten Aufgaben, sondern Informationen über Kaufmotivationen und -bedürfnisse. Anstatt von Feldforschung werden oft quantitative Datensätze basierend auf Primär- oder Sekundärforschung verwendet. \textcite{tomlinUXOptimization2018} beschreibt dass es selten möglich ist, Marketing Personas anstelle von Design Personas zu benutzen, da sie wichtige Aufgaben beschreiben und auf Feldforschung basieren sollten.
\parencite{tomlinUXOptimization2018}

\textbf{Proto-Personas} werden von \textcite{tomlinUXOptimization2018} als am Weitesten verbreitet angesehen, da sie kostengünstig sind und nicht viel Zeit benötigt wird, um sie zu erstellen. Proto-Personas basieren oft auf Sekundärforschung, oder sogar nur auf der Intuition der Designer:innen. Obwohl die Verwendung einer Proto-Persona einem Vorgehen ohne Personas vorzuziehen ist, ist es empfehlenswert, sie im Laufe der Zeit zu validieren und zu verbessern. Sie können somit einen guten Startpunkt für agile Prozesse in Design und Entwicklung darstellen und nach einiger Zeit, mit gesammelten Erkenntnissen, zu Design Personas aufgewertet werden.
\parencite{tomlinUXOptimization2018}

Im Rahmen dieser Arbeit sind \textbf{Design Personas} und \textbf{Proto-Personas} von Interesse.

\todo{Überlegen ob dem Persona Abschnitt nochwas fehlt}
\todo{Oder auch, ob nach Personas noch ein Abschnitt fehlt (vor Aufgaben und Scenarios)}
