\subsection{Erste Eindrücke}
\label{sec:impressions}

Zunächst wurde den Teilnehmer:innen der Ausgangspunkt des ersten Szenarios präsentiert. Diese Ansicht wurde von mehreren als übersichtlich sowie aufgeräumt beschrieben. In diesem Kontext wurde bereits einmal lobend erwähnt, dass die Quell- und Zielstrukturen gleichzeitig zu sehen sind. Nur wenige Teilnehmer:innen fingen in dieser Phase bereits an zu klicken, meist wurde nur gescrollt und geschaut.

Auf den Titel der Seite ("CQL Buffet") wurde drei Mal eingegangen. Einmal wurde erkannt, dass es sich hierbei um den aktuellen Name der Anwendung handelt, in den anderen zwei Fällen wünschten sich die Teilnehmer:innen einen konkreteren Name, der besser zum Inhalt passt, und beschreibt was auf der Seite passiert, beziehungsweise durchgeführt werden kann.

Auf die Daten des ersten Szenarios, welche in dieser explorativen Phase bereits sichtbar waren, wurde nur wenig eingegangen. Einmal wurde angemerkt, dass die Abkürzungen unbekannt sind. Dies schien dazu beizutragen, dass die Person sich weniger gut in der Anwendung zurechtgefunden hat.

\subsubsection{Benennung von Bereichen}

Bereits ohne die Anwendung zu bedienen, konnte der Großteil die Hauptbereiche korrekt identifizieren. Sechs Teilnehmer:innen konnten den linken Bereich konkret als Quelldaten und den rechten Bereich als Zielstruktur benennen. Diese Erkenntnis wurde von zwei Personen damit begründet, dass die Attribute der Objektklasse (rechts) mit typischen Schlüsselwörtern benannt worden sind (\texttt{key}, \texttt{cmt}, \dots).

Im Gegensatz dazu war es zwei Personen nicht möglich, die Bereiche zu benennen. Zum Einen wurde darauf gehofft, dass sich dies im ersten Szenario ändert, sobald mehr über die zu absolvierende Aufgabe bekannt ist. Zum Anderen konnte kein Unterschied zwischen links und rechts erkannt werden, da die enthaltenen Daten sehr ähnlich aussehen. Das kann der Aufgabenstellung und dem Fakt, dass die Quelldaten nicht bekannt sind, zugeordnet werden, da im ersten Szenario die Spalten der Quelltabelle sehr den Attributen der Objektklasse ähneln \footnote{Die Benennung der Spalten der Quelltabelle wurde so gewählt, damit Nutzer:innen sich auf die Anwendung konzentrieren können, und sich nicht mit unbekannten Quelldaten befassen müssen.}.

Die restlichen Personen (zwei) konnten zwar die unterschiedlichen Bereiche identifizieren, diese aber nicht konkret benennen. Es wurde erkannt, dass im rechten Bereich Eingaben getätigt werden können, und dass links etwas damit zu tun haben muss. In diesem Zusammenhang wurde erkannt, dass die Elemente im linken Bereich anklickbar sind. Es wurde auch vermutet, dass sie per \textit{Drag and Drop} benutzt werden können, was allerdings in der aktuellen Version nicht möglich ist. Das scrollende Menü im linken Bereich wurde positiv eingeschätzt, da so nicht alles auf einmal angezeigt wird, was schnell überwältigend werden könnte.

\subsubsection{Funktionen}

Falls Teilnehmer:innen auf die im linken Menü verfügbaren Funktionen eingegangen sind, waren die Meinungen dazu unterschiedlich. Einige konnten damit noch nichts anfangen, andere überlegten aber bereits, was sie mit ihnen machen könnten. Mehrfach wurde erwähnt, dass die Funktionen für Berechnungen benutzt werden könnten, einmal wurde explizit die Möglichkeit von Typumwandlungen (\textit{type casting}) angesprochen. Eine Person erkannte auch, dass es sich um Funktionen aus SQL, beziehungsweise der PostGIS-Erweiterung für PostgreSQL handelt.

\subsubsection{Symbole}

In Einzelfällen wurde bereits auf die Symbole eingegangen, die Datentypen darstellen, und dies wurde auch in zwei Fällen konkret so erkannt. Eine Person drückte bereits Verwirrung über das Symbol für Wahrheitswerte (\textit{Booleans}) aus, und dachte es würde sich um bedienbare Schalter handeln würde. In diesem Fall wurde nicht erkannt dass es sich um das den Datentyp beschreibende Symbol handelt.

\subsubsection{Automatische Typ-Filterung}

Es wurde von zwei Teilnehmer:innen beobachtet, dass sich die Auswahl auf der linken Seite reduziert, sobald auf der rechten Seite ein Feld ausgewählt wird. Jedoch wurde nur in einem Fall ein Rückschluss auf die Datentypen gemacht. Die Funktionsweise war somit dem Großteil noch nicht klar, was aber auch daran lag dass nur die Wenigsten die Oberfläche schon aktiv bedient haben.

\subsubsection{Wenig beachtete Funktionen}

Eine Funktionalität, auf die nur zwei Personen eingegangen sind, ist die Möglichkeit, die Elemente im Auswahlmenü über die zwei Buttons im oberen Bereich zu filtern. Durch Ausprobieren wurde herausgefunden, wie diese Buttons den Inhalt filtern. Die meisten Teilnehmer:innen sind jedoch gar nicht auf die Buttons eingegangen, und waren sich anscheinend auch nicht über die Möglichkeit des Filterns in diesem Bereich im Klaren. Dieser Trend setzte sich während den Szenarios weiterhin fort.

Ein:e Teilnehmer:in griff auf das Konfigurationsmenü im Header-Bereich zu, was allerdings nur URLs zum SimplexService (Queryables, Funktionen, etc. - Vgl. \ref{section:ogc}) enthält, und somit den Prototyp für die Szenarios vorbereitet. In diesem Fall wurde darauf hingewiesen, dass dieses Menü keine Relevanz für den Usability-Test hat, und im finalen Produkt nicht enthalten sein wird.

\subsubsection{Verbesserungsvorschläge}

In der ersten Phase wurden bereits zwei Verbesserungsvorschläge ausgedrückt. Einerseits wurde der Titel der Seite kritisiert, dieser sollte den Inhalt der Anwendung besser wiederspiegeln. Des Weiteren wurde vorgeschlagen, im Auswahlmenü besser darzustellen, wie die Überschriften ("Abfragbare Felder" und "Funktionen") funktionieren. Durch Klicken auf diese wird die Ansicht auf den Anfang der dazugehörigen Liste eingestellt, was zum Beispiel durch einen Pfeil markiert werden könnte, wie eine Person vorschlug.
