\newcommand{\rfqsfn}{\footnote{\texttt{/rec/filter/queryables-schema}
(\url{https://portal.ogc.org/files/96288\#rec_filter_queryables-schema})}\todo[color=yellow]{OGC
scheint anfällig dafür zu sein, Links nicht beizubehalten (1. weil Entwurf, zweitens verlinkt
"external identifier" in \citetitle{ogcFiltering} auf 404) - wäre es dann sinnvoll, den aktuellen
Stand von relevanten Teilen in den Anhang zu packen?}}
\newcommand{\rfgqog}{\footnote{\texttt{/req/filter/get-queryables-op-global}
(\url{https://portal.ogc.org/files/96288\#req_filter_get-queryables-op-global})}}
\newcommand{\rfgqol}{\footnote{\texttt{/req/filter/get-queryables-op-local}
(\url{https://portal.ogc.org/files/96288\#req_filter_get-queryables-op-local})}}

\subsection{Queryables}

\citetitle{ogcFiltering} beschreibt \textit{Queryables} als Token, welche ein eine Eigenschaft einer
Ressource darstellen, die in einem Filterausdruck verwendet werden können
\parencitealias{ogcFiltering}. Der vorläufige Standard sieht vor, dass ein \ac{api}-Endpunkt
angeboten wird, der über die existierenden Queryables, deren Namen und Typen informiert.
\parencitealias{ogcFiltering}

Dabei existieren zwei Pfade. Unter \texttt{/queryables} werden die Attribute zurückgegeben , die für
alle \textit{collections} existieren\rfgqog. Jede \textit{collection} verfügt des Weiteren über einen
Unterpfad \texttt{/collections/\{collectionId\}/queryables}, welcher die spezifischen Attribute für
die \textit{collection} angibt\rfgqol.
\parencitealias{ogcFiltering}

Der vorläufige Standard schlägt vor\rfqsfn, dass jedes ausgelieferte Attribut über einen Titel und,
wenn nötig, über eine Beschreibung verfügt. Des Weiteren sollte jedes Attribut einen Typ angeben,
außer es handelt sich um einen räumlichen Typ. Diese werden mittels einer Referenzierung des
passenden \ac{json}-Schema des Geometrietyps gekennzeichnet. Zeitliche Typen sollen als
\texttt{string} beschrieben werden und mit einem entsprechenden Format spezifiziert. Des Weiteren
wird definiert, wie mit Aufzählungen und Wertebereichen umgegangen werden sollte.
\parencitealias{ogcFiltering}

\subsubsection{Queryables im SimplexService}

\lstinputlisting[
  label=lst:queryables,
  caption={Antwort auf \texttt{/queryables}-Anfrage für \textit{collection} "Gemeinden" (Auszug)},
  language=json,
  escapeinside=``
]{assets/queryables.json}

\todo[inline]{Standardfall für Queryables in SimplexService erklären (Szenario 1, Objektklassen,
Felder, Attribute)}

\todo[inline]{Auf Quelltext \ref{lst:queryables} eingehen}

\todo[inline]{Gejointe Queryables beschreiben}
