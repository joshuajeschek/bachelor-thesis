\subsection{Studienablauf}

Die Usability-Studie fand in zehn Einzelsitzungen statt. Dabei wurde zuerst die Motivation für die Usability-Studie erklärt, und eine Einordnung in das restliche System Simplex4Data gegeben. Daraufhin wurde das Vorgehen erklärt, insbesondere wurden die Teilnehmer:innen mit den Konzepten der Szenarios (im Kontext von Usability-Testing, siehe \ref{sec:scenarios}), und dem \ac{CTA} vertraut gemacht. Des Weiteren wurde getestet, ob das Teilen des Bildschirms funktioniert.

Dies stellt auch einen guten Moment dar, um sich auf die zu testende Oberfläche zu beziehen. Die Teilnehmer:innen wurden zuerst gebeten, den ersten Eindruck zu beschreiben, und Bereiche im Block-Editor zu benennen. Somit konnte auch gut an das \ac{CTA} herangeführt werden.

Den Kernteil der Testsitzungen bestand aus drei Szenarios, die absolviert wurden. Deren Inhalt dieser ist in Abschnitt \ref{sec:study-szenarios} genauer beschrieben. Nach jedem Szenario wurde die Einfachheit der Bewältigung des Szenarios auf einer Skala von 1 bis 7 erfragt (\ac{SEQ}). Des Weiteren bestand die Möglichkeit tiefgehender auf Probleme bei der Benutzung einzugehen, oder den Teilnehmer:innen Feedback zum \ac{CTA} zu geben.

Nach dem Absolvieren der Szenarios wurden mehrere inhaltliche Fragen gestellt, sowie eine Bewertung der Einfachheit der Benutzung des Block-Editors auf einer Skala von 1 bis 10 erfragt. Den Teilnehmer:innen wurde außerdem die Möglichkeit gegeben, die aufgetretenen Usability Probleme zu priorisieren und somit den Verlauf der Entwicklung zu beeinflussen.

Im Vorfeld der Studie wurden zwei Dokumente erstellt. Beim ersten handelt es sich um ein Skript für die Moderation (Anhang \ref{app:moderation}). Darin ist die genaue Struktur der Testsitzungen definiert. Des Weiteren wurde ein Informationsblatt für die Testteilnehmer:innen vorbereitet, in dem die wichtigsten Eckdaten und Links zu den einzelnen Szenarios, gesammelt aufgelistet sind (Anhang \ref{app:handout}).
