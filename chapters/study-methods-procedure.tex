\subsection{Ablauf der Studie}

Die Usability-Studie fand in zehn Einzelsitzungen statt. Dabei wurde zuerst die Motivation für die Usability-Studie erklärt, und eine Einordnung in das restliche System Simplex4TwIS gegeben. Daraufhin wurde das Vorgehen beschrieben. Insbesondere wurden die Teilnehmer:innen mit den Konzepten der Szenarios (im Kontext von Usability-Testing, siehe \ref{sec:scenarios}), und dem \ac{CTA} vertraut gemacht. Des Weiteren wurde getestet, ob das Teilen des Bildschirms funktioniert.

Dies stellte auch einen guten Moment dar, um sich auf die zu testende Oberfläche zu beziehen. Die Teilnehmer:innen wurden zunächst gebeten, den ersten Eindruck zu beschreiben, und Bereiche im Block-Editor zu benennen. Auf diese Art und Weise konnten sich die Teilnehmer:innen auch mit dem \ac{CTA} vertraut machen.

Der Hauptteil der Testsitzungen bestand aus drei Szenarios, die nacheinander absolviert wurden. Deren Inhalt dieser ist Abschnitt \ref{sec:study-szenarios} genauer beschrieben. Nach jedem Szenario wurde die Einfachheit der Bewältigung des Szenarios erfragt (\ac{SEQ}). Des Weiteren bestand die Möglichkeit tiefgehender auf Probleme bei der Benutzung einzugehen, oder den Teilnehmer:innen Feedback zum \ac{CTA} zu geben.

Nach dem Absolvieren der Szenarios wurden mehrere inhaltliche Fragen gestellt, sowie eine Bewertung der Einfachheit der Benutzung des Block-Editors erfragt. Den Teilnehmer:innen wurde außerdem die Möglichkeit gegeben, die aufgetretenen Usability Probleme zu priorisieren und somit den Verlauf der Entwicklung zu beeinflussen.

Im Vorfeld der Studie wurden zwei Dokumente erstellt. Beim Ersten handelt es sich um ein Skript für die Moderation (Anhang \ref{app:moderation}). Darin ist die genaue Struktur der Testsitzungen definiert. Des Weiteren wurde ein Informationsblatt für die Testteilnehmer:innen vorbereitet, in dem die wichtigsten Eckdaten und Links zu den einzelnen Szenarios, gesammelt aufgelistet sind (Anhang \ref{app:handout}).
