\subsection{Relevanz im Rahmen dieser Arbeit}

Als Teil dieser Arbeit wurde ein Block-Editor entwickelt, der einige Eigenschaften mit den vorgestellten Konzepten teilt. Wie in \ref{sec:home-assistant} gezeigt, bietet Home Assistant eine Oberfläche, mit der der maschinenlesbare \ac{YAML}-Vorschriften erzeugt werden. Es wird eine Low-Code-Oberfläche eingesetzt, die es ermöglicht Blöcke miteinander zu kombinieren, und die Konfiguration über speziell angepasste Eingabefelder durchzuführen. Ähnliche Konzepte können für den Block-Editor zum Auswählen von Feldern und Feldtransformationen genutzt werden, da aus dieser Anwendung \ac{CQL}-Vorschriften generiert werden sollen, und die Datentypen der Felder oft bekannt sind. Dadurch können auch in diesem Fall die Eingabefelder auf den Typ der gewünschten Eingabe angepasst werden.

Die Programmierumgebung \Scratch{} und das davon inspirierte \Snap{} stellen bekannte Block-Editoren dar. Mit \DataSnap{} wurde eine Erweiterung zu \Snap{} vorgestellt, die das Ziel, Fachexpert:innen zu unterstützen, mit der vorliegenden Bachelorarbeit teilt. \DataSnap{} setzt zwar einige Funktionen um, die bereits in anderen Teilen von Simplex4TwIS gelöst wurden (namentlich Import und Visualisierung), kann im Kern jedoch auch genutzt werden, um Feldtransformationen zu definieren. Auch die Möglichkeit, in \Snap{} wiederverwendbare Funktionen zu definieren, könnte von Interesse sein.
