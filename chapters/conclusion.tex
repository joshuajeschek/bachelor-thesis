Im Rahmen dieser Bachelorarbeit wurde eine Erweiterung für das Data Warehouse Simplex4TwIS entwickelt, was die Konvertierung von Daten ohne die Benutzung von \acs{SQL} ermöglichen sollte. Um Tippfehler und häufiges Nachschlagen zu vermeiden, sowie die Anwendung auch für weniger technisch erfahrene Menschen zugänglich zu machen, wurde ein blockbasierter Ansatz gewählt. Dieser Ansatz wurde mittels einer formativen Usability-Studie validiert.

\pskip
Der entwickelte Block-Editor gliedert sich in die Anwendung Simplex4TwIS ein, und ermöglicht es mehrere Schritte der Datenhaltung durchzuführen. Zum Einen können mit ihm Fachdaten aus spezifischen Formaten in das harmonisierte und objektorientierte Format des Realitätsmodells überführt werden. Zum Anderen ist es möglich, den Editor zu nutzen, um aus mehreren Objektklassen Auswertungen in Form von SimplexSzenarios zu erstellen. Die Erweiterung für Simplex4TwIS wurde so konzipiert, dass die Ausgangsdaten, bei denen es sich entweder um Importtabellen oder Objektklassen handelt, auf der linken Seite in einem Auswahlmenü aufgelistet werden. In diesem Auswahlmenü befinden sich auch Funktionen, die genutzt werden können um Attribute zu konvertieren und miteinander zu kombinieren. Die Elemente aus dem Auswahlmenü lassen sich im rechten Bereich in die Zielstruktur integrieren, welche entweder durch die Definition von Objektklassen vorgegeben ist, oder frei durch die Nutzer:innen editierbar ist. Symbole informieren über die Datentypen der Felder und eintragbaren Elemente. Außerdem wird eine automatische Filterfunktion eingesetzt, die die Auswahlmöglichkeiten auf passende Elemente reduziert, sobald ein Feld angeklickt wurde. Komplexe Operationen können mit Hilfe des Editors definiert werden, indem Blöcke miteinander kombiniert werden. Der Editor produziert \ac{CQL}-Befehle, die in der restlichen Anwendung benutzt werden können und verhindert durch seine Funktionsweise außerdem \ac{SQL}-Injektion.

Der Prototyp beinhaltet alle für die Usability-Studie benötigten Funktionalitäten. Um ihn in der Praxis einzusetzen, muss die Entwicklung weiter vorangetrieben werden. Die Verwendung von Aggregatfunktionen sollte unterstützt werden und der Umfang des Funktionskatalogs sollte erhöht werden. Außerdem sollte anhand von komplexen Praxisbeispielen getestet werden, inwiefern der Editor diese unterstützt, oder ob weiterhin ein Modus für Expert:innen genutzt werden sollte, in dem es möglich ist, \ac{SQL} ohne Beschränkungen zu verwenden.

\pskip
Es wurde eine formative Usability-Studie mit zehn Teilnehmer:innen durchgeführt, wobei Testsitzungen meist 90 Minuten dauerten und das \acf{CTA} angewendet wurde. Es nahmen Angestellte und Kund:innen von Simplex4Data teil, die zu den typischen Nutzer:innen von Simplex4TwIS gehören. Im Zuge der Studie konnte festgestellt werden, dass der blockbasierte Ansatz gut angenommen wurde. Die Teilnehmer:innen konnten mit Hilfe des Editors Umweltdatensätze bearbeiten und gaben an, sich dabei produktiver zu fühlen. Der Block-Editor wurde mit der aktuell für Konvertierungen eingesetzten Lösung verglichen, und wurde durchgehend als einfacher in der Benutzung eingeschätzt. Während den Testsitzungen konnten durch \ac{CTA} und fehlerhafte Bedienung 20 Usability-Probleme aufgedeckt werden, von denen die am häufigsten aufgetretenen hier erwähnt werden sollen. Darunter fielen Schwierigkeiten bei der Interaktion mit dem Editor, beispielsweise die Bedienreihenfolge von Funktionen, und die Art und Weise, wie der Datentyp von Feldern bei der Erstellung von SimplexSzenarios ausgewählt wird. Außerdem waren einige Informationen missverständlich dargestellt. Kritisiert wurde die Wahl der Symbole für Datentypen, sowie die Darstellung von Attributen in der Zielstruktur.

\pskip
Die Integration des Block-Editors in Simplex4TwIS erfordert eine Fortführung der Entwicklung sowie die Klärung einiger offener Fragen. Zuerst können im Rahmen der Studie aufgetretene Probleme gelöst werden. Die gewählten Symbole für Datentypen sollten angepasst werden und die Hauptbereiche der Oberfläche besser benannt werden, um die Orientierung zu verbessern. Außerdem kann die oft gewünschte Möglichkeit, Elemente zu ersetzen, implementiert werden. Sowohl in den Konvertierungen, als auch bei der Erstellung von SimplexSzenarios sollte auf eine klare Darstellung der abfragbaren Felder geachtet werden. Dazu gehört die Nutzung eines Präfixes bei der zentralen Klasse, als auch eine gleichmäßige Darstellung im Auswahlmenü und in der Zielstruktur.

Eine zentrale Rolle in der Weiterentwicklung spielt eine Vereinheitlichung der Arbeitsrichtung. Im Prototyp ist es möglich, zuerst das Zielfeld oder zuerst ein Element aus dem Auswahlmenü auszuwählen. Beide Varianten wurden im Laufe der Studie genutzt, der Funktionsumfang der Methoden unterscheidet sich jedoch. In zukünftigen Versionen des Editors sollte dies behoben werden, beispielsweise durch eine Neuanordnung des Auswahlmenüs und den Verzicht auf eine Arbeitsrichtung.

Vor der Inbetriebnahme des Block-Editors im Praxisbetrieb ist es erforderlich, den Funktionskatalog zu vervollständigen. Im Anschluss ist eine Überprüfung des Auswahlmenüs hinsichtlich seiner Übersichtlichkeit sowie gegebenenfalls eine weitere Sortierung oder Kategorisierung erforderlich. Des Weiteren sollte geklärt werden, ob Operatoren in Infixnotation unterstützt werden sollten, und inwiefern dies umgesetzt werden sollte. Außerdem kann die Implementation von dynamischen Feldern untersucht werden, die es ermöglichen mehrere oder alle Datentypen einzufügen. Dies könnte sowohl für Felder von SimplexSzenarios vorteilhaft sein, als auch für Funktionsparameter, die mehrere Datentypen entgegennehmen.

\pskip
Die Benutzung des Block-Editors stellt sowohl eine Vereinfachung für Fachexpert:innen ohne technische Vorkenntnisse dar, als auch eine Bereicherung im Arbeitsablauf von Nutzer:innen mit \ac{SQL}-Erfahrung. Perspektivisch könnte es sogar möglich sein, ihn in Open Data Portalen zu nutzen, um es Laien zu ermöglichen, individuelle Auswertungen zu erstellen.
