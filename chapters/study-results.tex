\section{Ergebnisse und Analyse}

\todo[inline]{einleitende Worte}

Das \ac{CTA} wurde in den Testsitzungen auf zufriedenstellende Art und Weise durchgeführt. Einige Teilnehmer:innen stachen heraus, indem sie besonders viele ihrer Gedanken teilten und aktiv Lösungsvorschläge präsentierten. Anderen Testpersonen fiel es schwerer, kontinuierlich ihre Gedanken zu teilen und müssten häufiger ermuntert werden. In diesen Fällen wurde sich während der Testsitzung auf minimale Fragen beschränkt, die den Redefluss anregen sollten. Falls auf Schlüsselelemente während der Benutzung nicht eingegangen wurde, wurden im Nachhinein Fragen dazu gestellt. Dies war jedoch nur in wenigen Fällen notwendig, da die Handlungsabsicht meist klar war, und mit bekannten Problemen übereinstimmte.

Ein Unterschied zwischen internen und externen Testpersonen konnte nur sehr begrenzt festgestellt werden. Das zweite Szenario wurde von Mitarbeiter:innen als einfacher bewertet, wobei die meist größere \ac{SQL}- und Programmiererfahrung eine Rolle spielt. Die Vorversion des Block-Editors zur Bearbeitung von Konvertierungen wurde durch interne Testpersonen besser als von externen bewertet. Eine Ausnahme stellten Mitarbeiter:innen dar, die weniger Vorerfahrung mit dem Systemansatz haben. Die meisten Unterschiede zwischen den Einschätzungen der Testpersonen können auf unterschiedliche  Erfahrungsgrade zurückgeführt werden.

In Abschnitt \ref{sec:impressions} erfolgt zunächst eine Zusammenfassung der in den Vorgesprächen gewonnenen ersten Eindrücke der Teilnehmer:innen. Im Anschluss werden in Abschnitt \ref{sec:qualitative} die Erkenntnisse präsentiert, die im Laufe der Szenarios gewonnen wurden. Abschnitt \ref{sec:quantitative} widmet sich schließlich der Darstellung der quantitativen Resultate.
