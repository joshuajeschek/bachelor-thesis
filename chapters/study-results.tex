\section{Ergebnisse und Analyse}

\todo[inline]{wie wurde \acl{CTA} angenommen? (Teil von Einleitung)}
\todo[inline]{schauen, ob unterschied zwischen internen und externen, wenn nein auch sagen}

\begin{figure}[!ht]
  \pgfplotstableread[col sep=comma]{assets/study-results.csv}\datatable
  \centering
  \begin{tikzpicture}
    \begin{axis}[
        boxplot/draw direction=y,
        boxplot/every median/.style={ultra thick},
        xtick={1,2,3},
        xticklabels={Szenario 1, Szenario 2, Szenario 3},
        ytick={1,...,7},
        ymin=1,
        ylabel={\acl{SEQ}},
      ]
      \addplot+ [boxplot, plot1] table [y=a] {\datatable};
      \addplot+ [boxplot, plot2] table [y=b] {\datatable};
      \addplot+ [boxplot, plot3] table [y=c] {\datatable};
    \end{axis}
  \end{tikzpicture}
  \caption{Antworten auf die \acs{SEQ} nach jedem Szenario}
\end{figure}

\todo[inline]{Eingehen auf spezifisches Feedback zu den Szenarios als SEQ abgefragt wurde}

\begin{figure}[!ht]
  \pgfplotstableread[col sep=comma]{assets/study-results.csv}\datatable
  \centering
  \begin{tikzpicture}
    \begin{axis}[
        boxplot/draw direction=y,
        boxplot/every median/.style={ultra thick},
        xtick={1,2},
        xticklabels={Vorversion, Block-Editor},
        ytick={1,...,10},
        ymin=1,
        ylabel={Einfachheit der Benutzung},
      ]
      \addplot+ [boxplot, plot1] table [y=prev] {\datatable};
      \addplot+ [boxplot, plot2] table [y=ges] {\datatable};
    \end{axis}
  \end{tikzpicture}
  \caption{Einschätzung der Einfachheit der Benutzung nach dem Usability
    Test auf einer Skala von 1 (sehr schwer) bis 10 (sehr einfach)}
\end{figure}

\colorlet{technical}{plot1}
\colorlet{presentation}{plot2}
\colorlet{interaction}{plot3}
\colorlet{content}{plot4}

\begin{figure}[!ht]
  \centering
  \begin{tikzpicture}
    \begin{axis}[
        xbar=0pt,
        xmajorgrids=true,
        xtick={0,...,10},
        xmin=0,
        xmax=6,
        xlabel={Absolute Häufigkeit},
        /pgf/bar shift=0pt,
        legend style={legend cell align=left},
        legend pos=south east,
        axis y line*=none,
        axis x line*=bottom,
        tick label style={font=\footnotesize},
        legend style={font=\footnotesize},
        label style={font=\footnotesize},
        width=.6\textwidth,
        bar width=3.5mm,
        ymin=1,
        ytick={1,...,20},
        ytick style={draw=none},% <- added
        yticklabels={
            {Übersichtlichkeit Funktionsliste},
            {Präfix von Objektklassen},
            {Benennung Abfragbare Felder},
            {automatische Typfilterung},
            {Auswahl von Überladungen},
            {Metadaten von Szenario-Feldern},
            {Details zu Funktionen},
            {Zugriff auf die Quelldaten},
            {Angabe von statischen Werten},
            {Probleme mit Scrollen},
            {Drag \& Drop},
            {Ersetzen von Einträgen},
            {Benennung mittlerer Bereich},
            {Datentyp von Szenario-Feldern},
            {Benennung des Speichern-Buttons},
            {Icons für Datentypen},
            {Details zu Parametern},
            {Anzeige von Attributen in Ziel},
            {Parameterübernahme bei Überladungen},
            {Bedienreihenfolge von Funktionen},
          },
        area legend,
        y=6mm,
        enlarge y limits={abs=0.625},
        every axis plot/.append style={fill}
      ]
      \addplot[presentation] coordinates {(0,0)};
      \addplot[content] coordinates {(0,0)};
      \addplot[interaction] coordinates {(0,0)};
      \addplot[technical] coordinates {(0,0)};
      \addlegendentry{Darstellung}
      \addlegendentry{Inhalt}
      \addlegendentry{Interaktion}
      \addlegendentry{Technisch}
      \addplot[presentation] coordinates {(1,1)};  % Übersichtlichkeit Funktionsliste
      \addplot[content]      coordinates {(2,2)};  % Präfix von Objektklassen
      \addplot[presentation] coordinates {(2,3)};  % Benennung Abfragbare Felder
      \addplot[interaction]  coordinates {(2,4)};  % automatisches Typfilterung
      \addplot[interaction]  coordinates {(2,5)};  % Auswahl von Überladungen
      \addplot[presentation] coordinates {(2,6)};  % Metadaten von Szenario-Feldern
      \addplot[content]      coordinates {(2,7)};  % Details zu Funktionen
      \addplot[content]      coordinates {(3,8)};  % Zugriff auf die Quelldaten
      \addplot[interaction]  coordinates {(3,9)};  % Angabe von statischen Werten
      \addplot[technical]    coordinates {(3,10)}; % Probleme mit Scrollen
      \addplot[interaction]  coordinates {(4,11)}; % Drag\&Drop
      \addplot[interaction]  coordinates {(4,12)}; % Ersetzen von Einträgen
      \addplot[presentation] coordinates {(4,13)}; % Benennung mittlerer Bereich
      \addplot[interaction]  coordinates {(4,14)}; % Datentyp von Szenario-Feldern
      \addplot[presentation] coordinates {(4,15)}; % Benennung des Speichern-Buttons
      \addplot[presentation] coordinates {(5,16)}; % Icons für Datentypen
      \addplot[content]      coordinates {(5,17)}; % Details zu Parametern
      \addplot[presentation] coordinates {(5,18)}; % Anzeige von Attributen in Ziel
      \addplot[interaction]  coordinates {(5,19)}; % Parameterübernahme bei Überladungen
      \addplot[interaction]  coordinates {(6,20)}; % Bedienreihenfolge von Funktionen
    \end{axis}
  \end{tikzpicture}
  \caption{Häufigkeit des Auftretens verschiedener Probleme während der Usability-Studie. Gezählt wird die Anzahl der Testsitzungen, in der das jeweilige Problem aufgetaucht ist.}
  \label{figure:problems}
\end{figure}
