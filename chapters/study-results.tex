\section{Erkenntnisse}
\label{sec:results}

Im Folgenden wird die Usability-Studie ausgewertet. Darunter fallen sowohl die durch \acs{CTA} gewonnenen Erkenntnisse, als auch die Ergebnisse von Befragungen qualitativer und quantitativer Art.

Das \ac{CTA} wurde in den Testsitzungen auf zufriedenstellende Art und Weise durchgeführt. Einige Teilnehmer:innen stachen heraus, indem sie ihre Gedanken besonders oft teilten und aktiv Lösungsvorschläge präsentierten. Anderen Testpersonen fiel es schwerer, kontinuierlich ihre Gedanken zu teilen und mussten häufiger ermuntert werden. In diesen Fällen wurde sich während der Testsitzung auf minimale Fragen beschränkt, die den Redefluss anregen sollten. Falls auf Schlüsselelemente während der Benutzung nicht eingegangen wurde, wurden im Nachhinein Fragen dazu gestellt. Dies war jedoch nur in Einzelfällen notwendig und nie für komplette Testsitzungen der Fall. Die Handlungsabsicht war meist klar und konnte mit bekannten Probleme abgeglichen werden.

In Abschnitt \ref{sec:impressions} erfolgt zunächst eine Zusammenfassung der ersten Eindrücke der Teilnehmer:innen, die in den Vorgesprächen gewonnen wurden. Im Anschluss werden in Abschnitt \ref{sec:qualitative} die Erkenntnisse präsentiert, die im Laufe der Szenarios gewonnen wurden. Abschnitt \ref{sec:quantitative} widmet sich schließlich der Darstellung der quantitativen Resultate.
