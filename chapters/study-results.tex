\section{Ergebnisse und Analyse}

\begin{figure}
  \label{fig:seq}
  \pgfplotstableread[col sep=comma]{assets/study-results.csv}\datatable
  \begin{tikzpicture}
    \begin{axis}[
        ybar=2\pgflinewidth,
        xlabel={Usability Test},
        xtick=data,
        xticklabels from table={\datatable}{n},
        major x tick style=transparent,
        ytick={1,2,...,7},
        ymin=1,
        ylabel={\acl{seq}},
        ymajorgrids,
        width=\textwidth,
        bar width=8pt,
        legend cell align=left,
      ]

      \addplot[plot1, fill=plot1] table [y=a, x expr=\coordindex] {\datatable};
      \addplot[plot2, fill=plot2] table [y=b, x expr=\coordindex] {\datatable};
      \addplot[plot3, fill=plot3] table [y=c, x expr=\coordindex] {\datatable};

      \legend{Szenario 1, Szenario 2, Szenario 3}

    \end{axis}
  \end{tikzpicture}
  \caption{Antworten auf die \ac{seq} nach jedem Szenario}
\end{figure}

\begin{figure}
  \label{fig:sus}
  \pgfplotstableread[col sep=comma]{assets/study-results.csv}\datatable
  \begin{tikzpicture}
    \begin{axis}[
        ybar=2\pgflinewidth,
        xlabel={Usability Test},
        xtick=data,
        xticklabels from table={\datatable}{n},
        major x tick style=transparent,
        ytick={1,2,...,10},
        ymin=1,
        ylabel={Einfachheit der Benutzung},
        ymajorgrids,
        width=\textwidth,
        bar width=12pt,
        legend cell align=left,
      ]

      \addplot[plot1, fill=plot1] table [y=prev, x expr=\coordindex] {\datatable};
      \addplot[plot2, fill=plot2] table [y=ges, x expr=\coordindex] {\datatable};

      \legend{Vorversion, Block-Editor}

    \end{axis}
  \end{tikzpicture}
  \caption{Einschätzung der Einfachheit der Benutzung nach dem Usability Test}
\end{figure}
