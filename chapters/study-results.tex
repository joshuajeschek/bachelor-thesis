\section{Ergebnisse und Analyse}

\todo[inline]{wie wurde \acl{CTA} angenommen?}
\todo[inline]{schauen, ob unterschied zwischen internen und externen, wenn nein auch sagen}

\begin{figure}[!ht]
  \pgfplotstableread[col sep=comma]{assets/study-results.csv}\datatable
  \centering
  \begin{tikzpicture}
    \begin{axis}[
        boxplot/draw direction=y,
        boxplot/every median/.style={ultra thick},
        xtick={1,2,3},
        xticklabels={Szenario 1, Szenario 2, Szenario 3},
        ytick={1,...,7},
        ymin=1,
        ylabel={\acl{SEQ}},
      ]
      \addplot+ [boxplot, plot1] table [y=a] {\datatable};
      \addplot+ [boxplot, plot2] table [y=b] {\datatable};
      \addplot+ [boxplot, plot3] table [y=c] {\datatable};
    \end{axis}
  \end{tikzpicture}
  \caption{Antworten auf die \acs{SEQ} nach jedem Szenario}
\end{figure}

\begin{figure}[!ht]
  \pgfplotstableread[col sep=comma]{assets/study-results.csv}\datatable
  \centering
  \begin{tikzpicture}
    \begin{axis}[
        boxplot/draw direction=y,
        boxplot/every median/.style={ultra thick},
        xtick={1,2},
        xticklabels={Vorversion, Block-Editor},
        ytick={1,...,10},
        ymin=1,
        ylabel={Einfachheit der Benutzung},
      ]
      \addplot+ [boxplot, plot1] table [y=prev] {\datatable};
      \addplot+ [boxplot, plot2] table [y=ges] {\datatable};
    \end{axis}
  \end{tikzpicture}
  \caption{Einschätzung der Einfachheit der Benutzung nach dem Usability
  Test auf einer Skala von 1 (sehr schwer) bis 10 (sehr einfach)}
\end{figure}
