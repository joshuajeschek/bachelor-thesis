\subsection{Feedback}
Feedback-Methoden können abhängig vom Typ der Studie variieren. Formative Studien profitieren von
qualitativen Methoden, während formative Studien quantitatives Feedback einsetzen. Laut
\textcite{barnumUsabilityTesting2021} kann in beiden Fällen eine Benutzung der jeweils anderen
Herangehensweise die Erkenntnisse erweitern. Wie bereits in \ref{section:formative-summative}
festgestellt, ist es in zeitigeren Entwicklungsstadien vorteilhafter, formative Studien
durchzuführen. \todo{Überprüfen ob a) stimmig und b) so beschrieben} Qualitative Aussagen von
Teilnehmer:innen können wertvolle Einblicke in die größten Problembereiche geben und auf die
nächsten Entwicklungsschritte hindeuten.
\parencite{barnumUsabilityTesting2021}

\textcite{barnumUsabilityTesting2021} nennt Leistungsdaten und Präferenzinformationen als
quantitative Maße für Usability. Als Leistungsdaten gelten zum Beispiel die Zeit zum Absolvieren von
Aufgaben, Fehlerquote oder Erfolgsrate. Wie bereits in \ref{section:scenarios} angeschnitten,
spricht eine hohe Erfolgsrate jedoch nur von einem Mindestmaß an Usability und sollte durch weitere
Werte unterstützt werden.
