\subsection{Feedback}
\label{sec:feedback}

Feedback-Methoden können abhängig vom Typ der Studie variieren. Formative Studien profitieren von qualitativen Methoden, während summative Studien quantitatives Feedback einsetzen. Laut \textcite{barnumUsabilityTesting2021} kann in beiden Fällen eine Benutzung der jeweils anderen Herangehensweise die Erkenntnisse erweitern. Wie in \ref{sec:formative-summative} festgestellt, ist es in früheren Entwicklungsstadien vorteilhafter, formative Studien durchzuführen. Qualitative Aussagen von Teilnehmer:innen können wertvolle Einblicke in die größten Problembereiche geben und auf die nächsten Entwicklungsschritte hindeuten \parencite{barnumUsabilityTesting2021}.

\textcite{barnumUsabilityTesting2021} nennt Leistungsdaten (\textit{performance data}) und Präferenzinformationen (\textit{preference data}) als quantitative Maße für Usability. Als Leistungsdaten gelten zum Beispiel die Zeit zum Absolvieren von Aufgaben, Fehlerquote oder Erfolgsrate. Wie bereits in \ref{sec:scenarios} angeschnitten, spricht eine hohe Erfolgsrate jedoch nur von einem Mindestmaß an Usability und sollte durch weitere Werte unterstützt werden. Des Weiteren weist \textcite{barnumUsabilityTesting2021} darauf hin, dass auch unterschiedliche Grade von Erfolg beim Absolvieren von Aufgaben bestehen. So könnte eine Nutzer:in den schnellsten, vom Design vorgesehenen Weg zum Ziel benutzen, oder aber auch einen indirekten Pfad nehmen. Auch verschiedene in Anspruch genommene Hilfestellungen können von Usability Problemen sprechen \parencite{barnumUsabilityTesting2021}. Das Bewerten einer Aufgabe als Misserfolg kann auch verschiedene Gründe haben: Aufgeben, Abruch durch die Testleitung, oder die Annahme, dass die Aufgabe beendet sei, ohne dass sie das wirklich ist \parencite{barnumUsabilityTesting2021}. Eine weitere Metrik die einfach zu erheben ist, ist die Zeit zum Vollenden der Aufgaben. In der ersten Studie können diese Werte als Ausgangsbasis gesammelt werden, und in zukünftigen Studien zum Vergleich benutzt werden, so \textcite{barnumUsabilityTesting2021}. Zu beachten ist, dass diese einfach zu erhebenden Metriken nicht die gesamte Usability Erfahrung beschreiben können \parencite{barnumUsabilityTesting2021}.

Als Präferenzinformationen beschreibt \textcite{barnumUsabilityTesting2021} Antworten auf Fragebögen, welche nach Aufgaben und nach dem kompletten Test erhoben werden. Befinden sie sich auf Skalen (1 bis 5 oder 1 bis 10), können sie als quantitative Daten benutzt werden. Fragen mit offenem Ende liefern qualitative Informationen über die Erlebnisse der Teilnehmer:innen \parencite{barnumUsabilityTesting2021}. Zusätzlich sollten auch über Thinking-Aloud gewonnene Eindrücke berücksichtigt werden und, wenn verfügbar, nonverbales Feedback wie Körpersprache oder nonverbale Ausdrücke gesammelt werden \parencite{barnumUsabilityTesting2021}.

\subsubsection{Fragebögen}

Nach jedem Szenario Feedback über die Aufgabe zu sammeln, hat laut \textcite{barnumUsabilityTesting2021} den Vorteil, dass die Erinnerungen noch frisch sind. Dabei kann es sich auch um eine einzige Frage handeln. \textcite{sauroIfYou2010} nennt als Eigenschaften eines guten Fragebogens, dass er zuverlässig, sensibel, valide und kurz ist, sowie einfach zu beantworten, zu handhaben und zu bewerten sein sollte. \textcite{barnumUsabilityTesting2021} schlägt vor eins oder mehr der folgenden Themen abzufragen: Schwierigkeit der Durchführung, benötigte Zeit (von "weniger als erwartet" bis "mehr als erwartet"), die Wahrscheinlichkeit, dass dieses Feature erneut benutzt wird, und das Vertrauen in die erfolgreiche Bewältigung der Aufgabe.

\textcite{sauroIfYou2010} listet folgende Standard-Fragebögen für das Einholen von Feedback nach Aufgaben auf: \ac{ASQ}, \ac{NASA-TLX}, \ac{SMEQ}, \ac{UME} und \ac{SEQ}.

\vspace{\baselineskip}

Der \textbf{\ac{ASQ}} wurde von \textcite{lewisPsychometricEvaluation1991} eingeführt und beinhaltet drei Fragen bezüglich Schwierigkeit der Aufgabe, Bearbeitungszeit und der Menge der unterstützenden Informationen (Dokumentation, Online-Hilfe, etc.). Die Antworten werden auf einer Skala von 1 (stimme voll zu) bis 7 (stimme überhaupt nicht zu) angegeben.

\vspace{\baselineskip}

Der \textbf{\ac{NASA-TLX}} wurde von \textcite{hartDevelopmentNASATLX1988} entwickelt und berechnet eine Gesamtbewertung der Arbeitsbelastung basierend auf, geistiger, physischer und zeitlicher Anforderung, sowie Leistung, Aufwand und Frustration. \parencite{nasaNASATLX}

\vspace{\baselineskip}

Der \textbf{\ac{SMEQ}}, zuerst von \textcite{zijlstraConstructionScale1985} beschrieben, besteht aus einer einzelnen Skala von "gar nicht anstrengend" bis "extrem anstrengend". 

\vspace{\baselineskip}

Bei \textbf{\ac{UME}} handelt es sich um eine Methode, bei der nach jeder Aufgabe ein numerischer Wert von den Teilnehmer:innen erfragt wird, welcher sich auf die Aufgabe bezieht. Die erste Antwort kann arbiträr sein, die restlichen orientieren sich dann aber an den vorherigen Antworten - somit können die Aufgaben untereinander verglichen werden. Zu betonen ist, dass die Skala im Vorhinein nicht festgelegt ist, und von den Teilnehmer:innne selbst gewählt wird \parencite{mcgeeUsabilityMagnitude2003}.

\vspace{\baselineskip}

Die \textbf{\ac{SEQ}} besteht aus einer einzigen Frage: "Insgesamt war diese Aufgabe...", wobei von einer Skala von 1 (sehr schwierig) bis 5 (sehr einfach) ausgewählt werden kann. \textcite{tedescoComparisonMethods2006} stellte fest, dass diese Methode bei einer kleinen Stichprobengröße die beständigsten Ergebnisse liefert \footnote{Es fand ein Vergleich mit 4 anderen Methoden statt: \ac{ASQ}, \ac{UME}, eine Variation von \ac{SEQ} und die Erwartungs-Bewertung von \textcite{albertThisWhat2003}}. \textcite{sauroComparisonThree2009} verglich die \ac{SEQ} mit \ac{SMEQ} und \ac{UME}. Dabei punktete \ac{SEQ} mit einfacher Bedienbarkeit und Erlernbarkeit, während alle Methoden eine ähnliche Zuverlässigkeit aufwiesen.

\textcite{barnumUsabilityTesting2021} erklärt, dass Fragebögen nach Szenarios an die soeben absolvierte Aufgabe angepasst werden können. Während das gleiche für Fragebögen nach dem gesamten Test gilt, existieren auch da Standard-Fragebögen, wie \ac{SUS}, \ac{CSUQ} und \ac{NPS}.

\vspace{\baselineskip}

\textbf{\ac{SUS}} wurde ursprünglich 1986 vorgestellt und besteht aus 10 Fragen, die auf einer Likert-Skala bewertet werden \parencite{brookeSUSQuick1996}. Um den Gesamtwert-Wert zu berechnen, werden die einzelnen Antworten miteinander addiert und mit 2,5 multipliziert, um einen \ac{SUS}-Wert zwischen 0 und 100 zu erhalten. \textcite{sauroMeasuringUsability2011} führte eine Meta-Analyse von 500 Studien durch und errechnete einen durchschnittlichen \ac{SUS}-Wert von 68. Dieser Werte könnte laut \textcite{barnumUsabilityTesting2021} als Basiswert für iterative Studien benutzt werden.

\textcite{brookeSUSQuick1996} beschreibt folgende Vorgehensweise zur Anwendung von \ac{SUS}: Die Fragen sollten direkt nach dem Test gestellt werden, bevor etwaige andere Aktivitäten durchgeführt werden. Des Weiteren sollte Teilnehmer:innen ihre direkte Antwort geben, statt zu viel darüber nachzudenken. Schlussendlich betont \textcite{brookeSUSQuick1996}, dass optimalerweise eine Antwort auf jede Frage gegeben wird - im Falle von Meinungslosigkeit kann das der Mittelpunkt sein. Laut \textcite{barnumUsabilityTesting2021} ist die \ac{SUS} sehr weit verbreitet, da sie schnell anzuwenden ist, unabhäng von der Technologie des getesteten Produktes ist und sich über die Zeit als valide Methode für Studien ab fünf Teilnehmer:innen erwiesen hat.

\vspace{\baselineskip}

\textbf{\ac{CSUQ}} ist ähnlich zum \ac{PSSUQ}. Beide beinhalten 16 positive Fragen, welche auf einer Skala mit 7 Punkten bewertet werden \parencite{barnumUsabilityTesting2021}. Im Kontrast zu \ac{SUS} wird auch eine Bewertung von "nicht anwendbar" erlaubt \parencite{barnumUsabilityTesting2021}. Die Fragebögen ergeben einen Gesamtwert und drei Unterbewertungen von Systemqualität, Informationsqualität und Oberflächenqualität \parencite{lewisIBMComputer1995}.

\vspace{\baselineskip}

Beim \textbf{\ac{NPS}} handelt es sich um eine einzige Frage, die ungefähr wie folgt lautet: "Wie wahrscheinlich ist es, dass Sie [Unternehmen/Produkt] weiterempfehlen?". Dabei wird die Antwort auf einer Skala zwischen 0 und 10 gegeben. Teilnehmer:innen, die einen Wert unter 7 angeben, werden als \textit{Detractors} bezeichnet, bei einer Antwort über 8 werden sie als \textit{Promoters} eingestuft. Der \ac{NPS}-Wert wird aus einer Subtraktion des Prozentsatzes der \textit{Detractors} vom Prozentsatz der \textit{Promoters} gewonnen und kann somit zwischen -100 und +100 liegen. Aufgrund einer Untersuchung der Korrelation zwischen \ac{SUS} und \ac{NPS} von \textcite{sauroDoesBetter2010} stellt \textcite{barnumUsabilityTesting2021} fest, dass eine Daumenregel zum errechnen des \ac{NPS}-Wertes ist, den \ac{SUS}-Wert durch 10 zu teilen.
