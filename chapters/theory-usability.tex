\subsection{Usability}
\citefield{ISO924111}{shorttitle} definiert \textit{Usability} wie folgt:
\begin{quote}
  extent to which a system, product or service can be used by specified users to achieve specified goals with effectiveness, efficiency and satisfaction in a specified context of use
  \hspace*{\fill{}}\cite{ISO924111}
\end{quote}

Diese Definition ist kurz, umfasst jedoch trotzdem drei Hauptelemente, die \textit{Usability} ausmachen. Es sollte möglich sein, Aufgaben mit ausreichender Geschwindigkeit auszuführen (\textit{efficiency}) und das erwünschte Ziel zu erreichen (\textit{effectiveness}). Der dritte wichtige Faktor ist die Anwenderzufriedenheit (\textit{satisfaction}), welche mehr auf die subjektive Erfahrung von Nutzer*innen eingeht. So kann eine Aufgabe schnell und korrekt ausgeführt werden, aber dennoch nicht zu Zufriedenheit führen, da beispielsweise wichtige Informationen nicht angezeigt wurden, die über den Erfolg informieren. Der gleiche Effekt kann auch andersherum stattfinden - sollte die persönliche Einschätzung von Zufriedenheit sehr gut sein, können womögliche Probleme der Effizienz und Effektivität vernachlässigt werden, sodass das Produkt trotzdem als gut eingeschätzt wird. \cite{barnumUsabilityTesting2021}

Des Weiteren betont \citeauthor{barnumUsabilityTesting2021} \cite{barnumUsabilityTesting2021}, dass es sich um spezifische Nutzer*innen, Ziele und Nutzungskontexte handelt. Das bedeutet, dass \textit{Usability} nur für eine spezifische Gruppe an Menschen betrachtet werden kann, welche das Produkt benutzen werden und deren Ziele denen des Produkts entsprechen müssen. Außerdem kann die \textit{Usability} auch nur für den angedachten Nutzungskontext - die Umgebung in der ein Produkt genutzt wird - betrachtet werden.

\citeauthor{quesenberyDimensionsUsability2003} \cite{quesenberyDimensionsUsability2003} und \citeauthor{nielsenUsabilityEngineering1994} \cite{nielsenUsabilityEngineering1994} spezifizieren folgende 5 Attribute der \textit{Usability}:
\begin{itemize}
  \item \textbf{Effektivität}: wie umfassend und akkurat Aufgaben abgeschlossen werden,
  \item \textbf{Effizienz}: wie schnell dies erfolgt,
  \item \textbf{Ansprechende Gestaltung}: wie gut Interaktionen gesteuert werden und für Zufriedenheit gesorgt wird (von \citeauthor{nielsenUsabilityEngineering1994} vernachlässigt)
  \item \textbf{Fehlertoleranz}: wie das System Fehler verhindert und sich von Fehler erholt
  \item \textbf{Erlernbarkeit}: wie intuitiv das System ist, und wie Lernen im Laufe der Nutzung ermöglicht wird (von \citeauthor{nielsenUsabilityEngineering1994} in zwei Attribute geteilt)
\end{itemize}

Laut \citeauthor{barnumUsabilityTesting2021} \cite{barnumUsabilityTesting2021} hat der Faktor der Zufriedenheit in den letzten Jahren an Bedeutung zugenommen. Das wird auch davon unterstützt, dass \citeauthor{nielsenUsabilityEngineering1994} (\citeyear{nielsenUsabilityEngineering1994}) weniger Wert auf eine ansprechende Gestaltung legte als \citeauthor{quesenberyDimensionsUsability2003} (\citeyear{quesenberyDimensionsUsability2003}).
