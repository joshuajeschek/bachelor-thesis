\section{Usability Testing}

Um auf \textit{Usability Testing} einzugehen, lohnt es sich, einen genaueren Blick auf
\textit{Usability} zu werfen. \citefield{ISO924111}{shorttitle} definiert \textit{Usability} wie
folgt:
\begin{quote}
  extent to which a system, product or service can be used by specified users to achieve specified
  goals with effectiveness, efficiency and satisfaction in a specified context of use
  \hspace*{\fill{}}\shorthandtextcite*{ISO924111}
\end{quote}
Diese Definition ist kurz, umfasst jedoch trotzdem drei Hauptelemente, die \textit{Usability}
ausmachen. Es sollte möglich sein, Aufgaben mit ausreichender Geschwindigkeit auszuführen
(\textit{efficiency}) und das erwünschte Ziel zu erreichen (\textit{effectiveness}). Der dritte
wichtige Faktor ist die Anwenderzufriedenheit (\textit{satisfaction}), welche mehr auf die
subjektive Erfahrung von Nutzer*innen eingeht. So kann eine Aufgabe schnell und korrekt ausgeführt
werden, aber dennoch nicht zu Zufriedenheit führen, da beispielsweise wichtige Informationen nicht
angezeigt wurden, die über den Erfolg informieren. Der gleiche Effekt kann auch andersherum
stattfinden - sollte die persönliche Einschätzung von Zufriedenheit sehr gut sein, können womögliche
Probleme der Effizienz und Effektivität vernachlässigt werden, sodass das Produkt trotzdem als gut
eingeschätzt wird. \parencite{barnumUsabilityTesting2021}

Des Weiteren betont \citeauthor{barnumUsabilityTesting2021} \cite{barnumUsabilityTesting2021}, dass
es sich um spezifische Nutzer*innen, Ziele und Nutzungskontexte handelt. Das bedeutet, dass
\textit{Usability} nur für eine spezifische Gruppe an Menschen betrachtet werden kann, welche das
Produkt benutzen werden und deren Ziele denen des Produkts entsprechen müssen. Außerdem kann die
\textit{Usability} auch nur für den angedachten Nutzungskontext - die Umgebung in der ein Produkt
genutzt wird - betrachtet werden. \parencite{barnumUsabilityTesting2021}

\textcite{quesenberyDimensionsUsability2003} und \textcite{nielsenUsabilityEngineering1994}
spezifizieren folgende 5 Attribute der \textit{Usability}:
\begin{itemize}
  \item \textbf{Effektivität}: wie umfassend und akkurat Aufgaben abgeschlossen werden,
  \item \textbf{Effizienz}: wie schnell dies erfolgt,
  \item \textbf{Ansprechende Gestaltung}: wie gut Interaktionen gesteuert werden und für
    Zufriedenheit gesorgt wird (von \textcite{nielsenUsabilityEngineering1994} vernachlässigt)
  \item \textbf{Fehlertoleranz}: wie das System Fehler verhindert und sich von Fehler erholt
  \item \textbf{Erlernbarkeit}: wie intuitiv das System ist, und wie Lernen im Laufe der Nutzung
    ermöglicht wird (von \textcite{nielsenUsabilityEngineering1994} in zwei Attribute geteilt)
\end{itemize}
Laut \textcite{barnumUsabilityTesting2021} hat der Faktor der Zufriedenheit in den letzten Jahren an
Bedeutung zugenommen. Das wird auch davon unterstützt, dass
\textcite{nielsenUsabilityEngineering1994} weniger Wert auf eine ansprechende Gestaltung legte als
\textcite{quesenberyDimensionsUsability2003}.

\newpage

\textit{Usability Testing} ist ein Sammelbegriff für Methoden bei denen Benutzer*innen mit einem
System interagieren, um ein Ziel zu erreichen oder Szenario umzusetzen. Dabei sind die Umstände der
Interaktionen kontrolliert und Verhaltensdaten werden gesammelt
\parencite{wichanskyUsabilityTesting2000}. Ziel ist es, die \textit{Usability} des Systems zu
überprüfen und eventuelle Probleme zu identifizieren.

Dabei ist es laut \textcite{barnumUsabilityTesting2021} wichtig, dass die Nutzer*innen Aufgaben
erfüllen, die echt sind und eine Bedeutung für sie beinhalten. So werden Probleme, die bei der
Nutzung in der realen Welt auftreten effizienter aufgedeckt.

Wichtige Aspekte des \textit{Usability Testing} werden im weiteren Verlaufe des dieses Kapitels
beleuchtet.
