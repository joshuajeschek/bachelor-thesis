\section{Methodik}

Als Teil dieser Bachelorarbeit wurde eine Usability-Studie durchgeführt. Ziel derer war es, zeitig während der Entwicklung eines Editors für Konvertierungen und Szenarien festzustellen, ob sich die dafür gewählte Herangehensweise eignet. Usability-Probleme sollten zeitig identifiziert werden, um die weitere Entwicklung des Blockeditors zu beeinflussen. Abgesehen davon wurde auch erhoben, ob sich das Einfachheit der Benutzung im Vergleich zur Vorversion verbessert hat, und ob die Nutzer:innen das Gefühl haben, effizienter zu sein.

Durchgeführt wurde also eine formative Studie mit zehn Testpersonen (Vgl. \ref{sec:formative-summative}). Jede Testsitzung hatte einen Umfang von ca. 90 Minuten, in denen Datensätze des Themenbereichs Umwelt bearbeitet wurden, welche sonst auch typischerweise in Simplex4Data verarbeitet werden. Die Testsitzungen fanden in einer Online-Videokonferenz statt, wobei die Teilnehmer:innen ihren Bildschirm teilten.

Außerdem kam das \acf{CTA} zur Anwendung (Vgl. \ref{sec:think-aloud}). Dabei wurde versucht, so wenig wie möglich in das Testgeschehen einzugreifen, und nur minimale, das Teilen von Gedanken anregende, Aussagen zu tätigen.
