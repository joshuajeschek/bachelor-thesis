\section{Methodik der Usability-Studie}
\label{sec:study-methods}

Als Teil dieser Bachelorarbeit wurde eine Usability-Studie durchgeführt. Ziel derer war es, frühzeitig während der Entwicklung eines Editors für Konvertierungen und SimplexSzenarios festzustellen, ob die dafür gewählte Herangehensweise geeignet ist. Usability-Probleme sollten zeitig identifiziert werden, um die weitere Entwicklung des Block-Editors anzupassen. Abgesehen davon wurde auch erhoben, ob sich die Einfachheit der Benutzung im Vergleich zur Vorversion verbessert hat, und ob die Nutzer:innen das Gefühl haben, effizienter zu sein.

Durchgeführt wurde eine formative Studie mit zehn Testpersonen. Jede Testsitzung hatte einen Umfang von ca. 90 Minuten, in denen Datensätze des Themenbereichs Umwelt bearbeitet wurden, welche typischerweise in Simplex4TwIS verarbeitet werden. Die Testsitzungen fanden in einer Online-Videokonferenz statt, wobei die Teilnehmer:innen ihren Bildschirm teilten.

Außerdem kam das \acf{CTA} zur Anwendung (Vgl. \ref{sec:think-aloud}). Dabei wurde versucht, so wenig wie möglich in das Testgeschehen einzugreifen, und nur minimale Aussagen getätigt, die das Teilen von Gedanken anregen sollten.
