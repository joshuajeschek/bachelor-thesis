Im Rahmen der Digitalisierung und Open Data Intitiativen sollen Umweltdaten der Öffentlichkeit zugänglich gemacht werden. Die für dieses Ziel nötigen Prozesse erweißen sich jedoch als zeit- und kostenintensiv, und auf lange Zeit nicht tragfähig.

Aktuell ist es im Bereich Umweltdaten üblich, dass mehrere Fachsysteme mit eigenen Modellen existieren, die erhobene Daten verwalten und miteinander verknüpfen \parencite{grossmannEnvVisioUniverselle2021}. Außerdem müssen verschiedene Bereitstellungsformate bedient werden, wie beispielsweise die 34 von INSPIRE\footnote{\textcite{inspireRichtlinie20072007}} definierten Themenbereiche. Da diese oft Daten aus mehreren Fachsystemen beinhalten, sind komplizierte Datentransformationen nötig, die oft ausgeführt werden müssen \parencite{grossmannEnvVisioUniverselle2021}. Diese aufwendigen Prozesse benötigen viel Zeit und Geld, und können meist nicht von Fachexpert:innen durchgeführt werden, denen die technische Expertise fehlt.

Das Data Warehouse Simplex4TwIS geht diese Probleme an, indem es die Datenhaltung stark vereinfacht. Umweltdaten werden in einem homogenen, harmonisierten Format gespeichert. Die genutzten einfachen, sich wiederholenden Strukturen ermöglichen eine vereinfachte Aggregation von Datensätzen und Erstellung von Auswertungen. Die Konvertierung der Daten kann dadurch zwar teilweise automatisiert werden, es besteht jedoch weiterhin die Notwendigkeit anzugeben, auf welche Art und Weise die Daten in ein harmonisiertes Format überführt werden können und ausgewertet werden sollen. Dieser Prozess bleibt aufwendig und benötigt technische Expertise.

Da Simplex4TwIS dafür für das Harmonisieren und Auswerten von Daten weiterhin \acs{SQL}-Abfragen einsetzt, ist es für Fachexpert:innen ohne technische Kenntnisse nicht möglich, diese Aufgaben selbst zu bewältigen. Des Weiteren ist die manuelle Eingabe von \acs{SQL}-Abfragen mühsam, anfällig für Tippfehler und setzt voraus, dass die Benutzer:innen die Struktur der Eingabedaten kennen.

\pskip
Im Rahmen dieser Arbeit sollte eine Lösung enstehen, die die aktuellen Prozesse in Simplex4TwIS vereinfacht und für mehr Menschen zugänglich macht. Es wurde eine Erweiterung für Simplex4TwIS entwickelt, um Daten in das harmonisierte Datenmodell von Simplex4TwIS zu überführen, und aus gespeicherten Datensätzen neue Auswertungen zu erstellen. Der enstandene Editor verfolgt einen blockbasierten Ansatz, wodurch die eben genannten Probleme angegangen werden sollten.

Es wurde eine formative Usabilitystudie durchgeführt, um die Effektivität der blockbasierte Herangehensweise zu überprüfen, Usability-Probleme frühzeitig aufzudecken und die weitere Entwicklung des Block-Editors zu lenken. Für diesen Zweck wurde durch das \acf{CTA} und ein abschließende Befragung der Teilnehmer:innen festgestellt, welche Hürden während der Benutzung auftraten, und ob eine allgemeine Produktivitätssteigerung im Umgang mit den Umweltdaten empfunden wurde.

\pskip
Im ersten Teil dieser Arbeit wird ein Überblick über diverse theoretische Grundlagen gegeben, angefangen mit technischen Grundlagen. Zuerst wird auf das Data Warehouse Simplex4TwIS eingegangen und erläutert, wie Datenverwaltung, -transformationen und -auswertung in Simplex4TwIS durchgeführt wird. Danach wird die Funktionsweise des \textit{OGC API - Features}-Dienst erklärt, der als Grundlage für den Block-Editor dient. Im weiteren Verlauf des Kapitels werden Grundlagen des Usability Testings dokumentiert, die im praktischen Teil zur Anwendung kommen. Abschließend wird auf verwandte Arbeiten eingegangen, die relevant für die Entwicklung des Block-Editors sind, da sie ähnliche Probleme zu lösen versuchen oder eine vergleichbare Herangehensweise gewählt haben.

Der zweite Teil der Arbeit befasst sich mit der Entwicklung des Block-Editors und der durchgeführten Usability-Studie. Dabei werden zuerst die Ziele der Entwicklung, sowie das Ergebnis und bereits identifizierte Schwachstellen vorgestellt. Daraufhin wird die Methodik der Usability-Studie erläutert. In Abschnitt \ref{sec:results} werden die Erkenntnisse der Usability-Studie präsentiert. Im Anschluss werden Schlussfolgerungen gezogen, und eine Zukunftsaussicht über die weitere Entwicklung des Block-Editors gegeben.
