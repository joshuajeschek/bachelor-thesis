Open-Data-Initiativen und -Richtlinien zielen auf eine verbesserte Zugänglichkeit von Umweltdaten ab. Diese Daten sollten der breiten Bevölkerung flächendeckend und in einheitlichen Formaten zur Verfügung gestellt werden. Die Umsetzung dieser Wunschvorstellung erweist sich jedoch als schwierig, da die bestehenden Datenformate und die damit verbundenen Werkzeuge zu komplex sind und weder von Fachexpert:innen mit wenig technischer Expertise noch von Nutzer:innen von Open-Data-Portalen einfach bedient werden können.

In der Praxis ist es aktuell üblich, dass mehrere Fachsysteme mit eigenen Modellen existieren, die Umweltdaten erheben und speichern \parencite{grossmannEnvVisioUniverselle2021}. Gleichzeiteig müssen verschiedene Bereitstellungsformate bedient werden, wie beispielsweise die 34 von INSPIRE\footnote{\textcite{inspireRichtlinie20072007}} definierten Themenbereiche. Da diese oft Daten aus mehreren Fachsystemen beinhalten, sind komplizierte Datentransformationen nötig, die oft ausgeführt werden müssen \parencite{grossmannEnvVisioUniverselle2021}. Diese aufwendigen Prozesse benötigen viel Zeit und Geld, und können meist nicht von Fachexpert:innen durchgeführt werden, denen die technische Expertise fehlt.

\pskip
Das Data Warehouse Simplex4TwIS geht diese Probleme an, indem es die Datenhaltung stark vereinfacht. Umweltdaten werden in einem homogenen, harmonisierten Format gespeichert. Die genutzten einfachen, sich wiederholenden Strukturen ermöglichen eine vereinfachte Aggregation von Datensätzen und Erstellung von Auswertungen. Die Konvertierung der Daten kann dadurch zwar teilweise automatisiert werden, es besteht jedoch weiterhin die Notwendigkeit anzugeben, auf welche Art und Weise die Daten in ein harmonisiertes Format überführt werden können und ausgewertet werden sollen. Dieser Prozess bleibt aufwendig und benötigt technische Expertise.

Da Simplex4TwIS dafür für das Harmonisieren und Auswerten von Daten weiterhin \acs{SQL}-Abfragen einsetzt, ist es für Fachexpert:innen ohne technische Kenntnisse nicht möglich, diese Aufgaben selbst zu bewältigen. Des Weiteren ist die manuelle Eingabe von \acs{SQL}-Abfragen mühsam, anfällig für Tippfehler und setzt voraus, dass die Benutzer:innen die Struktur der Eingabedaten kennen.

\pskip
Im Rahmen dieser Arbeit wird eine Lösung vorgestellt, die die aktuellen Prozesse in Simplex4TwIS vereinfacht und für mehr Menschen zugänglich macht. Es wurde eine Erweiterung für Simplex4TwIS entwickelt, um Daten in das harmonisierte Datenmodell von Simplex4TwIS zu überführen, und aus gespeicherten Datensätzen neue Auswertungen zu erstellen. Der enstandene Editor verfolgt einen blockbasierten Ansatz, wodurch die eben genannten Probleme angegangen werden sollten.

Es wurde eine formative Usabilitystudie durchgeführt, um die Effektivität der blockbasierten Herangehensweise zu überprüfen, Usability-Probleme frühzeitig aufzudecken und die weitere Entwicklung des Block-Editors zu lenken. Für diesen Zweck wurde durch das \acf{CTA} und ein abschließende Befragung der Teilnehmer:innen festgestellt, welche Hürden während der Benutzung auftraten, und ob eine allgemeine Produktivitätssteigerung im Umgang mit den Umweltdaten empfunden wurde.

\pskip
In Kapitel \ref{sec:theory} wird ein Überblick über diverse theoretische Grundlagen gegeben, angefangen mit technischen Grundlagen. Zuerst wird auf das Data Warehouse Simplex4TwIS eingegangen. Da im Kern der Arbeit eine Erweiterung für Simplex4TwIS entstanden ist, wird die allgemeine Funktionsweise von Simplex4TwIS vorgestellt, und wie relevante Prozesse ablaufen, die durch den Editor verbessert werden sollen. Dazu gehört das Laden von Daten in das Realitätsmodell, sowie die Erstellung von Auswertung basierend auf vorhandenen Datensätzen. Danach wird die Funktionsweise des \textit{\acs{OGC} \acs{API} - Features}-Dienst erklärt, der als \acs{API}-Standard von Simplex4TwIS genutzt wird und als Grundlage für den Block-Editor dient. Im weiteren Verlauf von Kapitel \ref{sec:theory} werden Grundlagen des Usability Testings dokumentiert, die im praktischen Teil zur Anwendung kommen. Abschließend wird auf verwandte Arbeiten eingegangen, die relevant für die Entwicklung des Block-Editors sind, da sie ähnliche Probleme zu lösen versuchen oder eine vergleichbare Herangehensweise gewählt haben. Beleuchtet wird einerseits der blockbasierte Ansatz der Editoren in Home Assistant, andererseits die Programmierumgebungen Scratch, Snap! und DataSnap.

Kapitel \ref{sec:study} befasst sich mit der Entwicklung des Block-Editors und der durchgeführten Usability-Studie. Dabei wird zunächst auf die Ziele der Entwicklung eingegangen. Im Anschluss wird der entstandene Prototyp vorgestellt, auf die technische Umsetzung eingegangen und bekannte Schwachstellen identifiziert. Die Methodik der durchgeführten Usability-Studie wird in Abschnitt \ref{sec:study-methods} erläutert, während die aus ihr gewonnenen Erkenntnisse in Abschnitt \ref{sec:results} präsentiert werden. Im Detail wird auf die ersten Eindrücke der Teilnehmer:innen eingegangen, sowie eine Auswertung der qualitativen und quantitativen Erhebungen durchgeführt.

Die Erkenntnisse werden in Kapitel \ref{sec:discussion} diskutiert. Das darauffolgende Kapitel fasst die Arbeit zusammen und gibt eine Zukunftsaussicht über die weitere Entwicklung des Block-Editors.
