\subsection{\acl{CQL}}

\boxnote{%
  An sich definiert sowohl \citetitle{ogcCatalogue2016} als auch \citetitle{ogcFiltering} CQL.
  Zweiteres definiert den Kern als "Simple CQL".
}

Bei der \acf{CQL} \footnote{Nicht zu verwechseln mit der \textit{Contextual} Query Language, welche
  zuerst auch als Common Query Language bezeichnet wurde.
  \parencite{thelibraryofcongressCQLContextual2023, ZINGGentle2003}} handelt es sich um eine
Abfragesprache, welche durch das \ac{OGC} zum internationalen Standard erhoben wurde und von allen
konformen \textit{\ac{OGC} Catalogue Services} untersützt wird. \parencitealias{ogcCatalogue2016} In
einem neuen Entwurf \parencitealias{ogcFiltering} wird eine Variation von \ac{CQL} eingeführt: Simple
\ac{CQL}. Sie wird dazu benutzt um Filterbedingungen für Abfragen an \ac{OGC} API-Dienste zu
erstellen und existiert in zwei Formaten: \textit{CQL\_TEXT} und \textit{CQL\_JSON}. \parencitealias{ogcFiltering} Im
Weiteren wird sich auf Simple \ac{CQL} bezogen.

Der Syntax von \ac{CQL} basiert auf der \texttt{WHERE}-Klausel in \acs{SQL}, und unterstützt
boolesche Prädikate, Textvergleiche, zeitliche Datentypen und raumbezogene Operatoren. Die Sprache
kann um Prädikate, Operatoren und Datentypen erweitert werden.
\parencitealias{ogcCatalogue2016}

Als typischer Anwendungsfall für \ac{CQL} dienen Geoinformationssysteme, mit denen sich die \ac{OGC}
befasst. Daher sind im Gegensatz zu \ac{SQL} geometrische Operatoren wie \texttt{EQUALS},
\texttt{DISJOINT}, \texttt{INTERSECT} oder \texttt{CONTAINS} bereits Teil der Sprache. Die in
\ac{CQL} vorhandenen Datentypen umfassen sowohl primitive Datentypen, als auch zeitliche und
raumbezogene Datentypen.
\parencitealias{ogcCatalogue2016}

Der Entwurf des dritten Teil des OGC-API-Standards \parencitealias{ogcFiltering}, sieht die
Möglichkeit vor, Abfragen über \ac{CQL} zu filtern. Laut Anforderung /req/filter/filter-param soll
dies von Endpunkten wie \texttt{/collections/{collectionId}/items} unterstützt werden, welche
Instanzlisten zurückgeben. Die Filterbedingungen, werden über den GET-Parameter \texttt{filter}
spezifiziert und können mit \textit{CQL\_TEXT} und \textit{CQL\_JSON} umgesetzt werden.
\parencitealias{ogcFiltering}

Während Filterprädikate boolsche Werte und \texttt{NULL} ergeben, können sie aus komplexeren
Funktionen zusammengesetzt sein, welche auch andere Datentypen als Rückgabewert besitzen können.
Über den Endpunkt \texttt{/functions} können Clients Informationen über die verfügbaren Funktionen
abrufen.

\boxnote{%
  Wichtig für praktischen Teil:
  \begin{itemize}
    \item Funktionen
          \begin{itemize}
            \item Selbstauskunft über /functions
            \item einfache Operatoren, spatial and temporal functions
            \item information über inputs gegeben, output typ wird von /functions bestimmt
          \end{itemize}
    \item Verwendung zum Erstellen von Nicht-Filter Ausdrücken
          \begin{itemize}
            \item Filter bestehen aus komplexen Ausdrücken
            \item Kombination aus Funktionen kann Transformationen an Queryables darstellen
          \end{itemize}
    \item Queryables
  \end{itemize}
}
