% Quantitative vs Qualitative Tests
% Formative vs Summative Studien

\subsection{Testmethoden}
intro...
\subsubsection{Formative und Summative Studien}
Usability Tests können in zwei Typen unterteilt werden: \textbf{Formative} und \textbf{Summative Studien}.

\textbf{Formative Studien} werden während der Entwicklung durchgeführt, um Probleme zu diagnostizieren und zu lösen. Der Umfang der einzelnen Studien ist relativ klein, im Entwicklungszyklus können sie jedoch meist wiederholt durchgeführt werden. Die Ergebnisse von formativen Studien können direkt in die Entwicklung einfließen, wobei erneute Studien überprüfen können, ob die Probleme aus vergangen Durchläufen behoben wurden. Des Weiteren können formative Studien aufzeigen, was Nutzer*innen wichtig ist, und somit die Entwicklung in eine Richtung lenken, die die Usability des Endprodukts drastisch verbessert. Erkenntnisse sind in zeitigen Stadien der Produktentwicklung besonders wichtig, da es im Verlauf des Entwicklungszyklus immer schwieriger wird, grundlegend falsche Entscheidungen zu berichtigen.

Für die Zwecke dieser Arbeit ist eine formative Studie angemessen. Daher werden Schlüsselelemente von formativen Studien, wie zum Beispiel Benutzerprofile, Szenarien und das \acl{TA}, im weiteren Verlauf detailliert vorgestellt.
\todo{Link}
\todo{drauf eingehen wieso angemessen}

\textbf{Summative Studien} werden durchgeführt wenn ein Produkt (fast) fertiggestellt ist. Dabei soll unter Usability und Funktionalität auch die Zufriedenheit der Nutzer*innen bewertet werden. Diese Art von Studie findet in einem größeren Rahmen statt und erfordert in der Regel viele Teilnehmer, um statistische Relevanz zu erreichen. Ähnlich zu formativen Studien werden Teilnehmer*innen Szenarien und Aufgaben bereitgestellt, darüber hinaus wird jedoch nicht weiter mit ihnen interagiert. Somit können Metriken wie die Zeit für die Bearbeitung von Aufgaben und Abschlussquoten erhoben werden.

Die Ergebnisse von summativen Studien sind meistens numerisch, während Ergebnisse von formativen Studien konkrete Problembeschreibungen beinhalten. Es ist also einfacher, summativen Studien miteinander zu vergleichen, während formative Studien konkretere Probleme auflisten und in einer agilen Entwicklungsumgebung zeitig und schnell zu Verbesserungen führen können.
\todo{Agil - soll nicht so wirken als wäre agil dafür wichtig?}
\todo{Ergebnisse besser erklären}

\cite{barnumUsabilityTesting2021}

\subsubsection{\acl{TA}}
Das \ac{TA} ist eine weit verbreite Methode, um die Usability von Applikationen einzuschätzen. Dabei sollen Nutzer*innen ihre Erfahrungen, Gedanken, Handlungen und Gefühle beim Interagieren mit der Anwendung ausdrücken. Auf diese Art gesammelte Erkenntnisse \todo{wording} bieten Einblick in Denkprozesse von Nutzer*innen und können somit Entwicklungsentscheidungen lenken. \cite{alhadretiRethinkingThinking2018}

Im Folgenden werden anhand von drei Studien verschiedene Funktionsweisen von \ac{TA} vorgestellt, und Auswirkungen auf den Usability Testing Prozess beleuchtet.

Typischerweise werden zwei Typen von \ac{TA} unterschieden: \ac{CTA} und \ac{RTA}. Bei \ac{CTA} teilen die Teilnehmer*innen ihre Gedanken sofort beim Ausführen der Aufgaben, während dies bei \ac{RTA} erst im Anschluss daran passiert. Einerseits kann \ac{CTA} es einfacher machen Problembereiche zu identifizieren, da die zusätzlichen Informationen in Echtzeit bereitgestellt werden, andererseits existieren folgende Bedenken: Das zum Handeln parallel Sprechen könnte ungewohnt und irritierend für Teilnehmer*innen sein und dadurch zu einer unangenehmen Situation führen. Des Weiteren könnte \ac{CTA} Denkprozesse aktiv beeinflussen und somit die Art und Weise in der Aufgaben abgearbeitet werden verändern. Auch die zusätzliche Zeit die dadurch oft in Anspruch genommen wird sollte bei Betrachtung der Aussagekraft der erhobenen Daten beachtet werden. \todo{zu verworren?} Im Gegensatz dazu beeinflusst \ac{RTA} die Denkprozesse von Teilnehmer*innen nicht, und leidet somit nicht unter den möglichen Problemen \ac{CTA}. Problematisch an dieser Methode ist jedoch, dass sich Teilnehmer*innen korrekt an die Ausführung der Aufgabe erinnern müssen. Außerdem besteht die Gefahr, dass spätere Handlungen und Teilschritte vorherige Gedanken beeinflussen oder relativieren. So könnten wichtige Erkenntnise verloren gehen. Vereinzelt wird auch auf einen hybriden Ansatz gesetzt, bei dem die Erkenntnise von \ac{CTA} durch eine Nachbesprechung angereichert werden. \cite{alhadretiRethinkingThinking2018}

In einer \citeyear{alhadretiRethinkingThinking2018} von \citeauthor{alhadretiRethinkingThinking2018} \cite{alhadretiRethinkingThinking2018} durchgeführten Studie wurden \ac{CTA}, \ac{RTA} und ein hybrider Ansatz miteinander verglichen. Dabei wurden folgende Metriken betrachtet: Allgemeine Leistung beim Ausführen von Aufgaben, Erfahrungen der Teilnehmer*innen, Menge und Qualität der gefundenen Usability-Probleme, sowie die Kosten zur Nutzung der Methoden. Am besten schnitt das \acl{CTA} ab, da es am meisten Probleme fand und das beste Feedback von Teilnehmer*innen erlangte \todo{erlangte??}. Es konnte auch kein Unterschied in der Leistung der Teilnehmer*innen im Vergleich zu Durchläufen ohne \ac{TA} festgestellt werden. Außerdem ist es vorteilhaft, \ac{CTA} gegenüber \ac{RTA} und einem hybriden Ansatz zu wählen, da diese doppelt so viel Test- und Analysierungszeit benötigten. \cite{alhadretiRethinkingThinking2018} \todo{Zitierung nochmal nötig?}

Abgesehen von den soeben betrachteten Typen des \acl{TA} existieren auch Unterschiede im Umfang der Interaktion zwischen Testadministrator*innen und Teilnehmer*innen. \citeauthor{olmsted-hawalaThinkaloudProtocols2010} \cite{olmsted-hawalaThinkaloudProtocols2010} testeten folgende auftretenden Kategorien, räumten jedoch ein dass die Realität oft nuancierter ist:
\begin{itemize}
  \item Traditionell: Außer "Bitte reden Sie weiter." werden keine weiteren Nachfragen angestellt.
  \item Sprach-Kommunikation: In Form von "um-hum" oder "un-hum" werden Teilnehmer*innen zum Weiterreden angeregt. Nach ausreichend langen Stillephasen wird mit dem zuletzt geäußerten Wort das \acl{TA} weiter angeregt. \footnote{Eine Interaktion könnte beispielsweise so aussehen:\\ Teilnehmer*in: "Das war komisch..." \\ -- 15 Sekunden Stille -- \\ Administrator*in: "Komisch?"} \todo{akzeptable Übersetzung von "Speech Communication"?}
  \item Coaching: Es wird aktiv in eingegriffen, indem direkte Fragen gestellt werden und assistiert wird, falls Teilnehmer*innen zu große Schwierigkeiten haben.
\end{itemize}
Die Ergebnisse der \citeyear{olmsted-hawalaThinkaloudProtocols2010} durchgeführten Studie zeigen, dass Teilnehmer*innen, bei denen Coaching angewendet wurde, erfolgreicher als andere waren und zufriedener mit der getesteten Website waren. Die Zeit zur Beendigung der Aufgaben unterschied sich nicht zwischen den verschiedenen Herangehensweisen und, genauso wie bei \citeauthor{alhadretiRethinkingThinking2018}, konnte kein Unterschied zur Kontrollgruppe (ohne \ac{TA}) festgestellt werden. Eine Beeinflussung der Ergebnisse scheint also nur beim Coaching aufzutreten - während die restlichen Variationen des \ac{TA} weder die Genauigkeit noch die Geschwindigkeit abändern. Es wird betont, dass eine genaue Beschreibung des verwendeten \acl{TA} wichtig ist, um einen hohen Grad an Reproduzier- und Vergleichbarkeit zu erreichen.

Eine Studie von \citeauthor{rheniusEvaluationConcurrent1990}, durchgeführt im Jahre \citeyear{rheniusEvaluationConcurrent1990}, kam zu leicht anderen Ergebnissen. Es wurde festgestellt, dass die Zeit zum Lösen von Aufgaben bei der Verwendung von \ac{TA} steigt, die Genauigkeit jedoch nicht davon beeinflusst wurde. \citeauthor{rheniusEvaluationConcurrent1990} kamen auch zu dem Schluss, dass der Einfluss von \ac{TA} nur in den frühen Phasen der Gewöhnung an die Aufgaben auftritt.
\cite{rheniusEvaluationConcurrent1990}

Somit kann also festgehalten werden, dass die Verwendung von \ac{TA} wenn überhaupt nur eine minimale Auswirkung auf die Zeit zur Fertigstellung von Aufgaben hat. Eine Beeinflussung auf die Testergebnisse konnte in keiner der Studien festgestellt werden, außer bei der Verwendung von Coaching. Das \acl{TA} stellt sich somit als gutes Werkzeug zur Durchführung von Usability Studien heraus. Insbesondere eine Verwendung von \acl{CTA} und eine Beschränkung auf minimale Interaktionen mit den Teilnehmer*innen scheint vorteilhaft.
