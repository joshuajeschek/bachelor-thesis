\subsection{Evaluation}\todo[noline]{cross reference with \ref{sec:feedback}}
Im Rahmen der Usability-Studie wurden sowohl quantitative, als auch qualitative Daten erhoben. Aufgrund der intensiven Testsitzungen haben die gesammelten Rückwertungen eine hohe Qualität und können genutzt werden, um Schmerzpunkte in der Benutzung des Block-Editors aufzuspüren. Die quantitativen Erhebungen wurden genutzt, um diese Daten anzureichern und allgemeine Trends zu beschreiben.

Nach jedem Szenario wurde die \acf{SEQ} angewendet. Dabei wurde die folgende Frage gestellt:
\begin{quote}
  \textit{
    Auf einer Skala von 1 bis 7, wie einfach oder schwer fanden Sie es, die Aufgabe zu absolvieren? Dabei entspricht 1 "sehr schwer" und 7 "sehr einfach".}
\end{quote}
Bei dem zweiten und dritten Szenario wurde außerdem darauf hingewiesen, dass die Bewertungen untereinander stimmig sein sollten - das heißt, falls das zweite Szenario als schwieriger empfunden wurde, sollte die Bewertung auch sinken. Somit sollte herausgefunden werden, wie die Komplexität der Aufgaben empfunden wird. Außerdem wurde am Ende der Testsitzungen um eine Einschätzung des gesamten Block-Editors erfragt:
\begin{quote}
  \textit{
    Bewerten Sie die Einfachheit der Benutzung auf einer Skala von 1 bis 10, wobei 1 "sehr schwer" entspricht, und 10 "sehr einfach".
  }
\end{quote}
Nach Beantwortung dieser Frage wurden diejenigen Nutzer:innen, die die alte Bearbeitungsoberfläche für Konvertierungen kannten, gebeten, dieses auf der gleichen Skala zu bewerten. Dies diente dazu, eine etwaige Verbesserung in der neuen Herangehensweise festzustellen.

Unabhängig von den numerischen Bewertungen wurde im Laufe der Testsitzungen konstant Feedback gesammelt. Einerseits wurden Äußerungen der Teilnehmer:innen notiert, die sie im Zuge von \ac{CTA} getätigt haben, andererseits wurden auch ihre Handlungen und auftretende Probleme notiert. Falls auf diese Probleme nicht während des Szenarios eingegangen wurde, konnte danach darüber gesprochen werden.

Am Ende der Testsitzung wurden Fragen zur Benutzung des Block-Editors gestellt. Diese sind in Anhang \ref{app:moderation} aufgelistet. Der Fragenkatalog zielt sowohl darauf ab, die allgemeine Meinung zum Block-Editor abzufragen, als auch über die gesammelten Probleme zu reden und herauszufinden, welche Funktionalitäten den Nutzer:innen wichtig sind. Oft führte dies auch zu einem offenen Gespräch welches für weitere, spezifische Einblicke in das Nutzungsverhalten sorgen konnte.
