\subsection{Szenarios}
\label{sec:scenarios}

Damit die Benutzung einer Anwendung bei Usability Tests beobachtet und bewertet werden kann, müssen den Teilnehmer:innen spezifische und realistische Aufgaben gegeben werde. Dies betonen \textcite{mccloskeyTaskScenarios2014} und \textcite{barnumUsabilityTesting2021}. Szenarios werden verwendet, um die Aufgaben mit einer Erklärung und Kontext anzureichern \parencite{mccloskeyTaskScenarios2014}. Sie bilden somit einen Weg den Nutzer:innen zu beschreiben, welche Aktionen sie in der Anwendung ausführen sollen, ohne ihnen konkret zu beschreiben, was sie tun sollen.

\textcite{mccloskeyTaskScenarios2014} beschreibt drei Eigenschaften von guten Aufgaben: Sie sollten realistisch sein, als Handlung ausführbar sein und es vermeiden Tipps zu geben oder Unterschritte zu beschreiben.

Eine realistische Aufgabe macht es einfacher für Nutzer:innen sich in das Szenario zu versetzen und die Aufgabe auszuführen, während sie die Oberfläche benutzen. Es kann einfacher sein, den Teilnehmer:innen etwas Spielraum in den Details der Aufgabe zu geben, damit sie so vorgehen können, wie es für sie realistisch ist. Damit Szenarios realistischer wirken, empfiehlt \textcite{barnumUsabilityTesting2021}, die Sprache der Nutzer:innen statt der des Produktes zu verwenden. Zweiteres kann auch die unerwünschte Folge haben, dass bereits Informationen über den vorgesehenen Lösungsweg offenbart werden.
\parencite{mccloskeyTaskScenarios2014}

Außerdem ist es besser, nach einer Handlung zu fragen, statt sie beschreiben zu lassen. \textcite{nielsenFirstRule2001} geht darauf ein, dass dies immer besser ist, da eine Einschätzung nicht so genaue Ergebnisse birgt wie die Beobachtung der Handlung. Grund für diese Annahme ist die Forschungsarbeit von \textcite{nielsenMeasuringUsability1994}, aus der hervorgeht, dass Nutzer:innen bei zwei unterschiedlichen Designs eine Meinung darüber abgeben können, welches besser ist, aber diese Information wird immer genauer sein, wenn sie die Möglichkeit hatten, die Oberfläche selbst zu benutzen. Die Interaktion mit der Anwendung kann Einblicke in Problembereiche und Frustrationen bieten, die durch eine einfache Beschreibung nicht verfügbar gewesen wären. \parencite{mccloskeyTaskScenarios2014} \todo{Zitierung in diesem Paragraph überprüfen (alles stammt von \textcite{mccloskeyTaskScenarios2014}, außer die Sätze die extra mit Zitationen versehen sind)}

Zu guter Letzt sollte es vermieden werden, den Teilnehmer:innen in der Aufgabenbeschreibung Tipps zu geben oder den Lösungsweg zu detailliert auszuführen. Ausdrücke, die in der Benutzeroberfläche verwendet werden, sollten vermieden werden, da so das Verhalten der Teilnehmer:innen beeinflusst wird \parencite{mccloskeyTaskScenarios2014, barnumUsabilityTesting2021}. Beschreibungen von Unterschritten enthalten oft Tipps über den Aufbau und die vorgesehene Art und Weise der Anwedungsbenutzung. Dies kann dazu führen, dass weniger hilfreiche Informationen über das Verhalten der Nutzer:innen gewonnen werden. \textcite{barnumUsabilityTesting2021} betont deshalb dass es besser ist ein Ziel anzugeben, statt einer Reihenfolge an Schritten um dieses Ziel zu erreichen.
\parencite{mccloskeyTaskScenarios2014}

Eine Usability Studie besteht typischerweise aus mehreren Szenarios. \textcite{barnumUsabilityTesting2021} geht darauf ein, in welcher Reihenfolge Szenarios organisiert werden können. Das erste Szenario kann kurz gehalten werden, und auf den allgemeinen ersten Eindruck der Anwendung eingehen. Ein kurzes erstes Szenario gibt den Teilnehmer:innen die Möglichkeit sich mit dem Prozess und den Methoden der Studie vertraut zu machen, und unvorhergesehene Probleme technischer oder logistischer Art können gelöst werden. Außerdem empfiehlt es sich Missverständnisse nach dem ersten, kurzen Szenario aus dem Weg zu räumen, damit während einem längeren Szenario keine Unterbrechung entsteht, die den Gedankenfluss und die Ergebnisse stören könnte. Die nachfolgenden Szenarien können aufeinander aufbauen, nach Häufigkeit der Ausführung beziehungsweise Wichtigkeit der Aufgabe sortiert werden, oder einem vorgesehenen Aufgabenfluss folgen, der somit gleichzeitig überprüft wird.
\parencite{barnumUsabilityTesting2021}

Laut \textcite{barnumUsabilityTesting2021} ist das Ende von Szenarien auch von Bedeutung für die Ergebnisse einer Usability Studie. Unsicherheit bei den Teilnehmer:innen, ob eine Aufgabe abgeschlossen ist, ob sie korrekt durchgeführt wurde oder noch nicht beendet ist, kann auf Probleme im Design der Anwendung hindeuten. Deshalb ist es wichtig diese Informationen zu erheben, darauf zu achten wie die Aufgaben beendet werden, und den Teilnehmer:innen die Möglichkeit zu geben, zu entscheiden, wann sie eine Aufgabe absolviert haben.
\parencite{barnumUsabilityTesting2021}

Gute gewählte und formulierte Aufgaben haben den Vorteil, dass Verhaltensmuster, und häufig auftretende Muster erkannt werden können \parencite{barnumUsabilityTesting2021}. Des Weiteren besteht die Möglichkeit, die Erfolgsrate als grobes Maß für die Usability zu benutzen \parencite{nielsenSuccessRate2001}. Andere Maße und Informationen sollten jedoch weiterhin gesammelt werden, da Erfolg nur von einem Mindestmaß an Usability spricht \parencite{nielsenSuccessRate2001}.
