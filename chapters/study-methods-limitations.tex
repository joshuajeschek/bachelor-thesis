\subsection{Grenzen der Usability-Studie}
Die Erkenntnisse der Usability-Studie geben einen guten Einblick in vorhandene Usability-Probleme, welche grob nach Häufigkeit des Auftretens sortiert werden können. Diese Metrik lässt jedoch keine Aussage über den Härtegrad der aufgetretenen Probleme zu. Es muss weiterhin für jedes Problem einzeln und nach subjektivem Empfinden entschieden werden, inwiefern es die Nutzer:innen am erfolgreichen Erfüllen von Aufgaben hindert, die empfundene Produktivität beeinflusst oder zunächst zu vernachlässigen ist.

\pskip
Des Weiteren ist festzuhalten, dass die begrenzte Anzahl von Teilnehmer:innen die All­ge­mein­gül­tig­keit der Ergebnisse beeinflusst. Da die befragten Nutzer:innen nur einem Teil der gesamten Nutzer:innenbasis von Simplex4TwIS entspricht, sind die gesammelten quantitativen Daten nicht repräsentativ. Darunter fallen sowohl die Antworten auf die \ac{SEQ} nach den Szenarios als auch die Einschätzung der allgemeinen Einfachheit der Benutzung. Sie dienen nur der Anreicherung der Aussagen der Teilnehmer:innen und als Denkanstoß, um die angegebene Punktzahl zu begründen.

Eine größere Anzahl von Teilnehmer:innen könnte mehr Usability-Probleme aufdecken, wobei jedoch nicht bekannt ist, wie groß der zu erwartende Effekt wäre.

\pskip
Im Vorfeld der Studie wäre es möglich gewesen die Teilnehmer:innen um eine Selbsteinschätzung bezüglich ihrer \ac{SQL}- und Programmierkenntnisse zu bitten. Da dies unberücksichtigt blieb, kann lediglich eine allgemeine Einschätzung, die aus Gesprächen hervorgeht, getroffen werden. Außerdem wäre es von Interesse gewesen, mehr Menschen zu engagieren, die über keine \ac{SQL}- und Programmierkenntnisse verfügen. Im Rahmen der Studie konnten nur drei Personen die Oberfläche testen, bei denen dies der Fall war. Da ein Ziel des Prototyps war, die Benutzung für diese Personengruppe zu erleichtern, wäre es einfacher festzustellen, wie erfolgreich dieses Ziel erreicht wurde.

\pskip
Der Umfang der Usability-Studie ermöglichte es nicht, die gesamte Komplexität von einigen Problemstellungen zu testen, die in der Praxis auftreten. Wie bereits in \ref{sec:future} erwähnt, müssen in der Praxis beispielsweise mehr Funktionen bereitgestellt werden. Demnach ist es unklar, inwiefern die gesammelten Ergebnisse übertragbar sind.
