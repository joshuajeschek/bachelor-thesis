\subsection{Technische Umsetzung}
\todo[noline]{welche infos sind hier zu kurz gekommen?}
Der Block-Editor gliedert sich in die restliche Anwendung, Simplex4Data, ein. Dabei wird die Serveranwendung unter Verwendung von Python und dem Django-Framework \parencite{djangosoftwarefoundationDjango} programmiert. Das Frontend ist eine Browser-Anwendung, die mit \ac{html}, JavaScript und \ac{css} (\ac{sass}) umgesetzt wurde.

Das Backend generiert \ac{html}, welches an den Client gesendet wird. Da es sich bei dem vorliegenden Editor jedoch um eine sehr dynamische Anwendung handelt, wird nur ein Grundgerüst auf dem Server gebildet. Auf dem Client wird per \ac{api} abgefragt, welche abfragbaren Felder, Funktionen und auszufüllende Felder existieren. Aus dem so erhaltenen \ac{json} werden dann UI-Elemente mit Nunjucks \parencite{mozillaNunjucks} generiert. Des Weiteren wurde die JavaScript-Bibliothek jQuery \parencite{openjsfoundationJQuery} genutzt, um Zustände zu verwalten und auf Interaktionen zu reagieren.

Die für die Funktionsweise benötigten Daten stammen komplett aus dem \textit{API - Features} Dienst \parencite{ogcAPI} von Simplex4Data. Simplex4Data speichert die in den Quelltabellen enthaltenen Spalten und die Attribute der Objektklasse, inklusive Metadaten wie Titel, Schlüssel, Beschreibung und Datentyp. Diese werden über den \ac{api}-Endpunkt \texttt{/queryables} ausgegeben (Vgl. \ref{sec:queryables}). Somit ist es möglich, bei Konvertierungen den korrespondierenden Endpunkt der Quelltabelle anzufragen, sowie den der Objektklasse, um sowohl die Metadaten der Ausgangsdaten, als auch die der Zielstruktur abzufragen. Gleichweise kann beim Editieren von Szenarien der \texttt{/queryables}-Endpunkt der dazugehörigen Objektklassen genutzt werden, um die relevanten abfragbaren Felder zu bestimmen. Simplex4Data hat für diesen Zweck den \textit{API - Features} Dienst erweitert, um die Attribute mehrerer, miteinander verbundener Objektklassen auszugeben. Bei diesem Schritt werden auch die Präfixe, abhängig von der Objektklasse, gesetzt.

Auch die im Editor nutzbaren Funktionen werden aus der \ac{api} abgerufen, vom \texttt{/functions}-Endpunkt (Vgl. \ref{sec:function}).
\todo[noline]{how do types differ between queryables and functions? information loss?}

Bildungsvorschriften, welche im Block-Editor erstellt werden, werden in \ac{cql} verwaltet und im Anschluss an den \ac{api}-Dienst geschickt. Laut \citetitle{ogcFiltering} müssen konforme Dienste die Möglichkeit unterstützen, mit \ac{cql} zu filtern. Simplex4Data bietet außerdem die Möglichkeit an, in \ac{api}-Abfragen Ausgabefelder zu definieren. Somit kann der komplette Funktionsumfang des Editors mit der \ac{api} abgebildet werden.
