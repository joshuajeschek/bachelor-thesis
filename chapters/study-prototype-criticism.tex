\subsection{Kritik}
\label{sec:criticism}

Im Zuge dieser Arbeit wurde der Block-Editor nicht komplett fertig programmiert. Er ist noch nicht in das restliche System Simplex4TwIS integriert: Über den Editor erstellte Konvertierungen und SimplexSzenarios können nicht abgespeichert oder zu Beginn der Bearbeitung eingelesen werden. Dies wird vor einer endgültigen Integration noch umgesetzt werden müssen, beeinflusst aber nicht das Testen des Konzepts in diesem Rahmen. Des Weiteren wurden einige offene Funktionalitäten identifiziert, die in geringerem Maße zur grundlegenden Herangehensweise der blockbasierten Datenverarbeitung beitragen und somit als weniger relevant erachtet wurden. Das Fehlen dieser Funktionalitäten könnte die Einfachheit der Nutzung jedoch beeinflussen.

Dazu gehört zum Beispiel das Umsortieren von bereits definierten Strukturen. Es ist nicht möglich, einen definierten Block zu verschieben, zu kopieren, oder ausgefüllte Parameter zu tauschen. Somit könnte es vereinfacht werden, mit komplexen Definitionen zu arbeiten und Fehler zu beheben.

Außerdem könnte es in Zukunft hilfreich sein, einzelne Blöcke einzuklappen, um horizontalen Platz zu sparen. Die eingeklappte Version könnte dann eine Zusammenfassung anzeigen, ohne die Möglichkeit der Bearbeitung zu geben. Dies kann praktisch für komplexe Definitionen von Konvertierungen oder SimplexSzenarios sein, die sonst zu viel Platz einnehmen würden und insbesondere beim ersten Lesen überwältigend sein können.

Aktuell ist der Editor so konzipiert, dass zuerst die äußeren Elemente (z.B. Funktionen) ausgewählt werden, und dann mit weiteren Elementen befüllt werden (Parameter, d.h. Funktionen oder Attribute). Diese Herangehensweise ist eng an die Definition von Funktionen in gängigen Programmiersprachen angelegt und könnte verwirrend für Menschen sein, deren Hintergrund nicht-technischer Natur ist. Um Abhilfe zu schaffen, könnte eine Option eingeführt werden, bereits ausgewählte Elemente in Funktionen einzusetzen, ohne diese neu Auswählen zu müssen. Das Element (eine Funktion oder ein Attribut) könnte dann als erster Parameter in der neu ausgewählten Funktion gesetzt werden, und somit erhalten bleiben. Auch umgekehrt bestehen noch Verbesserungsmöglichkeiten, da es nicht möglich ist, eine umklammernde Funktion aufzulösen, ohne die Parameter komplett zu verwerfen. Vor einer möglichen Implementierung muss jedoch geklärt werden, wie mit Datentypen umgegangen werden soll, die sich vom äußeren Feld zum inneren Feld (Parameter) unterscheiden.

Der Block-Editor unterstützt zwar bereits Funktionen mit dem gleichen Name, die sich in Anzahl oder Typ der Parameter unterscheiden (Überladungen, bzw. \textit{Overloads}), bietet aber keinen Weg, um bereits befüllte Parameter beim Austauschen der Überladung zu übernehmen. Wird zum Beispiel eine Funktion mit zwei Parametern ausgefüllt, dann aber auf die Version mit drei Parametern abgeändert, sind wieder alle drei Parameter leer. Die zwei ursprünglich definierten Parameter sind noch zwischengespeichert, und demnach nicht verloren. Falls die Nutzer:innen zurückschalten, ist die zuerst definierte Version weiterhin enthalten. Es besteht jedoch keine Heuristik, die Parameter aus einer Funktionsüberladung in die nächste übernimmt. Dabei müsste auf die Typen der Parameter geachtet werden, und auf etwaige Änderungen in der Bedeutung der Parameter beim Anpassen der Überladung.

Einige Funktionen verfügen über mehrere Rückgabetypen, welche basierend auf den Eingangsdaten oder deren Typen einen anderen Datentyp zurückgeben. Die \texttt{/functions}-Schnittstelle unterstützt die Ausgabe solcher Funktionen \parencitealias{ogcFiltering}, der Block-Editor jedoch nicht. Um auf Basis eines jeden \textit{API - Features}-Dienst arbeiten zu können, sollte dies behoben werden. Eine offene Frage ist jedoch, wie solche Funktionen im Block-Editor dargestellt werden sollen, und inwiefern es Sinn ergibt dies zu unterstützen. Funktionen mit mehreren Rückgabetypen könnten dazu führen, dass es schwieriger wird die strukturelle Korrektheit der Abfragen zu garantieren, da es nicht mehr eindeutig ist, welchen Datentyp der angegebene Funktionsaufruf produziert. Um dem entgegenzuwirken, erscheint es sinnvoller, mehrere Versionen der Funktion zur Verfügung zu stellen, die jeweils nur einen Rückgabetyp besitzen, und ihr Verhalten somit klarer definieren.

Im Laufe der Entwicklung des Prototyps wurde nicht beschlossen, welche Funktionen standardmäßig im Block-Editor enthalten sein sollten. Bisher war es in Konvertierungen möglich, alle vorhandenen PostGIS- und PostgreSQL-Funktionen zu nutzen. Der für die Usability-Studie verwendete Prototyp enthält nur eine Teilauswahl dieser Funktionen und ist somit noch nicht in der Lage alle benötigten Abfragen für den Praxisbetrieb zu unterstützen. Sollten mehr Funktionen hinzugefügt werden, ist zu überprüfen, ob das Auswahlmenü weiterhin übersichtlich genug ist.

Die Verwendung von Aggregatsfunktion wie \texttt{MAX}, \texttt{SUM} oder \texttt{ST\_Multi}, welche Teil von \ac{SQL} oder PostGIS sind, wird noch nicht unterstützt. Das in Abbildung \ref{fig:buffet-scenario} gezeigte SimplexSzenario kann nicht ausgeführt werden, da die von \ac{SQL} benötigte \texttt{GROUP BY}-Klausel nicht definiert wird. Es sollte überprüft werden, inwiefern dies automatisch anhand der benutzten Funktionen geschehen kann. Dafür müssen die Funktionen mit Metadaten angereichert werden, die ausdrücken, ob es sich bei ihnen um Aggregatsfunktionen handelt.
