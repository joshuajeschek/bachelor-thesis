\subsection{Kritik}
\label{sec:criticism}

Im Zuge dieser Arbeit wurde der Block-Editor nicht komplett fertig programmiert. Er noch nicht in das restliche System Simplex4Data integriert, über den Editor erstellte Konvertierungen und Simplex-Szenarien können nicht abgespeichert oder zu Beginn der Bearbeitung eingelesen werden. Dies wird vor einer endgültigen Integration noch umgesetzt werden müssen, beeinflusst aber nicht das Testen des Konzepts in diesem Rahmen. Des Weiteren wurden einige offene Funktionalitäten identifiziert, die in geringerem Maße zur grundlegenden Herangehensweise der blockbasierten Datenverarbeitung beitragen und somit als weniger relevant erachtet wurden. Das Fehlen dieser Funktionalitäten könnte die Einfachheit der Nutzung jedoch herabsetzen.

Dazu gehört zum Beispiel das Umsortieren von bereits definierten Strukturen. Es ist nicht möglich, einen definierten Block zu verschieben, zu kopieren, oder ausgefüllte Parameter zu tauschen. Somit könnte es vereinfacht werden, mit komplexen Definitionen zu arbeiten und Fehler zu beheben.

Außerdem könnte es in Zukunft hilfreich sein, einzelne Blöcke einzuklappen, um horizontalen Platz zu sparen. Die eingeklappte Version könnte dann eine Zusammenfassung anzeigen, ohne die Möglichkeit der Bearbeitung zu geben. Dies kann praktisch für komplexe Definitionen von Konvertierungen oder Simplex-Szenarien sein, die sonst zu viel Platz einnehmen würden und insbesondere beim ersten Lesen überwältigend sein können.

Aktuell ist der Editor so konzipiert, dass zuerst die äußeren Elemente (z.B. Funktionen) ausgewählt werden, und dann mit weiteren Elementen befüllt werden (Parameter, d.h. Funktionen oder Attribute). Diese Herangehensweise ist eng an die Definition von Funktionen in gängigen Programmiersprachen angelegt und könnte verwirrend für Menschen sein, deren Hintergrund nicht-technischer Natur ist. Um Abhilfe zu schaffen, könnte eine Option eingeführt werden, bereits ausgewählte Elemente von Funktionen zu umgeben. Das Element (eine Funktion oder ein Attribut) \todo{zu verworren?} könnte dann als erster Parameter in der neu ausgewählten Funktion gesetzt werden, und somit erhalten bleiben. Auch umgekehrt bestehen noch Verbesserungsmöglichkeiten, da es nicht möglich ist, eine umklammernde Funktion aufzulösen, ohne die Parameter komplett zu verwerfen.

Der Block-Editor unterstützt zwar bereits Funktionen mit dem gleichen Name, die sich in Anzahl oder Typ der Parameter unterscheiden (Überladungen, bzw. \textit{Overloads}), bietet aber keinen Weg um bereits befüllte Parameter beim Austauschen der Überladung zu übernehmen. Wird zum Beispiel eine Funktion mit zwei Parametern ausgefüllt, dann aber auf die Version mit zwei Parametern abgeändert, sind wieder alle drei Parameter leer. Die original\todo{original?} definierten zwei Parameter sind noch zwischengespeichert, und demnach nicht verloren. Falls die Nutzer:innen zurückschalten, ist die zuerst definierte Version weiterhin enthalten. Es besteht jedoch keine Heuristik, die Parameter aus einer Funktionsüberladung in die nächste übernimmt. Dabei müsste auf die Typen der Parameter geachtet werden, und auf etwaige Änderungen in der Bedeutung der Parameter beim Anpassen der Überladung.

\todo[inline]{vielleicht hier queryables vs functions einbringen?}
\todo[inline]{vielleicht anbringen dass kein automatisches typecasting bei Simplex-Szenarien?}
\todo[inline]{nicht alle funktionen von postgis etc.}
