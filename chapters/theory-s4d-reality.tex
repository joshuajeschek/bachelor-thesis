\subsection{Das Realitätsmodell}
\label{sec:simplex-reality}

Das Realitätsmodell bildet das zentrale Element von Simplex4TwIS. In der Datenbank werden die gesammelten Daten gemäß dieses Modells homogen abgelegt. Dabei erfolgt die Speicherung harmonisiert, das heißt alle Daten werden in einem gleichmäßigen Datenbankschema abgelegt - es werden keine fachspezifischen Modelle benötigt. So werden Auswertungen vereinfacht und Interoperabilität gewährleistet. Außerdem werden fachspezifische Strukturen in Einzelteile zerlegt und nur atomare Objekte abgespeichert.

Die Grundelemente des Modells bilden Objekte, Attribute und Verbindungen, welche immer gleiche Datenstrukturen besitzen \parencite{grossmannFachsystemeSchemaevolution2024}. Objekte bilden einzelne Datenelemente ab und sind Teil von Objektklassen, die diese beschreiben. Jede verfügbare Dateneinheit wird als Objektklasse abgebildet, seien es reale Dinge wie Bäume, Häuser oder Gemeinden, oder Prozesse und Handlungen die für einen Datensatz von Relevanz sind. Jede Objektklasse besitzt die gleichen Standardfelder, die oft benötigt werden. Tabelle \ref{tab:s4d-fields} listet die verfügbaren Standardfelder auf. Metadaten wie Name, Kurzbeschreibung und Kommentar zu den Standardfeldern einer Objektklasse werde über das gleiche Modell beschrieben und in identischen Strukturen erfasst. Somit ist es möglich, die Metadaten je nach Objektklasse anzupassen. Informationen, die über die Standardfelder hinausgehen, können über Attribute angefügt werden, die separat verwaltet werden und mit ihren Objekten verbunden werden \parencite{grossmannFachsystemeSchemaevolution2024}. Attribute können verschiedene Datentypen annehmen, einschließlich Gleitkommazahlen und geometrische Werte.

\begin{table}[!ht]
  \centering
  \begin{tabular}{|| l | l | l ||}
    \hline
    Schlüssel & Name               & Datentyp   \\[0.5ex]
    \hline\hline
    ndx       & Interner Schlüssel & Text       \\
    key       & Fachschlüssel      & Text       \\
    typ       & Typ                & Text       \\
    nam       & Name               & Text       \\
    dsc       & Kurzbeschreibung   & Text       \\
    cmt       & Kommentar          & Text       \\
    beg       & Anfang             & Datum/Zeit \\
    fin       & Ende               & Datum/Zeit \\
    \hline
  \end{tabular}
  \caption[Standardfelder in Simplex4TwIS]{Standardfelder in Simplex4TwIS. Sie sind in jeder Objektklasse mit den gleichen Schlüsseln vorhanden, Metadaten wie Name und Kurzbeschreibung der Felder kann jedoch angepasst werden. \parencite{simplex4datagmbhSimplex4TwIS}}
  \label{tab:s4d-fields}
\end{table}

Über Objekte und Attribute können die meisten Daten abgebildet werden, einen großen Mehrwert schaffen jedoch Verbindungen \parencite{rudolfUmweltdatenIntelligenz2021}. Verbindungen können zwischen zwei Objekten angelegt werden, um Beziehungen auszudrücken, deren Bedeutung beim Anlegen der Verbindungsklasse festgelegt wird. Somit können komplexe und fachübergreifende Sachverhalte mit einfachen Strukturen abgebildet werden \parencite{grossmannFachsystemeSchemaevolution2024}.

Das Realitätsmodell wird auf allen Ebenen der Datenhaltung benutzt und ist selbstbeschreibend. Auch die Metadaten werden mit dem gleichen Modell beschrieben \parencite{grossmannFachsystemeSchemaevolution2024}. Die Standardfelder aus Tabelle \ref{tab:s4d-fields} finden beispielsweise auch bei der Definition von Objektklassen, Verbindungsklassen, Standardfeldern und Attributen Verwendung.
