\subsection{Der Block-Editor}

Es wurde sich für die Entwicklung einer Low-Code-Oberfläche entschieden, in der einzelne Elemente als Blöcke dargestellt werden. Wie diese bei der Bearbeitung von Konvertierungen aussieht, ist in Abbildung \ref{fig:buffet-simple} zu sehen. Der Informationsfluss wurde von links nach rechts konzipiert, sodass links die Ausgangsdaten zu finden sind, die rechts eingetragen werden können.

Daher befinden sich im linken Bereich sowohl die abfragbaren Tabellenspalten (als "Abfragbare Felder" bezeichnet) und die Funktionen, die verwendet werden können. Über diesen beiden beiden Abschnitten sind Schaltflächen zum Filtern untergebracht, sodass die abfragbaren Felder oder Funktionen wahlweise ausgeblendet werden können. Die Elemente werden als große Schaltflächen dargestellt und enthalten relevante Informationen: Der Datentyp wird als Text (grau, oben links) und als Symbol (rechts) angezeigt. Bei Funktionen handelt es sich hierbei um den Rückgabetyp. \todo{Funktionen mit mehreren Rückgabetypen (Probleme)} Abfragbare Felder weisen ihre Namen und Schlüssel auf (z.B. "Bevölkerung" und "bevoelkerung"), während Funktionen über einen Name und eine Beschreibung verfügen (z.B. "=" und "Gleichheitsprüfung"). \todo{beschreiben was dadurch gelöst wird! Nachschlagen!}

\begin{figure}[ht]
  \centering
  \includegraphics[width=.95\textwidth]{assets/buffet-simple.png}
  \caption{Verwendung des Block-Editors zur Erstellung einer Konvertierung der Klasse "Gemeinde"}
  \label{fig:buffet-simple}
\end{figure}

Im rechten Bereich der Oberfläche wird die Zielstruktur dargestellt. Im Falle der Konvertierung wird diese durch die Klassendefinition vorgeschrieben. Die Eingabemaske für die einzelnen Attribute besteht aus einem Symbol für den Datentyp, dem Attribut-Name und -Schlüssel, sowie einem an den Datentyp angepassten Eingabefeld. Sobald ein Eingabefeld mit einer Auswahl befüllt wurde, besteht die Möglichkeit, diese wieder durch die Schaltfläche am rechten Rand zu entfernen. Falls das Feld durch eine Funktion gebildet werden soll, muss zuerst die Funktion ausgewählt werden, wodurch wiederrum zu Parametern korrespondierende Eingabefelder entstehen. Diese können dann analog mit Attributen oder Funktionen befüllt werden. Somit ist es möglich, beliebig komplexe Schachtelungen zu erstellen\footnote{Eine Obergrenze für den Grad der Schachtelung könnte maximal durch begrenzten Bildschirmplatz erreicht werden. Da pro Schachtelung nur wenig horizontaler Platz verloren geht, liegt diese Grenze jedoch höher als in der Praxis benötigt. \todo{ist das so? bisschen relativieren?}}. Die Zusammengehörigkeit von Parametern und Funktionen wird durch einen vertikalen Strich an der linken Seite dargestellt. Abbildung \ref{fig:buffet-scenario} zeigt, wie dies mit einer komplexeren Abfrage aussieht.

\begin{figure}[ht]
  \begin{center}
    \includegraphics[width=.95\textwidth]{assets/buffet-selected.png}
  \end{center}
  \caption{Erstellung einer Konvertierung der Klasse "Gemeinde", das Feld "Name" ist ausgewählt.}
  \label{fig:buffet-selected}
\end{figure}

Die dargestellten Datentypen erfüllen nicht nur eine informative Funktion, sondern sind auch inhaltlich relevant. Felder können nur mit Elementen befüllt werden deren Datentyp übereinstimmt. Somit soll ein Minimum an Korrektheit an der Abfragen sichergestellt werden.\todo{später drauf eingehen, ob das so geklappt hat (ja, keine fehlerhaften Abfragen wurden generiert, aber ob inhaltlich richtig nicht überprüfbar!)} Außerdem wird diese Einschränkung dazu genutzt, die Zuordnungsmöglichkeiten zu reduzieren. Wird auf das zu einem Attribut gehörige Eingabefeld geklickt, wird der Auswahlbereich links auf Elemente des Datentyps des Attributs gefiltert. Dies ist anhand des Attributs "Name" in Abbildung \ref{fig:buffet-selected} zu sehen. Umgedreht ist es auch der Fall: Wird links ein Attribut oder eine Funktion angeklickt, werden rechts die Felder ausgegraut, in die die Auswahl nicht eingefügt werden kann. \todo{braucht es da noch ein Bild?}
\todo[inline]{Bearbeitungsreihenfolge RtL oder LtR, filtern}

\begin{figure}[ht]
  \begin{center}
    \includegraphics[width=.95\textwidth]{assets/buffet-scenario.png}
  \end{center}
  \caption{Verwendung des Block-Editors zur Erstellung eines SimplexSzenarios. \todo{Bild mit mehreren Schachtelungen, ohne groupy by}}
  \label{fig:buffet-scenario}
\end{figure}

Der Block-Editor kann auch verwendet werden, um SimplexSzenarios zu erstellen. Dabei werden mehrere Bedienelemente hinzugefügt. Ein wichtiger Unterschied zu Konvertierungen besteht darin, dass die abfragbaren Felder aus den Attributen mehrerer Klassen bestehen. Diese unterscheiden sich einerseits im Präfix ihrer Schlüssel, zur schnellen Identifizierung wurde jedoch auch eine farbliche Anpassung vorgenommen. Sowohl das Symbol für den Datentyp, als auch die Umrandung sorgen für eine farbliche Sortierung der Objektklassen. Diese Farben werden auch im Kopf des linken Bereichs widergespiegelt, dort kann nun analog zum Elementtyp (Felder oder Funktionen) auch nach Klassen gefiltert werden. \todo[noline]{auf bild eingehen}

Des Weiteren verfügt der Block-Editor im Szenario-Modus über einen Button, um neue Felder hinzuzufügen. Dabei muss bereits der gewünschte Datentyp gewählt werden. Während standardmäßig Felder vom Typ Text hinzugefügt werden, können die anderen Datentypen über ein Dropdown ausgewählt werden.

Sowohl Titel als auch Schlüssel von so kreierten Feldern kann bearbeitet werden. Die Darstellung ist die gleiche wie bei Konvertierungen, mit dem Unterschied dass die bearbeitbaren Informationen gepunktet unterstrichen sind und ein Stift-Symbol daneben angezeigt wird.
