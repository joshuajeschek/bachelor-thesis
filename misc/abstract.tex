\vspace*{0pt plus 1fill}
\begin{tucsimplesection}{Abstract}
  Die Bereitstellung und Auswertung von Umweltdaten erfordert regelmäßig die Konvertierung von Datensätzen zwischen verschiedenen Formaten. Der Konvertierungsprozess ist mit einem hohen Aufwand verbunden und erfordert technische Expertise.
  Das Data Warehouse Simplex4TwIS reduziert den Aufwand bei der Datenkonvertierung, indem es die Daten in einem harmonisierten, objektorientierten Datenmodell abspeichert. Dazu werden die Fachdaten in ein einheitliches Format konvertiert, aus dem dann beliebige Auswertungen generiert werden können.
  Für die Bewältigung dieser Aufgaben ist jedoch nach wie vor die manuelle Eingabe von \acs{SQL}-Abfragen erforderlich, was sich nachteilig auf die Usability auswirkt und insbesondere Menschen mit weniger technischen Kenntnissen benachteiligt.
  Im Rahmen dieser Arbeit wird ein blockbasierter Ansatz präsentiert, der eine effiziente Bewältigung der Datenkonvertierung in Simplex4TwIS ermöglicht.
  Ein Block-Editor wurde entwickelt, dessen Bedienung ohne \acs{SQL}-Kenntnisse möglich ist und der häufiges Nachschlagen sowie Tippfehler vermeiden soll. Im Rahmen einer formativen Studie konnte nachgewiesen werden, dass der Ansatz von den typischen Nutzer:innen von Simplex4TwIS positiv bewertet wird und als Verbesserung im Vergleich zur aktuellen Lösung betrachtet wird. Darüber hinaus wurden Schwachpunkte des Editors identifiziert, die in der weiteren Entwicklung adressiert werden sollten.
  Der Editor gestaltet den Konvertierungsprozess effizienter als zuvor und ermöglicht es mehr Nutzer:innen, Umweltdaten zu konvertieren. Er stellt somit eine sinnvolle Erweiterung für Simplex4TwIS dar.
\end{tucsimplesection}
\vspace*{0pt plus 2.5fill}
