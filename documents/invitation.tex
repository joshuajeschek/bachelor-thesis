\ifdefined\STANDALONE
  \section*{Einladung zu Usability-Studie}
\else
  \chapter{Einladung zur Usability-Studie}
  \label{app:invitation}
\fi

\begin{tabular}{ l l }
  \textbf{Zeitaufwand} & 90 Minuten                                                                   \\
  \textbf{Ort}         & Online-Konferenzraum von Simplex4Data                                        \\
  \textbf{Termin}      & flexibel nach Absprache, jeder Wochentag möglich                             \\
  \textbf{Kontakt}     & \href{mailto:joshua.jeschek@simplex4data.de}{joshua.jeschek@simplex4data.de}
\end{tabular}

\vspace{2\baselineskip}

% Tool existiert schon
\noindent
Im Rahmen meiner Bachelorarbeit arbeite ich an der Entwicklung einer Low-Code Anwendung, die das System Simplex4TwIS erweitert. Diese Anwendung zielt darauf ab, die Konvertierung und Bearbeitung von Datensätzen durch einen blockbasierten Ansatz zu vereinfachen. Das Tool ermöglicht effiziente und benutzerfreundliche Datenbearbeitung für Fachleute - unabhängig davon, ob sie bereits Erfahrung mit dem Simplex-Ansatz und Datenbanken haben oder nicht.

Für die Weiterentwicklung und Optimierung dieses Projekts bin ich auf der Suche nach Fachleuten, die im Arbeitsalltag mit (Geo)-Datensätzen umgehen und an einer Usability-Studie teilnehmen wollen. Ein tiefgehendes Verständnis von Datenbanken ist für die Teilnahme nicht erforderlich. Ihr Fachwissen und Ihre Erfahrungen sind von unschätzbarem Wert, um die Anwendung so zu gestalten, dass sie den Herausforderungen der Praxis gerecht wird.

Geplant ist, eine qualitative Studie durchzuführen, bei der Sie die Möglichkeit haben, die Anwendung in einer Videokonferenz unter der Verwendung von Bildschirmfreigabe auszutesten. Dabei werden praxisnahe Problemstellungen bearbeitet, um ein Verständnis für die Funktionsweise und den Nutzen der Anwendung zu entwickeln. Dies bietet auch die Gelegenheit, eventuelle Fehler zu identifizieren und durch Ihr Feedback und Ihre Verbesserungsvorschläge direkt zur Weiterentwicklung der Anwendung beizutragen.

Mit Ihrer Teilnahme können Sie die Entwicklung der Anwendung beeinflussen, und somit dafür sorgen, dass sie einfacher zu benutzen ist und den Anforderungen der Realität gerechnet wird.

Für weitere Informationen und bei Interesse an einer Teilnahme kontaktieren Sie mich bitte unter \texttt{\href{mailto:joshua.jeschek@simplex4data.de}{joshua.jeschek@simplex4data.de}}. Ich würde mich über ihre Teilnahme freuen.

\begin{flushright}
  Mit freundlichen Grüßen,

  Joshua Jeschek

  \vspace{\baselineskip}

  TU Chemnitz\ifdefined\STANDALONE\\\else{}; \fi
  Simplex4Data GmbH

  März 2024
\end{flushright}

\ifdefined\STANDALONE
  \vfill
  \begin{center}
    \includegraphics[scale=.75]{../assets/tuc.pdf}
  \end{center}
\fi
