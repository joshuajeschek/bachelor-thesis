\ifdefined\STANDALONE\else
  \chapter{Handout Usability Studie}
  \label{app:handout}
\fi

\subsection*{Szenario 1: Import von Bundesländern}
\href[pdfnewwindow=true]{https://db01.simplex4data.de:444/develop/joshua/cql/?functions=/develop/joshua/simplexservice/functions&queryables=/develop/joshua/simplexservice/scenarios/1/collections/14-1-103/queryables&output_class=/develop/joshua/simplexservice/scenarios/1/collections/14-1-104/queryables}{Startpunkt Szenario 1}

\vspace{\baselineskip}\noindent
Gegeben ist eine Importtabelle mit Daten zu den Bundesländern. Sie wollen diese in das Simplex4Data System importieren. Dazu existiert bereits eine Objektklasse mit den nötigen Attributen.

\vspace{\baselineskip}\noindent
Die Importtabelle sieht in etwa so aus:

\begin{flushleft}
  \begin{tabular}{||c | c | c | c | c | c | c | c | c ||}
    \hline
    bezeichnung & name      & ars    & nuts   & nbd    & ibz    & bemerkung & wirksamkeit & geom       \\ [0.5ex]
    \hline\hline
    'Freistaat' & 'Sachsen' & '14'   & 'DED'  & 'ja'   & 20     & '--'      & 2014-02-01  & <geometry> \\
    \hline
    \vdots      & \vdots    & \vdots & \vdots & \vdots & \vdots & \vdots    & \vdots      & \vdots     \\
    \hline
  \end{tabular}
\end{flushleft}

\vspace{\baselineskip}\noindent
Die Objektklasse enthält folgende Attribute:
\begin{flushleft}
  \begin{tabular}{ || l || }
    \hline
    Bundesland            \\
    \hline
    key - ARS             \\
    typ - Bezeichnung     \\
    title - Name          \\
    cmt - Bemerkung       \\
    beg - Wirksamkeit     \\
    nuts - NUTS           \\
    nbd - NBD             \\
    ibz - IBZ             \\
    fl - Flächengeometrie \\
    \hline
  \end{tabular}
\end{flushleft}

\vspace{\baselineskip}\noindent
Weisen Sie die Daten aus der Importtabelle den korrespondierenden Attributen aus der Objektklasse zu, und überprüfen Sie das Resultat.

\clearpage

\subsection*{Szenario 2: Import von Gemeinden}
\href[pdfnewwindow=true]{https://db01.simplex4data.de:444/develop/joshua/cql/?functions=/develop/joshua/simplexservice/functions&queryables=/develop/joshua/simplexservice/scenarios/1/collections/14-1-100/queryables&output_class=/develop/joshua/simplexservice/scenarios/1/collections/14-1-105/queryables&filter=on}{Startpunkt Szenario 2}

\vspace{\baselineskip}\noindent
Analog zu den Bundesländern sollen Gemeinden importiert werden. Die Daten in der Quelltabelle liegen nicht einfach zur Zuordnung bereit, sondern müssen teilweise umgeformt werden.

\noindent Die Quelltabelle enthält auch Einträge, die sich nicht auf Gemeinden beziehen. Diese
können daran erkannt werden, dass \texttt{gemeinde\_id} nicht angegeben ist, und sollen nicht importiert werden.

\vspace{\baselineskip}\noindent
Hier ein Auszug aus der Quelltabelle:
\begin{flushleft}
  \begin{tabular}{|| c | c | c | c | c | c | c | c ||}
    \hline
    gemeindename  & breitengrad & \dots  & bevoelkerung & flaeche & gemeinde\_id & \dots  & land\_id \\ [0.5ex]
    \hline\hline
    'Saarbrücken' & 49.236608   & \dots  & 179634       & 167.52  & '10'         & \dots  & '10'     \\
    \hline
    \vdots        & \vdots      & \vdots & \vdots       & \vdots  & \vdots       & \vdots & \vdots   \\
    \hline
  \end{tabular}
\end{flushleft}

\vspace{\baselineskip}\noindent
Die Objektklasse enthält folgende Attribute:
\begin{flushleft}
  \begin{tabular}{ || l || }
    \hline
    Gemeinde                                        \\
    \hline
    title - Name                                    \\
    key - ARS                                       \\
    bevoelkerung\_je\_km2 - Bevölkerung je km2      \\
    bevoelkerung\_weiblich - Bevölkerung weiblich   \\
    bevoelkerung\_männlich - Bevölkerung männlich   \\
    bevoelkerung\_insgesamt - Bevölkerung insgesamt \\
    flaeche\_km2 - Fläche in km2                    \\
    pt - Punktgeometrie                             \\
    \hline
  \end{tabular}
\end{flushleft}

\vspace{\baselineskip}\noindent
Der ARS (Amtlicher Regionalschlüssel) setzt sich wie folgt zusammen:
\[
  \underbrace{\scalebox{3}{10}}_{\texttt{land\_id}}\overbrace{\scalebox{3}{0}}^{\texttt{bezirk\_id}}\underbrace{\scalebox{3}{41}}_{\texttt{kreis\_id}}\overbrace{\scalebox{3}{0100}}^{\texttt{verband\_id}}\underbrace{\scalebox{3}{10}}_{\texttt{gemeinde\_id}}
\]

\clearpage

\subsection*{Szenario 3: Auflistung der Adressen in einem Leipziger Ortsteil}
\href[pdfnewwindow=true]{https://db01.simplex4data.de:444/develop/joshua/cql/?functions=/develop/joshua/simplexservice/functions&queryables=/develop/joshua/simplexservice/scenarios/1/collections/1-1-119/queryables?joins=%3E1-1-119:1-3-121-obj-0:nn,%3C1-3-121-obj-0:1-1-108-obj-strasse,%3E1-1-119:1-3-120-obj-1:nn,%3C1-3-120-obj-1:1-1-103-obj-ortsteil&filter=on}{Startpunkt Szenario 3}

\vspace{\baselineskip}\noindent
In diesem Szenario sind die Objektklassen bereits importiert. Dabei handelt es sich um:
\begin{itemize}
  \item Adressen (Hausnummer, PLZ, Geometrie)
  \item Straßen (Straßenname)
  \item Ortsteile (Ortsteil-Name)
\end{itemize}

\vspace{\baselineskip}\noindent
Die Klassen sind miteinander verknüpft und sollen nun zu einer Übersicht aller Adressen \textbf{im Ortsteil "Lindenau"} zusammengestellt werden.

\vspace{\baselineskip}\noindent
Die Liste sollte am Ende folgende Informationen enthalten:
\begin{itemize}
  \item Straßenname
  \item Hausnummer
  \item Adresszusatz
  \item Postleitzahl
  \item Ortsteil-Name
  \item Punktgeometrie(n)
\end{itemize}
