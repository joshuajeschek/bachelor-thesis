\ifdefined\STANDALONE
  \section*{Skript für Moderation der Usability-Studie}
\else
  \chapter{Skript für die Moderation der Usability-Studie}
  \label{app:moderation}
\fi

\subsection*{Begrüßung}
\subsection*{Motivation und Zielstellung erklären}
\begin{itemize}
  \item Anwendung greift an zwei Stellen des Simplex Systems an
        \begin{itemize}
          \item Import \textrightarrow{} Rohdaten in die interne Struktur (Importtabellen \textrightarrow{} Objektklassen)
          \item Erstellen von SimplexSzenarios, neue Sicht auf bereits importierte Daten (Dabei können mehrere Objektklassen miteinander kombiniert werden, insofern verknüpft) (Objektklassen \textrightarrow{} Szenarios)
        \end{itemize}
  \item Ziel ist es, den Prozess ohne Tippfehler und häufiges Nachschlagen zu bewältigen
  \item Auch für Personen die nicht bewandert in SQL etc. sind
  \item Auch mit komplexeren Operationen an den Daten, als 1:1 Übernahme von Quelle zu Ziel
\end{itemize}

\subsection*{Erklärung des Vorgehens}
\begin{itemize}
  \item schon mal an Usability-Studie teilgenommen?
  \item Szenarios erklären (erkläre Situation, Zielstellung, Sie versuchen diese zu erreichen)
  \item Handout weniger als Aufgabenblatt, sondern mehr als Gedankenstütze für uns beide, dort sind die wichtigsten Fakten nochmal vermerkt. Aber an sich erzähle ich alles. (Dort befinden sich auch die Links zu den Startpunkten der einzelnen Szenarios)
  \item Screen-Sharing
  \item Thinking Aloud / lautes Nachdenken erklären
        \begin{itemize}
          \item Während der Benutzung der Anwendung so viele Gedanken wie möglich teilen
          \item weniger Spekulation meinerseits
          \item Sehr hilfreich gleich zu hören, was Sie frustriert oder verwirrt, aber auch was sie gut finden oder was sie erwarten bevor sie etwas tun / erwartet hätten danach
          \item fühlt sich vielleicht nicht sofort natürlich an, aber so können wir auch mehr Informationen sammeln als wenn wir im Nachinein darüber reden.
          \item Beispiele: "Ich mag das, weil...", "Das hab ich nicht als Reaktion auf meinen Klick erwartet, sondern..."
          \item Kann auch sein dass ich an bestimmten Stellen nachhake
        \end{itemize}
  \item Szenarios sind dafür gedacht die Anwendung zu testen nicht Sie.
\end{itemize}

Fragen?

\ifdefined\STANDALONE
  \clearpage
\fi
\subsection*{Vor Beginn des ersten Szenarios}
\begin{itemize}
  \item Was ist ihr Eindruck?
  \item Können Sie beschreiben was Sie über die Oberfläche in diesem Zustand denken? So können wir auch schon das Thinking Aloud austesten.
  \item Benennung von Bereichen in der Oberfläche
\end{itemize}

\subsection*{Szenarios}
\begin{enumerate}
  \item Import Bundesländer
  \item Import Gemeinden
  \item[] DATENBANK UMSTELLEN
  \item SimplexSzenario Adressen Leipzig
\end{enumerate}

\subsection*{Nach jedem Szenario}
\begin{itemize}
  \item SEQ: Auf einer Skala von 1 bis 7, wie einfach oder schwer fanden Sie es, die Aufgabe zu absolvieren? (1 sehr schwer, 7 sehr einfach)
  \item darauf basierend Möglichkeit auf Probleme einzugehen
  \item Feedback zu Thinking Aloud
  \item Liegt noch was auf dem Herzen?
\end{itemize}

\subsection*{Nach den Szenarios}
\begin{itemize}
  \item Haben Sie sich von der Anwendung in der Erfüllung der Aufgaben unterstützt gefühlt?
  \item Was fanden Sie am Schlechtesten?
  \item Was fanden Sie am Besten?
  \item Bewerten Sie die Einfachheit der Benutzung auf einer Skala von 1 bis 10. (1 ist sehr schwer, 10 ist sehr einfach.)
  \item Falls das alte System bekannt ist, wie würden Sie dieses auf der gleichen Skala einschätzen?
  \item Fühlen Sie sich produktiver mit dieser Oberfläche, oder finden Sie diese zufriedenstellender?
  \item Halten Sie diese Oberfläche, sobald sie in Simplex4Data integriert ist, als eine sinnvolle und nützliche Erweiterung?
  \item Haben Sie noch weitere Vorschläge für die Erweiterung und Verbesserung der Anwendung, wie können wir sie nützlicher gestalten?
  \item Wie würden Sie ähnliche Aufgaben wie die hier betrachteten Aufgaben sonst angehen?
\end{itemize}

\subsection*{Vielen Dank für Ihre Zeit und nützliche Hinweise}
